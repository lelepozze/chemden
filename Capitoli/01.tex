\chapter*{Metodi fisici}

Per i chimici è essenziale essere capaci di determinare la struttura dei composti con cui lavorano. Senza la conoscenza della struttura, i chimici non possono progettare il modo di sintetizzare il composto e non possono nemmeno intraprendere gli studi per approfondirne l'attività biologica. Affinché la struttura di un composto possa essere determinata, il composto deve essere prima isolato e successivamente analizzato

Un tempo l'identificazione di un composto organico si basava sulla determinazione della sua formula molecolare mediante analisi elementare, sulla determinazione di sue proprietà fisiche e su semplici test chimici che indicavano la presenza di determinati gruppi funzionali.
Queste procedure non erano comunque sufficienti a caratterizzare molecole con strutture complesse ed inoltre serviva una grande quantità di prodotto

Al giorno d'oggi per identificare i composti organici si usano una serie di tecniche strumentali. Queste sono veloci e richiedono piccole quantità di materia. Inoltre forniscono molte più informazioni sulla struttura rispetto ai test chimici
Le tecniche strumentali sono:
\begin{itemize}
\item La spettrometria di massa (MS)
\item La spettroscopia infrarossa (IR)
\item La spettroscopia di risonanza magnetica nucleare (NMR)
\end{itemize}

La spettrometria di massa permette di determinare la massa molecolare e la formula molecolare di un composto, come alcune sue caratteristiche strutturali. La spettroscopia infrarossa permette di determinare quali gruppi funzionali sono presenti nel composto. La spettroscopia di risonanza magnetica nucleare permette di determinare la connettività della molecola e la disposizione degli atomi di idrogeno nella molecola
La spettrometria di massa è l'unica tecnica che non utilizza radiazione elettromagnetica.

La spettroscopia è lo studio dell'interazione della materia con la radiazione elettromagnetica. Questa è energia radiante che ha un comportamento sia di particella che di onda. L'insieme dei diversi tipi di radiazione elettromagnetica costituisce lo spettro elettromagnetico; ad un particolare intervallo di frequenza si associa una certa energia.

La particella di radiazione elettromagnetica è detta \emph{fotone}. Tuttavia, la radiazione si comporta anche come onda. È possibile definire delle grandezze proprie delle onde, quali la frequenza (\nu) e la lunghezza d'onda ($\hat{\nu}$). La velocità di propagazione della radiazione elettromagnetica è detta c e vale circa 3·10\ap{8} m/s.

Le classi di composti organici sono descritte in tabella \ref{tab:classi}

\begin{table}
\setlength\extrarowheight{8pt}
\begin{tabular}{lclc}
Nome & Struttura & Nome & Struttura\\
Alcani & \chemfig[atom sep=1.5em]{C([2]-)([4]-)([6]-)([8]-)} & Aldeidi & \ce{RCHO}\\
Alcheni & \chemfig[atom sep=1.5em]{C([:120]-)([:-120]-)=C([:60]-)([:-60]-)} & Chetoni & \ce{RCOR}\\
Alchini & \chemfig[atom sep=1.5em]{-C~C-} & Acidi carbossilici & \ce{RCOOH}\\
Nitrili & \chemfig[atom sep=1.5em]{-C~N} & Esteri & \ce{RCOOR}\\
Alogenuri alchilici & \ce{RX} & Ammidi & \ce{RCONH2}\\
Eteri & \ce{ROR} & & \ce{RCONHR}\\
Alcoli & \ce{ROH} & & \ce{RCONR2}\\
Fenoli & \ce{ArOH} & Ammine (primarie) & \ce{RNH2}\\
Aniline & \ce{ArNH2} & Ammine (secondarie) & \ce{R2NH}\\
 & & Ammine (terziarie) & \ce{R3N}\\
\end{tabular}
\caption{Classi di composti organici}
\label{tab:classi}
\end{table}












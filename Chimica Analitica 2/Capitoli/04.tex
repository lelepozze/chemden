\chapterpicture{header_06}
\chapter{Tecniche spettroscopiche}
Le tecniche spettroscopiche si basano sull'interazione radiazione elettromagnetica-materia. Esse possono essere:
\begin{itemize}
\item \textit{Tecniche atomiche}: tecniche che prevedono l'analisi di atomi. Il campione deve essere trattato per produrre elementi
\item \textit{Tecniche molecolari}: tecniche che analizzano la molecola. La molecola, in questo caso, va preservata
\end{itemize}

\marginpicture{04_001}{Radiazione elettromagnetica}{}

Secondo la meccanica quantistica, le onde elettromagnetiche hanno un comportamento sia corpuscolare che ondulatorio. In questa trattazione, le onde elettromagnetiche verranno
considerate corpuscolari, in quanto si adattano meglio al al mondo microscopico

Le tre caratteristiche principali delle onde elettromagnetiche sono la lunghezza d'onda ($\lambda$), la frequenza ($\nu$) e l'ampiezza ($A$).
Una grandezza derivata è il numero d'onda, che si esprime in cm$^{-1}$.

La natura corpuscolare descrive la luce come un flusso di particelle, ovvero di fotoni. L'energia di un fotone dipende dalla frequenza e da $h$, ovvero la costante di Plank.
\[
E = h \cdot \nu
\]

Il modello corpuscolare può essere applicato a tutte le componenti della materia, sia atomi che molecole. Si possono quindi descrivere gli orbitali come funzioni d'onda.
I contributi dell'energia totale di un atomo sono molteplici:
\[
E_{tot} = E_{nuclei} + E_{elettroni} + E_{vibrazionale} + E_{rotazionale} + E_{traslazionale} 
\]

\paragraph{Energia traslazionale}
L'energia traslazionale è dovuta al movimento dell'atomo stesso. Quest'energia non è quantizzata, in quanto lo spazio in cui si muove la molecola è enorme rispetto alle sue dimensioni.
Questo comporta, di fatto, che i livelli quantici esistono,  ma sono così vicini tra loro che generano un continuo

\paragraph{Energia rotazionale}
L'energia rotazionale è associata alla rotazione delle molecole attorno un asse; può essere quantificabile ed è quantizzata, in quanto la rotazione è confrontabile con le dimensioni della molecola

\paragraph{Energia vibrazionale}
L'energia vibrazionale è dovuta a vibrazioni degli atomi, sia assiali che equatoriali. L'atomo singolo non può vibrare, ma una molecola può dare movimenti di stretching o bending.
Quest'energia è utilizzata nella spettroscopia IR e trova applicazione nell'identificazione dei gruppi funzionali, quindi è preferibile quando si deve effettuare un analisi qualitativa.

\halfpicture{04_002}{Livelli vibrazionali di uno stato elettronico}{}

\paragraph{Energia degli elettroni}
Quest'energia è più elevata rispetto alle altre, in quanto si è in grado di spostare elettroni da un orbitale all'altro. Inoltre, si può passare da un livello vibrazionale all'altro, per il principio di Frank-Condon.
In questa spettroscopia (UV-Vis) si possono iniziare a dare informazioni quantitative.

\section{Spettroscopia UV-Vis}
Nella spettroscopia di assorbimento, si ha l'assorbimento d un fotone, con livelli di energia pari alla differenzia di energia dei due livelli energetici.

\marginpicture{04_003}{Spettro della luce visibile}{}

Solitamente, per le molecole organiche, si hanno i seguenti livelli di energia, per le varie transizioni
\begin{itemize}
\item Transizione $\sigma \to \sigma^\ast$ 110 - 135 nm
\item Transizioni $\pi \to \pi^\ast$ e $n \to \sigma^\ast$ 180 - 255 nm
\item Transizione $n \to \pi^\ast$ > 285 nm
\end{itemize}

L'esperimento di assorbimento atomico è composto da una sorgente, un campione e un rivelatore. Si vede che $I_0$ è la luce iniziale, mentre $I$ è la luce dopo aver attraversato il campione.
Possibilmente, la luce che investe il campione deve essere monocromatica, ovvero con una singola lunghezza d'onda

La legge che regola questo fenomeno è la \emph{legge di Lambert-Beer}.
\[
A = \epsilon \cdot b \cdot C
\]
dove $\epsilon$ è il coefficiente di estinzione molare, $b$ è il cammino ottico, $C$ è la concentrazione dell'analita nel campione e $A$ è l'assorbanza.

Sia dato un volume di materia, con lunghezza $b$ e area laterale $S$. Se si irradia una luce, con potenza $P_0$, all'interno del volume di materia, si otterrà una potenza $P$.
La potenza $P_0$ si dissipa per assorbimento, diffusione (scattering) e riflessione. Questi contributi sono sottrattivi e sono diversi dall'assorbimento.

\marginpicture{04_005}{Fenomeni di assorbimento, diffusione e riflessione}{}

\halfpicture{04_004}{Luce che attraversa una cella}{}

La diminuzione di potenza infinitesimale è
\[
-\frac{dP}{P} = \frac{\alpha}{S} dn
\]
dove $dn$ è la variazione infinitesima delle molecole, $S$ è la superficie e $\alpha$ è il coefficiente di proporzionalità e può essere assimilato alla possibilità che un fotone venga assorbito dalla molecola.
Si integra questa legge
\[
-\int_{P_0}^P \frac{dP}{P} = \int_0^n \frac{\alpha}{S} dn \rightarrow -\int_{P_0}^P \frac{dP}{P} = \frac{\alpha}{S} \int_0^n dn  \rightarrow \ln \biggl(\frac{P_0}{P}\biggr) = \frac{\alpha}{S} \cdot n
\]

Si converte il logaritmo naturale in logaritmo decimale
\[
\ln \biggl(\frac{P_0}{P}\biggr) = \frac{\alpha}{S} \cdot n \rightarrow 2.303 \cdot \log \biggl(\frac{P_0}{P}\biggr) = \frac{\alpha}{S} \cdot n
\]

Si converte la superficie $S$ in volume $V$ e cammino ottico $b$
\[
2.303 \cdot \log \biggl(\frac{P_0}{P}\biggr) = \alpha n \cdot \frac{b}{V}
\]

Si converte il numero di particelle in funzione del numero di Avogadro $N_A$ e numero di moli
\[
2.303 \cdot \log \biggl(\frac{P_0}{P}\biggr) = \frac{\alpha b}{V} \cdot \text{mol} \cdot N_A
\]

Si trasforma il volume da cm$^3$ a dm$^3$, quindi si moltiplica per 1000.
\[
2.303 \cdot \log \biggl(\frac{P_0}{P}\biggr) = \alpha b N_A \cdot \text{mol} \cdot \frac{1}{V \cdot 1000}
\]

Riarrangiando, si ottiene
\[
\log \biggl(\frac{P_0}{P}\biggr) = \frac{\alpha}{V} \cdot \frac{n N_A b}{2.303 \cdot 1000}
\]

Raggruppando un po' di termini
\begin{align*}
& A = \log \biggl(\frac{P_0}{P}\biggr) = \log \biggl(\frac{1}{T}\biggr)\\
& \epsilon = \frac{\alpha N_A}{2.303 \cdot 1000}\\
& C = \frac{\text{mol}}{V} = M\\
\end{align*}

Si ottene la legge di Lambert-Beer
\[
A = \epsilon \cdot b \cdot C
\]

La legge di Lambert-Beer è la linearizzazione di un logaritmo decimale, quindi forma una retta. Apparentemente, l'assorbanza sembra lineare, ma non lo è.
Il processo vero e proprio è un esponenziale negativo
\[
P = P_0 \cdot 10^{-\epsilon b C} \quad T = \frac{P_0}{P} = 10^{-\epsilon b C}
\]
Se la concentrazione è molto elevata. l'intensità va praticamente subito a zero.

\marginpicture{04_006}{Attenuazione della luce, secondo la legge di Lambert-Beer}{}

Affinché la legge di Lambert-Beer sia valida, è necessario che:
\begin{itemize}
\item Ci sia un fascio di luce monocromatico
\item Il solvente non deve assorbire la luce
\item Non ci siano reazioni chimiche. Questo non è necessario, però è più difficile interpretare i risultati.
\end{itemize}

La legge di Lambert-Beer viene utilizzata per quantificare una sostanza. Il grafico ottenuto viene detto \emph{spettro di assorbimento}.
Da questo grafico, si può ricavare la lunghezza d'onda migliore per determinare l'analita incognito. Si prende la lunghezza d'onda con il massimo assorbimento $\lambda_{max}$.

Lo spettro di assorbimento prevede che ci sia una banda.
Questo perché sono presenti dei livelli rotazionali e vibrazionali per un dato stato elettronico e questo crea un continuo di transizioni possibili.

La quantificazione di una sostanza si può fare quando si ottiene $\lambda_{max}$; questo consente di avere una pendenza elevata nella retta della legge di Lambert-Beer.
Si ricordi che più alta è la pendenza della retta di calibrazione e più sensibile è la misura.

Per ottenere la retta di calibrazione, si trova l'assorbanza di alcune soluzioni a titolo noto e si costruisce la retta per punti. L'andamento è lineare

La legge di Lambert-Beer è valida solo in un intervallo, infatti al crescere della concentrazione, si verificano notevoli deviazioni dalla linearità.
Questo avviene perché, come visto prima, il processo di assorbimento non è un processo lineare.
Se non c'è variazione di assorbanza, allora non è possibile determinare cambiamenti di concentrazione.
Il dominio di questa retta è molto limitato; si possono misurare assorbanze da 10$^{-2}$ A fino a 1.5 A (e fino a 4 A con strumenti di ultima generazione)

Oltre alle cause intrinseche della misura, vi sono anche delle cause chimico-fisiche.
L'alta concentrazione, oltre a non permettere di analizzare la soluzione, può causare la formazione di aggregati di molecole, che alterano lo spettro di assorbimento.
Inoltre, sempre a causa dell'alta concentrazione, si possono formare equilibri chimici a carico delle specie di analita.

L'assorbanza dipende anche dal coefficiente di estinzione molare $\epsilon (\lambda)$. Un valore ottimale di $\epsilon (\lambda)$ è 10$^4$, in quanto per una concentrazione $10{^-4}$ e un cammino ottico di 1 cm, l'assorbanza è pari a 1.

\section{Strumentazione}

Gli spettrofotometri UV-Vis possono essere a singolo o a doppio raggio.
Uno strumento a singolo raggio presenta una sorgente, un monocromatore, il campione e il rivelatore. L'elemento disperdente del monocromatore può essere un prisma o un reticolo
La lunghezza d'onda della luce non potrà mai essere monocromatica, tuttavia il monocromatore può scomporre la luce in un intervallo di lunghezze d'onda estremamente ristretto.
La trasmittanza si determina poiché la potenza emessa dalla lampada è nota.

\halfpicture{04_007}{Spettrofotometro a doppio raggio}{}

Nello strumento a doppio raggio, a differenza di quello a singolo raggio, la luce viene divisa in due. Un raggio va nella cella del campione, mentre l'altro va verso una cella contenente il riferimento.
Il riferimento solitamente è il solvente senza il campione. In questo modo, la differenza di assorbanza è l'assorbanza del campione senza quella del solvente.
Nello strumento a singolo raggio, viene prima misurato il bianco, ovvero il solvente senza l'analita e poi viene sottratto dai campioni

\subsection{Sorgenti}
Le sorgenti utilizzate sono:
\begin{itemize}
\item \textit{Lampada al deuterio}: questa lampada emette da 160 nm a 380 nm; è quindi una sorgente UV. La lampada combina due atomi di deuterio per formare deuterio molecolare.
La formazione del legame libera energia.
\item \textit{Lampada al tungsteno}: questa lampada emette da 350 nm a 2200 nm; è quindi una sorgente Vis.
\item \textit{Lampada a xeno}: è la lampada più utilizzata in quanto emette tra 180 nm e  800 nm. Inoltre presenta una stabilità di intensità rispetto alla lunghezza d'onda
\end{itemize}

Una buona sorgente emette energia in modo costante nel tempo e emette, idealmente, la stessa energia per tutte le lunghezze d'onda.
Ci possono essere delle variazioni di intensità dovute a:
\begin{itemize}
\item Sublimazione dei materiali dei filamenti
\item Alimentazione non costante nel tempo
\item Periodo di preriscaldamento insufficiente
\end{itemize}

\subsection{Monocromatore}
Ci sono tre tipi di monocromatore: a filtro, a prisma, e a reticolo di diffrazione.
A differenza dei filtri, i prismi e i reticoli possono selezionare le lunghezze d'onda della luce in modo continuo

\paragraph{Filtro semplice}
I filtri trasmettono solo solo un range di lunghezze d'onda, sopra/sotto un certo valore.
La selezione della lunghezza d'onda perciò sarà scarsa, ma per ovviare a questo si possono mettere più filtri in serie.
La banda passante sarà dai 30 ai 250 nm e la trasmittanza dell'ordine del 5 - 30 \%

\paragraph{Filtri a interferenza}
Il filtro a interferenza è formato da una serie di materiali deposti in serie.
Uno strato dielettrico è racchiuso tra due strati di pellicola metallica semitrasparente, a loro volta racchiusi in di lamine di vetro.
La luce passa attraverso lo strato metallico e viene deviata in modo tale da entrare e uscire continuamente dallo strato dielettrico.

\marginpicture{04_008}{Filtro a interferenza}{}

L'interferenza si crea tra i raggi riflessi, e serve per far passare solo una specifica lunghezza d'onda.
La lunghezza d'onda massima è
\[
\lambda_{max} = \frac{2 d n}{N}
\]
dove $n$ è l'indice di rifrazione del materiale, $d$ è la distanza percorsa dalla luce all'interno del tubo e $N$ è il numero di volte che la luce viene riflessa.

La distanza percorsa dalla luce può essere espressa come:
\[
d = \frac{2t}{\cos \theta} = n \lambda'
\]
dove $\lambda'$ è la lunghezza d'onda della radiazione del materiale dielettrico

Per $\theta$ molto piccolo, $\cos \theta \to 1$, quindi
\[
d = 2t = n \lambda'
\]

Quindi la lunghezza d'onda trasmessa è
\[
\lambda = \frac{2 t n d}{n}
\]

Esistono anche dei selettori di lunghezza d'onda angolari , che vengono chiamati filtri a cuneo


\paragraph{Prismi}
Il prisma riesce a scomporre la luce nelle sue componenti a causa del fenomeno chiamato \emph{dispersione}.
Il potere dispersivo del prisma è dipendente dalla lunghezza d'onda.
Il materiale di cui è composto il prisma è solitamente vetro (per la luce visibile) o quarzo (per l'UV).

Come accennato prima, la dispersione è dipendente dalla lunghezza d'onda; quindi non è lineare.
Il potere dispersivo del mezzo ottico ($\omega$) viene calcolato come
\[
\omega = \frac{\delta_F - \delta_C}{\delta_D}
\]
La differenza al numeratore indica l'ampiezza dell'angolo di dispersione totale della luce, mentre $\delta_D$ indica la dispersione media.

\halfpicture{04_010}{Scomposizione della luce con un prisma}{}

In ambito ottico, si preferisce esprimere il potere dispersivo con numero di Abbe, che è il reciproco del potere dispersivo
\[
\nu = \frac{1}{\omega} = \frac{n_D - 1}{n_F - n_C}
\]
dove $n_D$, $n_F$ e $n_C$ sono gli indici di rifrazione, rispettivamente, medio, massimo e minimo.

\subsubsection{Reticoli}
I reticoli sono costruiti con una superficie opportunamente scanalata con rette parallele ed equidistanti e ricoperte da un materiale riflettente.

\halfpicture{04_011}{Funzionamento di un reticolo a diffrazione}{}

Il reticolo funziona per il principio fisico della diffrazione.
Il raggio più a sinistra percorre più strada rispetto al raggio al centro, che tuttavia vengono rifratti con lo stesso angolo.
La differenza di percorso dei due raggi è
\[
\Delta = CD - AB
\]

Se $\Delta$ è multiplo della lunghezza d'onda del raggio, si creerà una differenza costruttiva, altrimenti sarà distruttiva
\[
CD = d \cdot \sin i
\]
\[
AB = - d \sin r
\]
Il segno meno indica che la diffrazione avviene con angolo opposto.

Se si è in presenza di interferenza costruttiva
\[
n \lambda = CD - AB \quad \text{con n intero}
\]

Sostituendo, si ottiene
\[
n \lambda = d \cdot (\sin i + \sin r)
\]

Con accorgimenti ottici, è possibile far passare la radiazione con $n = 1$, ottenendo quindi una radiazione monocromatica.

La dispersione del reticolo è la capacità di separare le lunghezze d'onda in modo opportuno. Ci sono due tipi di dispersione, ovvero quella angolare e quella lineare
La dispersione angolare è 
\[
D_a = \frac{dr}{d\lambda}
\]
Questa dispersione è la variazione dell'angolo in funzione della variazione di lunghezza d'onda

Mentre la dispersione lineare è
\[
D_l = \frac{dy}{d\lambda}
\]
Questa dispersione indica la variazione della distanza del piano focale rispetto alla variazione della lunghezza d'onda.

Si passare da una all'altra per mezzo della distanza focale $F$
\[
D = \frac{dy}{d\lambda} = F \cdot \frac{dr}{d\lambda}
\]
Si può anche utilizzare il suo reciproco $D^{-1}$
\[
D^{-1} = \frac{d\lambda}{dy} = \frac{1}{F} \frac{d\lambda}{dr}
\]

La dispersione angolare può essere ottenuta differenziando rispetto a $r$
\[
n \lambda = d (\sin i + \sin r) \longrightarrow n d\lambda = d \cos r dr
\]

Allora $D_a$ e $D^{-1}$ sono 
\[
D_a = \frac{dr}{d\lambda} = \frac{n}{d \cos r} \qquad D^{-1} = \frac{d \lambda}{dy} = \frac{d \cos r}{n F} 
\]
 
La dispersione angolare aumenta al diminuire della distanza tra le scanalature ($d$), inoltre per piccoli angoli $\cos r \to 1$, quindi $D^{-1}$ diventa
\[
D^{-1} \approx \frac{d}{n F} 
\]

Si può dire che la dispersione di un reticolo è costante, o quasi, con la lunghezza d'onda della luce.
Quindi, per un dato piano focale, le lunghezze d'onda si distribuiranno uniformemente.

Il potere risolutivo dei monocromatori è
\[
R = \frac{\lambda}{\Delta \lambda}
\]

Nel monocromatore sono presenti anche delle fenditure per selezionare meglio la lunghezza d'onda.
Più la luce si avvicina a essere monocromatica, più la situazione tende all'idealità, secondo la legge di Lambert-Beer

\subsection{Rivelatori}
Il primo dispositivo visto è il fotomoltiplicatore, che è costituito da una serie di diodi che sfruttano l'effetto fotoelettrico.
Si verifica un effetto a cascata, in quanto l'elettrone eccitato dalla luce va a colpire un secondo elettrodo, che a sua volta libera elettroni.
L'effetto a cascata può essere così grande da arrivare a un milione di volte

\marginpicture{04_012}{Funzionamento di un tubo fotomoltiplicatore}{}

Un altro rivelatore è l'array di diodi, che analizza tutte le lunghezze d'onda contemporaneamente.
La luce comunque deve essere scomposta dal monocromatore.
Ogni zona fotosensibile viene quindi colpita con una lunghezza d'onda diversa.
Con questo rivelatore, lo spettro viene raccolto molto velocemente e per questo viene molto utilizzato in cromatografia, dove il campione è flussato.

\halfpicture{04_013}{Funzionamento di un array di diodi}{}

\section{Fluorescenza}
Quando un oggetto assorbe una radiazione elettromagnetica, si eccita. Per diseccitarsi, può liberare calore, oppure può emettere un fotone. Questo fenomeno si chiama \emph{emissione}.
La lunghezza d'onda del fotone emesso cambia, in quanto gli stati energetici coinvolti cambiano.
Le molecole fluorescenti sono molecole che possiedono un'elevata coniugazione $\pi$, e sono quindi strutture planari.
L'emissione presenta dei parametri:
\begin{itemize}
\item \textit{Spostamento di Stokes}: le energie di emissione sono generalmente più basse rispetto a quelle di assorbimento
\item \textit{Regole di Kasha}: la lunghezza di eccitazione non non cambia
\item \textit{Regola dell'immagine speculare}: lo spettro di emissione è speculare a quello di eccitazione
\end{itemize}

L'emissione di dipende dalla concentrazione di analita, dall'efficienza di assorbimento della radiazione e dalla resa quantica di fluorescenza.
Quest'ultima può essere espressa come
\[
\Phi = \frac{\text{fotoni}_{emessi}}{\text{fotoni}_{assorbiti}}
\]

L'emissione permette di determinare concentrazioni più basse rispetto all'assorbimento e permette inoltre di riconoscere le molecole sulla base dello spettro di fluorescenza

\halfpicture{04_014}{Differenza tra l'assorbimento e l'emissione}{}

\subsection{Strumentazione}
La strumentazione per la fluorescenza è differente da quella dell'assorbimento. In figura \ref{fig:sbemissione} è rappresentato lo schema a blocchi per uno strumento di emissione.
Si noti come sono presenti due monocromatori, uno di eccitazione e uno di emissione; il monocromatore di emissione è posto a 90 \degree da quello di eccitazione.

\fullpicturelab{04_015}{Schema di uno strumento che lavora in emissione}{}{sbemissione}

Si noti come la cella del campione diventi una sorgente addizionale e può essere descritta come una sorgente puntiforme che emette in tutte le direzioni.

\section{Spettroscopia atomica}
Nella spettroscopia atomica, l'analita è presente come atomo isolato, non come molecola; i salti di energia saranno molto diversi.
Lo spettro ricavato dalla spettroscopia atomica presenta delle righe, dette \emph{righe di assorbimento}. Le righe sono più sottili rispetto alle bande.

\halfpicture{04_016}{Righe di emissione}{}

Gli spettri di emissione presentano delle righe, dette \emph{righe di emissione}. Questo avviene perché un atomo più emettere solo poche radiazioni, se eccitato.
Quindi, si vede che le righe atomiche sono specifiche per un atomo; è possibile utilizzarle per quantificarlo, tramite la legge di Lambert-Beer.

La larghezza della riga è data dalla probabilità del processo di emissione (o di assorbimento).
La probabilità che la transizione avvenga è
\[
\int_{-\infty}^{+\infty} g(\nu) d\nu = 1 \quad \text{con} \quad g(\nu) \: \text{normalizzata}
\]
Un fotone di energia $h \nu$ potrà stimolare l'emissione di un fotone di energia compresa tra $h \nu$ e $h (\nu + d\nu)$.

Ci sono tre parametri che influenzano la larghezza della riga:
\begin{itemize}
\item Effetto della pressione
\item Effetto Doppler
\item Principio di indeterminazione di Heisenberg
\end{itemize}

Solitamente, per valutare l'allargamento della banda, si guarda la FWHM, ovvero la larghezza a metà altezza.
Si vede quindi che la FWHM è determinata da tre componenti:
\[
FWHM = FWHM_{Heisemberg} + FWHM_{Pressione} + FWHM_{Doppler}
\]
Questi effetti fanno si che le righe siano larghe tipicamente 10$^-3$ nm.
\paragraph{Pressione}
Se un atomo emette fotoni e viene urtato da un altro atomo, produce fotoni con frequenza differente; avviene quindi un allargamento della banda spettrale.
L'allargamento della banda aumenta anche all'aumentare della temperatura, in quando, come per la pressione, aumenta il numero di urti.
Questo allargamento viene anche definito 'allargamento di Lorentz'

\paragraph{Effetto Doppler}
Se gli atomi sono in movimento quanto emettono, producono un emissione che risente dell'effetto Doppler.
Si vede che sono possibili due fenomeni:
\begin{itemize}
\item Redshift: si ha uno spostamento della frequenza verso frequenze più basse
\item Blueshift: si ha uno spostamento della frequenza verso frequenze più alte
\end{itemize}

\marginpicture{04_017}{Effetto Doppler}{}

L'effetto Doppler mostra che la frequenza assume la forma
\[
\nu' = \nu \cdot \biggl(1 \pm \frac{v(x)}{c}\biggr)
\]
L'ultima parte del fattore correttivo è dovuto alla velocità effettiva rispetto a quella della luce

\paragraph{Principio di indeterminazione di Heisenberg}
L'unica temperatura dove il sistema è statico è 0 K. Questo effettivamente causa un allargamento trascurabile del picco.
Il principio di indeterminazione è
\[
\Delta \nu \cdot \Delta t > 1
\]

A seconda di che interazione il vapore atomico subisce, si possono avere diverse tecniche spettroscopiche:
\begin{itemize}
\item \textit{Assorbimento atomico}: il vapore viene irraggiato con una radiazione monocromatica che può essere assorbita sono da atomi di un determinato elemento
\item \textit{Emissione atomica}: il vapore atomico viene eccitato; il surplus di energia viene quindi emesso sottoforma di radiazione luminosa, che è specifica dell'elemento
\end{itemize}

Le tecniche associate sono descritte nella tabella \ref{tab:spettratomica}

\begin{table}
\begin{tabular}{lccc}
Atomizzazione & Temperatura (\degree C)& Tecnica & Abbreviazione\\
Fiamma & 1\,700 - 3\,150 & A,E,F & AAS, AES e AFS\\
Elettrotermica & 1\,200 - 3\,000 & A,F & ET-AAS e ET-AES\\ 
Plasma ad accoppiamento induttivo & 6\,000 - 8\,000 & E,F & ICP e ICP-OES\\
Plasma a DC & 6\,000 - 10\,000 & E & DCP\\
Arco elettrico & 4\,000 - 5\,000 & E & \\
Scintilla & 4\,000 (?) & E  &\\
\caption[Varie tipologie di spettroscopia atomica]{Si noti che A sta per assorbimento, E per emissione e F per fluorescenza}
\label{tab:spettratomica}
\end{tabular}
\end{table}

\section{Spettrofotometria di assorbimento atomico}
Il campione è portato allo stato di nuvola atomica tramite riscaldamento. La nuvola atomica viene irradiata con luce, che viene emessa da una sorgente luminosa opportuna.
La riga emessa dalla lampada viene assorbita sono dagli atomi interessati.
L'assorbimento fornisce una risposta quantitativa (tramite la legge di Lambert-Beer), ma anche qualitativa (in base alla lunghezza d'onda dell'assorbimento)

\halfpicturelab{04_018}{Schema a blocchi di un assorbimento atomico}{}{AAblocchi}

Lo strumento deve essere in grado di atomizzare un elemento in specie neutre, non cariche, altrimenti i livelli energetici cambiano ed è necessario commisurare l'assorbimento all'emissione della lampada.
Le energie delle transizioni atomiche sono molto basse e se si impiegasse una lampada continua non sarebbe possibile vedere i picchi in quanto l'atomo è in grado di assorbire solamente una minima parte di quella energia.
Ecco quindi che la lampada non deve essere una sorgente continua come nel caso della spettrofotometria UV-Vis, ma deve essere una sorgente non continua che emette le specifiche lunghezze d'onda richieste da quell'elemento, rendendo il rapporto $P/P_0$ misurabile.

Lo schema a blocchi dello strumento è rappresentato in figura \ref{fig:AAblocchi}.
Si nota che il monocromatore è posto dopo il campione, in quanto la fiamma può emettere altre lunghezze d'onda.
Questo schema è presente anche nell'ET-AAS, dove cambia il sistema di atomizzazione.

\halfpicture{04_019}{Differenza tra uno strumento a singolo raggio e a doppio raggio}{}

Come per lo spettrofotometro, lo strumento può essere a singolo raggio o a doppio raggio.
Nell'assorbimento atomico, lo strumento a singolo raggio può possedere una lampada continua, o una lampada a impulsi.
Lo strumento a impulsi permette di avere segnali più stabili e quindi più attendibili.
Nello strumento a doppio raggio, la luce viene divisa da un chopper, che fornisce anche un segnale a impulsi.
Il chopper è un disco con il 50 \% della superficie riflettente e l'altro 50 \% della superficie trasparente.

\subsection{Lampada a catodo cavo}
Si tratta di lampade speciali contenenti l'elemento da analizzare al catodo e un anodo di tungsteno; queste lampade emettono lo spettro a righe di un elemento specifico.
Sono riempite di un gas a bassissima pressione (tipicamente Ar) e la finestra cilindrica di uscita del fascio è in quarzo.
Si utilizza argon piuttosto che azoto, in quanto possiede alta energia di ionizzazione tale da eccitare tutti gli elementi che compongono il catodo.

Il meccanismo di emissione parte dalla ionizzazione dell'argon
\[
Ar \rightleftharpoons Ar^+ + e^-
\]

L'argon ionizzato migra (accelerando) verso il catodo, si schianta e assorbe elettroni per tornare neutro
\[
Ar^+ + e^- \rightleftharpoons Ar
\]
Nel processo viene liberata energia, sottoforma di calore.

\halfpicture{04_020}{Lampada a catodo cavo}{}

L'impatto consente di liberare gli atomi di cui è formato il catodo e quindi si forma una nuvola metallica. Questo processo si chiama sputtering.
Grazie al calore prodotto dall'impatto, gli atomi della nuvola metallica si eccitano e formano $M^\ast$.
L'atomo eccitato quindi ritorna allo stato fondamentale, emettendo una radiazione luminosa caratteristica dell'elemento di cui il catodo è fatto.

Le lampade a catodo cavo possono essere mono-elemento o multi-elemento, risultando queste ultime più pratiche, ma devono contenere elementi che non sovrappongono i propri spettri, che comunque i casi di
sovrapposizione degli spettri sono rarissimi nella tavola periodica.
L'assorbimento atomico permette di analizzare singolarmente gli elementi e vale ancora la Legge di Lambert-Beer.

Si vede che le righe di emissione della lampada a catodo cavo sono più sottili rispetto alle righe di assorbimento, in quanto nell'assorbimento si è in presenza degli effetti descritti prima, che causano l'allargamento della banda.

\halfpicture{04_021}{Dimostrazione del fenomeno dell'autoassorbimento}{}

La lampada a catodo cavo può presentare fenomeni di autoassorbimento.
La nuvola di atomi eccitati emette luce, tuttavia non è detto che tutti gli atomi della nube siano eccitati.
A volte gli atomi non eccitati assorbono la luce che viene emessa dalla nube.
Questo porta ad una perdita di emissione e comporta che le righe di emissione non siano più regolari, ma presentino degli avvallamenti.

\subsection{Atomizzatori}
Esistono tipi diversi di atomizzatori, ma il più diffuso è la fiamma lamellare.

\paragraph{Atomizzatore a fiamma lamellare}
Il cammino ottico è posto grande (solitamente 10 cm) per aumentare l’assorbanza e diminuire la concentrazione minima analizzabile.
La fiamma è solitamente alimentata con una miscela di acetilene e aria e sviluppa circa 2200 \degree C, ma possiede regioni a temperatura diversa.
La fiamma agisce in più fasi:
\begin{itemize}
\item Permette l'evaporazione istantanea del solvente.
\item Eliminazione di H$_2$O di coordinazione con formazione del composto anidro.
\item Zona redox di scambio tra catione e anione che compongono il sale con formazione dei corrispettivi atomi.
\item Possibilità di formazione di ioni.
\end{itemize}

Il flusso luminoso della lampada a catodo cavo oltrepassa la fiamma, dove avviene l'atomizzazione e il fenomeno di assorbimento e come rivelatore si utilizza un fotomoltiplicatore.
Normalmente una riga di emissione è più stretta di una di assorbimento per il fatto che la lampada e la fiamma sono in ambienti diversi e la fiamma produce più elevate agitazioni termiche.

\paragraph{Atomizzatore termo-elettrico}
Si pone una goccia di campione sulla piattaforma di L’Vov dal foro e il cilindro di grafite è circondato da spire che lo scaldano con un cambiamento di temperatura molto veloce con temperatura controllabile.
È possibile effettuare anche delle scansioni di temperatura e durante il funzionamento diventa incandescente al calor bianco risultando dannoso per gli occhi se osservato direttamente.

\marginpicture{04_022}{Atomizzatore termo-elettronico}{}

Questo sistema d atomizzazione presenta il vantaggio di produrre un segnale singolo per poco tempo, diversamente dalla fiamma che produce un segnale continuo per molto tempo.
È inoltre più sensibile in quanto il campione subisce una concentrazione producendo un segnale più grande e possedendo un più basso limite di rivelabilità, diversamente dalla fiamma, che diluisce il campione.
Tuttavia, presenta lo svantaggio di produrre segnali meno riproducibili.

\paragraph{Atomizzatore a vapori freddi}
Si utilizza per concentrazioni di analita bassissime e permette di aumentare l'efficienza di trasformazione da ionico ad atomico tramite l'impiego ad esempio del mercurio:
\[
Hg^{2+} + Sn^{2+} \rightleftharpoons Hg + Sn^{4+} 
\]
La soluzione viene strippata con N$_2$ e Hg (volatile) e passa all'interno di una cella a cammino ottico elevato e vengono così ottimizzati tutti i parametri come la diminuzione del rumore, l'atomizzazione aumentata,
l'aumento del cammino ottico, eliminazione dell'emissione della fiamma, eliminazione del rumore termico...

Un altro metodo impiega idruri volatili, come ad esempio per As e Sb a basse concentrazioni.
Questi elementi si trovano prevalentemente sotto forma di anioni e vengono convertiti in idruri formando arsina (AsH$_3$) e stibina (AbH$_3$) utilizzando NaBH$_4$ come donatore di idruro.
In fase gas H$^-$ cede l'elettrone all'arsenico (o dell'antimonio) trasformandoli nella forma atomica.
Questa tecnica permette di abbassare i limiti di rivelabilità.

\subsection{Interferenze}

Nell'assorbimento atomico c'è il problema delle interferenze, che possono essere di due tipi:
\begin{itemize}
\item Interferenze chimiche
\item Interferenze spettrali
\end{itemize}

Le interferenze chimiche sono dovute alla formazione di composti refrattari, come nel caso del Ca$^{2+}$ in presenza di PO$_4^{3-}$ con formazione di fosfato di calcio, che resiste alla decomposizione.
Questo problema può essere risolto impiegando ad esempio il plasma.
Le interferenze spettrali possono essere dovute alla presenza di righe di assorbimento di atomi diversi dall'analita; questo problema viene risolto cambiando riga.

\halfpicture{04_023}{Utilizzo di una lampada al deuterio per minimizzare il rumore di fondo	}{}

Un altra interferenza spettrale è dovuta alla presenza di specie radicaliche o molecolari, che producono un aumento del rumore di fondo.
In questo caso, il problema viene risolto utilizzando una lampada al deuterio, la cui radiazione passa alternativamente a quella del catodo cavo.
Date entrambe le radiazioni, è possibile fare il rapporto tra la lampada al deuterio e la lampada a catodo cavo.
È quindi possibile correggere il rumore di fondo, in quando la lampada al deuterio e la lampada a catodo cavo, in presenza di un assorbimento aspecifico, assorbono allo stesso modo; le intensità diminuiscono allo stesso modo.
Il rumore di fondo può essere, in questo modo, azzerato.
I correttori al deuterio compensano fino a 0.6 unità di assorbanza dal fondo; nei casi in cui tale compensazione non è sufficiente, si ricorre a metodi di correzione più sofisticati


\section{Spettrofotometria di emissione atomica}
L'emissione atomica sfrutta l'eccitazione termica, in quanto il rilassamento produce fotoni.
Lo schema dello strumento è rappresentato in figura \ref{fig:AES}.
È importante considerare la temperatura di lavoro, poiché questa fa variare sensibilmente il rapporto tra atomi allo stato eccitato e allo stato fondamentale.

\halfpicturelab{04_024}{Schema di uno strumento di emissione atomica}{}{AES}

L'equazione che regola il rapporto tra lo stato eccitato e fondamentale é:
\[
\frac{N_j}{N_0} = \frac{P_j}{P_0} \cdot e^{-\dfrac{E_j}{K_B T}}
\]
dove $P_j$ e $P_0$ sono parametri statistici che dipendono dal numero degli stati presenti per ogni livello quantico.

Ci sono due modi per eccitare gli atomi:
\begin{itemize}
\item Con la fiamma (1\,700 - 3\,150 \degree C); la tecnica prende il nome di F-AES
\item Con il plasma (6\,000 - 8\,000 \degree C); la tecnica prende il nome di ICP-AES
\end{itemize}

ICP è l'acronimo di Inductively Coupled Plasma.
Il plasma non è una fiamma, ma è uno stato della materia composto da gas ionizzati ad alta temperatura e il più utilizzato è il plasma di Argon.
Il plasma è elettricamente neutro con una certa percentuale ($\approx$ 5\%) di ionizzazione.
Esso viene confinato in una regione dello spazio tramite bobine a induzione elettromagnetica.

\paragraph{Torcia al plasma}
La torcia è fatta da tubi concentrici di quarzo; il gas (solitamente argon) fluisce a spirale all'interno, per mantenere la torcia stabile, e le bobine di induzione sono refrigerate ad acqua.
Quando scocca la prima scintilla in presenza di un campo magnetico il plasma è in grado di auto-sostenersi a causa della ionizzazione dell'Argon.
Il campo magnetico inoltre mantiene stabile e localizzato il plasma.

\marginpicture{04_025}{Torcia al plasma}{}

Lo spegnimento della torcia può avvenire per:
\begin{itemize}
\item Interruzione del campo magnetico
\item Interruzione del flusso di Argon
\item Introduzione del campione troppo brusca che causa una diminuzione della temperatura e per neutralizzazione troppo spinta di Ar$^+$ e elettroni.
\end{itemize}
Spesso si applica un campo magnetico dai 27 ai 40 MHz di frequenza.

\halfpicture{04_026}{Zone di temperatura nella torcia al plasma}{}

Il plasma non presenta né interferenze spettrali, né chimiche, perché qualsiasi materiale viene disgregato.
Il plasma ha una forma toroidale; il campione entra al centro del toro e viene subito distribuito.
Questo consente la nebulizzazione di liquidi, ma consente anche l'atomizzazione di solidi particellari o di gas.
Avvengono tutti i processi visti nell'assorbimento atomico.

L’analita viene introdotto sotto forma di aerosol e questo presenta scarse interferenze ed inoltre vista l'altissima energia in gioco sfrutta il fenomeno di emissione.
Tipicamente i plasmi possono avere due tipi di rivelatori:
\begin{itemize}
\item \textit{Rivelatori ottici}: il plasma emette tutte le righe di tutti gli elementi del campione e questo permette l'analisi multi-elemento e si utilizzano dei fotomoltiplicatori.
Il sistema prende quindi il nome di ICP-OES e sfrutta le proprietà ottiche degli elementi per discriminarli.
\item \textit{Rivelatori di massa}: sfrutta il fatto che il plasma ionizza pesantemente gli atomi e si discriminano in base alle masse degli ioni prodotti, riuscendo a discriminare anche gli isotopi.
Il sistema prende quindi il nome di ICP-MS.
Possono essere presenti più spettrometri di massa in cui il primo solitamente toglie le interferenze.
\end{itemize}

\subsection{Accoppiamento ICP-MS}
Nella maggior parte dei casi, è possibile analizzate tutti gli elementi della tavola periodica.
Ci sono delle eccezioni, come i gas nobili più leggeri, l'azoto e il fosforo.
Per azoto e fluoro, si utilizza l'ICP ottico in quanto lo spettrometro di massa rivela tipicamente cariche positive.

\fullpicture{04_027}{Accoppiamento ICP-MS}{}

Per utilizzare un ICP-MS, bisogna ottimizzare l'interfaccia dello spettrometro con il plasma.
Si utilizza un accoppiamento particolare, formato da un cono di ingresso e uno skimmer (cono di uscita).
Tra i due coni, c'è un sistema a vuoto, in quanto gli ioni necessitano di un vuoto spinto (10$^{10}$ bar) per poter essere analizzati senza interferenze.
Quando il plasma entra dentro la zona di vuoto, diventa supersonico e questo consente alla parte centrale di entrare all'interno dello skimmer; la composizione rimane inalterata.
Appena gli ioni entrano all'interno dello skimmer, vengono deflessi da lenti elettrostatiche, che consentono di analizzare solo le sostanze cariche.

\halfpicture{04_028}{Cono di ingresso e cono di skimming}{}

Solitamente, lo spettrometro di massa presenta dei quadrupoli per selezionare e analizzare gli ioni.
Usualmente, la risoluzione si aggira intorno a 1 u.m.a., tuttavia è possibile avere risoluzioni più basse a discapito della sensibilità
La risoluzione ridotta è uno svantaggio,in quanto i segnali possono ricevere delle interferenze da parte di altri ioni.
Ogni elemento viene individuato attraverso il suo isotopo con meno interferenze poliatomiche.

\halfpicture{04_029}{Lenti elettrostatiche}{}

\paragraph{Specie isobare}
Sono ad esempio degli isomeri strutturali di sostanze organiche e la loro massa è uguale, alla seconda cifra dopo la virgola, a quella dell’analita.
Sono specie isobare ad esempio CO e C$_2$H$_2$ che possiedono entrambi 28 di massa molecolare, ma con cifre decimali diverse.
Sono invece isobari veri isomeri conformazionali cis/trans che possiedono la stessa identica massa.

Per ovviare alle interferenze isobariche si può utilizzare una cella di collisione multipolo, dove una miscela gassosa (solitamente elio) fluisce all'interno del quadrupolo.
L'elio disgrega le interferenze molecolari, tramite trasferimento di energia da impatto.

\paragraph{Quantificazione per diluizione isotopica}
Questa tecnica è una tecnica di calibrazione che prevede che vengano variati i rapporti isotopici dell'analita nel campione, mediante aggiunte note dell'elemento di interesse, ma con composizione isotopica differente.
Dato che in spettrometria di massa si può misurare l'abbondanza isotopica di ciascun isotopo, la quantità originale dell'analita può essere ricavabile dalla misura dei rapporti isotopici.
Il calcolo più semplice è ottenuto aggiungendo un isotopo puro all'analita

\section{Tecniche ifenate}
Le tecniche ifenate sono formate dall'interfacciamento di un sistema separativo ad un sistema rivelativo (solitamente spettroscopico).
Si sfruttano i vantaggi di entrambi i sistemi:
\begin{itemize}
\item \textit{Cromatografia}: produce frazioni pure di analita
\item \textit{Spettroscopia}: fornisce informazioni su un componente puro
\end{itemize}
Le tecniche accoppiate più comuni sono GC-MS e LC-MS; altre tecniche meno diffuse, però utilizzate, sono GC-ICP-AES e LC-ICP-MS.

L'accoppiamento GC-ICP-AES ha un interfacciamento più semplice degli altri accoppiamenti,in quanto non avviene l'accoppiamento con la massa.
La colonna e l'interfaccia sono comunque termostatate, per evitare che gli analiti condensino dopo la colonna GC.
Viene prodotto uno spettro di emissione per ogni analita

\paragraph{Accoppiamento LC-ICP-MS}
Questo accoppiamento è utilizzato per la determinazione di ioni metallici, complessi e specie metallo-organiche.
La struttura è ottenuta sommando le componenti; inizialmente c'è una colonna cromatografica (per cromatografia liquida), successivamente c'è una torcia al plasma e infine è presente uno spettrometro di massa.

Il vantaggio principale di  questa tecnica è la versatilità dell'HPLC.
L'HPLC è uno strumento versatile, in quanto si può scegliere la strategia di separazione, con le tecniche viste precedentemente.
La scelta dipende dal tipo di analita, tuttavia le più utilizzate sono la cromatografia liquida a fase inversa e la cromatografia ionica.
Si noti anche che le due fasi, per la cromatografia liquida e l'ICP sono compatibili.

La tecnica presenta anche degli svantaggi, infatti, non si possono analizzare soluzioni molto concentrate in quanto queste sono problematiche per le conseguenti incrostazioni
che avvengono nel cono di skimming e per la difficoltà nel nebulizzarle.
Le soluzioni analizzate devono essere poco concentrate.

\paragraph{Accoppiamento GC-ICP-MS}
Questo accoppiamento è un po' più complicato, però è anche più efficace dell'LC-ICP-MS.
È necessario predisporre una linea di riscaldamento tra il gas-cromatografo e l'ICP, per evitare la condensazione delle sostanze in uscita.
Si noti inoltre che è necessario addizionare del gas al flusso uscente dal gas-cromatografo, in quanto il gas-cromatografo lavora con un flusso di circa 1 mL/min, mentre l'ICP-MS lavora con un flusso di 1 L/min.

\halfpicture{04_030}{Accoppiamento GC-ICP-MS}{}

Questo sistema presenta tuttavia una maggior risoluzione, in quanto la gas-cromatografia è più efficiente della cromatografia liquida nel separare le sostanze.
Inoltre, l'assenza di liquido permette al plasma di essere più secco e meno soggetto a interferenze e il problema delle incrostazioni non si presenta.

Per collegare i due strumenti, si usa una linea, detta \emph{transfer line}, che è formata da due tubi di acciaio riscaldati.
Il primo tubo contene un capillare di silice che trasporta il flusso tra i due strumenti.
Il gas di mark-up viene riscaldato nel secondo tubo, e successivamente introdotto nel primo, in modo tale da fluire intorno al capillare di silice

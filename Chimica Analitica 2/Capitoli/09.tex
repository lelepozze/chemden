\chapterpicture{header_11}
\chapter{Riassunto di Laboratorio}
I metodi di analisi chimico-strumentali presentano alcuni vantaggi e svantaggi rispetto alle tecniche di analisi chimiche.
Il limite di rivelabilità per una sostanza passa da circa 10$^{-4}$ M a valori di anche 10$^{-10}$ M ed è possibile effettuare delle analisi multi-elemento, a discapito però di costi di manutenzione e acquisto degli strumenti più elevati.
Le fonti di errore per analisi chimiche sono prettamente a carico dell'operatore, della vetreria e dei reagenti, mentre per un'analisi strumentale sono a carico dell'operatore, della vetreria, dei reagenti, della strumentazione e molto spesso anche della procedura.
Per i metodi strumentali è necessaria la calibrazione prima di poter effettuare la misurazione incognita e non sono metodi assoluti in quanto il dato fornito non è direttamente correlabile alla quantità di analita.

\paragraph{Determinazione degli Acidi Grassi in un olio}

Vengono determinate le frazioni relative a acidi grassi contenuti in un olio tramite gas-cromatografia.

\halfpicture{09_001}{Reazione di transesterificazione}{}

Gli acidi grassi saturi più comuni sono palmitico (C16) e stearico (C18), mentre l'acido grasso insaturo più comune è l'oleico (C18:1).
Per seguire una GC occorre trans-esterificare gli esteri in modo da trasformarli in esteri metilici più volatili.
Per l'analisi quantitativa è necessario solamente confrontare i tempi di ritenzione dei picchi associati agli acidi grassi con degli standard, tenendo conto che sussistono le seguenti regole:
\begin{itemize}
\item Il tempo di ritenzione aumenta all'aumentare degli atomi di carbonio
\item Gli esteri insaturi escono dopo dei corrispettivi saturi
\item Il tempo di ritenzione degli esteri insaturi aumenta all'aumentare del numero di doppi legami.
\end{itemize}

Per rendere più accurata la normalizzazione interna si possono calcolare i fattori di risposta misurando l'area dei picchi di una soluzione di acidi grassi di concentrazione $C_i$:
\[
f_i = \frac{A_i}{C_i}
\]

La concentrazione dell'acido grasso nel campione si ricava nel seguente modo:
\[
C_i = \frac{\frac{A_i}{fi}}{\sum \frac{A_i}{f_i}}
\]

Negli oli alimentari si può attuare una normalizzazione interna e la composizione relativa del campione è ricavata dal rapporto dell'area di ciascun componente e l'area totale:
\[
C_i = \frac{A_i}{A_{tot}} = \frac{A_i}{\sum A_i}
\]

Questo metodo, per essere valido, richiede due requisiti:
\begin{itemize}
\item Tutti i componenti devono essere eluiti
\item Tutti i componenti devono avere simili fattori di risposta.
\end{itemize}

\paragraph{Determinazione della caffeina in bevande gassate}
Si tratta di un'analisi qualitative e quantitativa in quanto viene anche determinato il tempo di ritenzione tramite calibrazione esterna.

\halfpicture{09_002}{Grafico spettrofotometrico della caffeina}{}

Il rivelatore dell'HPLC è uno spettrofotometro UV-Vis impostato su una lunghezza d'onda di 270 nm.
Per la determinazione della concentrazione si costruisce una retta di calibrazione esterna e si estrapola la concentrazione di caffeina nei campioni.

\paragraph{Determinazione degli anioni nell'acqua potabile}
Vengono determinati i cloruri, nitrati, solfati e fosfati contenuti in un campione di acqua potabile mediante cromatografia ionica.
Per la determinazione si esegue una calibrazione esterna con soluzioni a concentrazione nota degli anioni.
Si utilizza come eluente una miscela CO$_3^{2-}$ / HCO$_3^-$ in modo da rendere rivelabili solo i composti che formano acidi coniugati, nella successiva soppressione, più forti di H$_2$CO$_3$.

\halfpicture{09_003}{Grafico per la cromatografia ionica degli anioni in acqua potabile}{}

La soppressione è effettuata con trattamento in una colonna secondaria con una soluzione di H$_2$SO$_4$.
All'ingresso della colonna principale è presente una pre-colonna per evitare sporcamenti e preservare la colonna principale.
La concentrazione degli ioni viene estrapolata dalla retta di calibrazione. I limiti di legge per le sostanze sono:

\paragraph{Determinazione del fosfato nella Coca-Cola}
Per la determinazione dello ione fosfato (PO$_4^{3-}$) nella Coca-Cola si utilizza il metodo al blu di molibdeno, applicabile per concentrazioni di fosfato comprese tra 0.03 e 0.3 mg/L e gli interferenti possibili sono As (V), NO$_2^-$, S$^{2-}$, Cr (VI)...

\halfpicture{09_004}{Reazione dello ione fosfato}{}

Viene preparato quindi il cosiddetto 'reattivo misto' contente eptamolibdato di esammonio tetraidrato, acido solforico, acido ascorbico, tartrato di potassio e antimonile.
Per le soluzioni standard si aggiungono dei volumi di soluzione Standard da 300 ppm di KH$_2$PO$_4$.
L'assorbanza del blu di molibdeno, misurata a 886.6 nm o 707.6 nm (massimo relativo), va misurata dopo 10 minuti e non oltre 15 minuti dall'aggiunta del reattivo misto in quanto dopo questo intervallo di tempo inizia a calare.

\halfpicture{09_005}{Spettro della soluzione dopo 10 minuti}{}

Prima della misura degli standard è necessario azzerare lo spettrofotometro con dell'acqua MilliQ, leggere il bianco contenente acqua MilliQ e il reattivo misto e prima della misurazione del campione leggere l'assorbanza di un bianco campione contenente il campione diluito senza reattivo misto.
Per valutare l'effetto dell'interferza negativa di NO$_2^-$ si legge l'Abs del campione contenente questo composto.
Si costruisce poi una retta di calibrazione ponendo Abs vs C (PO$_4^{3-}$) e per interpolazione si determina la concentrazione di fosfato nel campione.

\paragraph{Determinazione dello zinco nei capelli}
Si determina la quantità di zinco in un campione di capelli, previa mineralizzazione con HNO$_3$ e HClO$_4$ e calibrazione dello spettrofotometro ad assorbimento atomico con soluzioni standard di zinco.
Per valutare l'eventuale effetto matrice di opera anche con il metodo della calibrazione interna (aggiunte standard).

La fiamma dell'AA alimentata a acetilene e aria può arrivare a temperature superiori ai $2\,000$ \degree C e l'introduzione delle soluzioni nello strumento avviene dalla più diluita (meno concentrata) alla meno diluita (più concentrata).
La lettura dell'assorbanza dello Zinco viene fatta alla lunghezza d'onda di lavoro di 213.9 nm.

Si ottiene sia per la calibrazione esterna, sia per quella interna due rette.
Si estrapolano i valori dello zinco e si vede l'eventuale effetto matrice per confronto delle pendenze delle due rette.

\paragraph{Analisi dell'acqua mediante spettrometro ICP-OES}
Si determina la quantità di zinco nei capelli e di alcuni ioni metallici (Ca, Co, Cu, Fe, Mn, Mg, Pb, Sn, Zn) nell'acqua potabile.
Le soluzioni da analizzare contengono un'aliquota di HNO$_3$ al 3.5 $\%$ utile a mantenere pulita la linea dell'ICP.
Il plasma ad argon dell'ICP arriva anche a temperature di $10\,000$ \degree C e viene utilizzato un auto-campionatore per l'introduzione delle soluzioni nello strumento.
Le misure di assorbanza vengono corrette in automatico dallo strumento sulla base dell'emissione dell'argon (standard interno).

\paragraph{Determinazione del Fluoruro in un collutorio tramite ISE}
Viene determinata la concentrazione di Fluoruro in un colluttorio mediante Elettrodo Iono-Selettivo al LaF$_3$, costituito da una membrana selettiva e da un elettrodo.
Il potenziale generato non è dovuto solo alla reazione redox, bensì al potenziale di membrana che si instaura tra i due lati.

\marginpicture{09_006}{Elettrodo ioniselettivo per il fluoruro}{}

L'elettrodo è costituito da un monocristallo di LaF$_3$ drogato con EuF$_2$ per dare conducibilità e all'interno è presente un elettrodo di 'riferimento interno' di Ag/AgCl in una soluzione di KCl e F$^-$ $\approx$ 0.01 M.
Per l'ISE al fluoruro il potenziale, analogamente all'elettrodo a vetro con H$_3$O$^+$, dipende dalla concentrazione (attività) di F$^-$ mediante:
\[
E_{ISE, F^-} = A + B \cdot pF^-_{est}
\]

Rispetto all'equazione che regola il potenziale per l'elettrodo a vetro presenta il segno positivo invece che negativo, intuitivamente dovuto al fatto che F$^-$ è uno ione negativo e H$_3$O$^+$ è positivo.
In realtà, l'ISE al fluoruro non è selettivo solo per F$^-$ e la costante di selettività si ricava dall'equazione di Nikolskii-Eisemann:
\[
E = A - B \log \biggl(a_{F^-} + \sum_j k_j a^{\frac{1}{z_j}}\biggl)
\]

In questo caso l'unico interferente importante è OH$^-$ e $k_{OH^-}$ $\approx$ 0.01 e affinché l'elettrodo possa misurare correttamente l'attività di F$^-$ è necessario rendere il termine additivo trascurabile:
\[
E = A - B \log (a_{F^-} + k_{OH^-} \cdot a_{OH^-})
\]

\halfpicture{09_007}{Schema del voltmetro utilizzato}{}

Mediante calibrazione esterna con soluzioni a concentrazione diversa di fluoruro si determinano A e B.
Il pH di lavoro deve essere minore di 7 in quanto l'ISE possiede linearità fino a 10$^{-6}$ M di F$^-$ perché poi interferisce la solubilità di LaF$_3$.
Se il pH di lavoro è minore di 5 possono instaurarsi delle reazioni che diminuiscono la quantità di Fluoruro:
\begin{align*}
& F^- + H_2O \rightleftharpoons HF + OH^-\\
& HF + F^- \rightleftharpoons [HF_2]^-\\
& 4 F^- + SiO_2 + 2 H_2O \rightleftharpoons SiF_4 + 4 OH^-\\
& n F^- + M^{3+} \rightleftharpoons MF_n \quad \text{con} \quad M = Al^{3+}, Fe^{3+}\\
\end{align*}

Le soluzioni di F$^-$ vengono perciò conservate in recipienti di plastica e si aggiungono dei complessanti, come il citrato di sodio, che eliminato i cationi metallici trivalenti.
Per svincolarsi dai coefficienti di attività, regolare il pH ed eliminare ioni complessanti le misure vengono effettuate con forza ionica costante, mediante aggiunta del TISAB
(Total Ionic Strenght Adjustment Buffer, a pH 5.0 e forza ionica di 1.75 M) contenente NaCl, acido acetico, acetato di sodio e citrato di sodio.

\halfpicture{09_008}{Calibrazione con il metodo di Gran}{}

Si effettua dapprima la calibrazione su due punti dell'elettrodo, successivamente si effettua la calibrazione esterna e poi si misura la fem del campione.
Si effettua poi la determinazione della concentrazione di fluoruro mediante aggiunte standard con il \emph{metodo di Gran} che permette di linearizzare risposte strumentali di tipo logaritmico:
\begin{align*}
& E = A + B \cdot pF^- \longrightarrow [F^-] = 10^{\frac{A-E}{B}}\\
& [F^-] = \frac{C_i V_i + C_s V_s}{V_i + V_s}\\
& C_i V_i + C_s V_s = 10^{\frac{A-E}{B}} \cdot (C_i + V_s)\\
& C_i V_i + C_s V_s = (C_i + V_s) \cdot \biggl(10^{-\frac{E}{BB}} 10^{\frac{A}{B}}\biggr)\\
& (C_i V_i + C_s V_s) 10^{\frac{A}{B}} = (C_i + V_s) 10^{-\frac{E}{BB}} = G
\end{align*}

$G$ è detta Funzione di Gran e restituisce il seguente grafico $G$ vs $V_s$ e per $G=0$ si può ricavare:
\[
C_i = \frac{C_s V_s}{V_i}
\]

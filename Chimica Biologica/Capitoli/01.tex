\part{Introduzione}

\chapterpicture{header_02}

\chapter{Chimica biologica}

La biologia è lo studio della materia vivente a livello molecolare,
ossia lo studio della chimica della vita. In particolare si occupa di
determinare la struttura chimica e tridimensionale delle molecole
biologiche, come queste interagiscono tra loro, in che modo la cellula
immagazzina e conserva l'nergia e quali sono i meccanismi di
replicazione delle molecole e quindi il passaggio delle informazioni
genetiche

Lo stato vivente è uno stato di non equilibrio. Quando lo stato vivente
raggiunge l'equilibrio termodinamico, lo stato vivente muore. Per
rimanere in uno stato di non-equilibrio, lo stato vivente necessita di
un costante apporto di energia. Lo stato vivente è caratterizzato da una
complessità chimica; è in continua evoluzione quindi muta nel tempo. Lo
stato biologico dà luogo a reazioni catalizzate (con catalizzatori
specifici, ``ad hoc''). Le condizioni di reazione sono condizioni
blande.

Lo stato biologico è capace di autoreplicarsi tramite un programma di
autoreplicazione

La materia vivente obbedisce alle stesse leggi chimiche e fisiche che
governano tutta la materia

La funzione biologica di una molecola è descritta semplicemente dalla
sua struttura e dai suoi gruppi funzionali.

La vita è caratterizzata da:
\begin{itemize}
\item Biodiversità: si vede come la vita
riguarda una vasta gamma di organismi, come licheni, meduse, microbi,
esseri umani, etc \ldots{}
\item Unità chimica: La materia vivente obbedisce
alle stesse leggi chimiche e fisiche che governano tutta la materia
\end{itemize}

La Terra si è formata 4.6 miliardi di anni fa, mentre la vita è comparsa
3.6 miliardi di anni fa. Si pensa che il sole e l'energia proveniente
dai fulmini abbiano fornito l'energia necessaria per la formazione di
molecole organiche a partire da \ce{H2O}, \ce{N2}e \ce{CO2}. Le prime
forme di vita sono i cianobatteri fotosintetici, che producono ce\{O2\}.
In seguito si sono sviluppati i batteri anaerobici, che sviluppano
\ce{O2} e quindi i batteri aerobici, ovvero tutti gli altri.

Diversi organismi hanno delle cellule biologiche molto simili, in quanto
si presume che la vita si sia evoluta da un predecessore ancestrale
comune.

\fullpicture*{1_001}{L'albero filogenetico della vita comprende tre domini: bacteria, archaea ed
eucarea.}

L'albero evolutivo giustifica questo fatto; quindi gli organismi hanno
sviluppato una complessità maggiore con l'evoluzione.

La vita necessita di:
\begin{itemize}
\item Energia
\item Molecole semplici, che fungono da
``mattoncini'' per la cellula
\item Meccanismi chimici, che sono reazioni
altamente coordinate
\end{itemize}

L'energia va estratta, immagazzinata, trasformata e utilizzata. In
queste fasi sono prodotte molte molecole semplici, come \ce{CO2},
\ce{NH3} e \ce{H2O}. In base alla fonte di energia si distinguono gli
organismi fototrofi, che si alimentano grazie alla luce, e organismi
chemiotrofi, che si alimentano grazie all'energia chimica

Le molecole semplici sono i blocchi per creare i polimeri, ovvero delle
molecole grandi. Le molecole piccole vanno convertite, polimerizzate e
degradate



I meccanismi chimici servono per convertire e sfruttare l'energia per
sintetizzare e degradare le macromolecole. I meccanismi chimici servono
per il mantenimento di uno ``steady state'' che possa sopportare e
consentire le reazioni di autoassemblamento e replicazione rispetto
all'ambiente esterno.

Quindi le questioni che la biochimica si pone sono da affrontare con un
approccio molecolare:
\begin{itemize}
    \item Quali sono le strutture chimiche dei componenti della materia vivente? E
    in che modo le strutture chimiche influenzano la funzione?
    \item Quali sono le interazioni importanti nel determinare la formazione delle sovrastrutture biologiche?
    \item Come viene estratta dall'ambiente l'energia necessaria al mantenimento e
    alla crescita?
    \item Come vengono immagazzinate le informazioni e come vengono trasmesse?
    \item Quali cambiamenti chimici sono connessi all'invecchiamento e alla morte?
\end{itemize}

La biochimica è una scienza moderna. Fino al 1800 si credeva alla teoria
del vitalismo, poi con la sintesi dell'urea dai composti inorganici, il
vitalismo smette di essere seguito. La sintesi dell'urea viene
effettuata da W\"ohler, nel 1828; la reazione è la seguente
\[
\ce{NH4(OCN) -> NH2-(C=O)-NH2}
\]

Nel 1860, Pasteur dimostrò che la fermentazione è legata alle cellule di
lievito viventi. Nel 1897, i Buchner estraggono una sostanza, detta
zimasi, da delle cellule di lievito morte. Questa sostanza garantisce la
fermentazione anche in assenza di cellule vive; questa è un'altra prova
contro il vitalismo. Si dimostra quindi che il metabolismo è composto da
reazioni chimiche.

Agli inizi del 1900 si scopre come funziona il legame chimico. Linus
Pauling determina la struttura di alcune proteine e predice l'esistenza
delle \alpha-eliche. Tra il 1930 e il 1950, grazie alla microscopia
elettronica, si inizia a indagare la struttura cellulare. Quindi si
scopre che la cellula è una foresta di composti chimici e sede di
moltissime reazioni.

Tra il 1932 e il 1939 Hans Krebs scoprì il ciclo metabolico che porta il
suo nome e con questa scoperta ci si inizia a chiedere da dove arriva
l'energia che viene utilizzata per la cellula. Viene identificata una
molecola, che ha la funzione di ``batteria'', ovvero l'ATP. L'ATP viene
prodotta da altre molecole con un alto valore energetico, quindi si
capisce che serve un intermedio energetico nella trasformazione di luce
in ATP.

Nel 1960 Peter Mitchell scopre il gradiente di protoni attraverso una
membrana e riesce a collegare che
\(\Delta G = \text{energia utilizzabile}\). L'enzima che catalizza la
reazione di ATP-sintasi è un grande complesso proteico di membrana
(Boyer e Walker, Nobel 1998).

Genetica

Nel 1944 Oswald Avery scopre che un acido nucleico è formato dal
desossiribosio e che questo polisaccaride è l'unità fondamentale del
DNA. Da questo arriva il dogma fondamentale della biologia, ovvero
\[
\text{DNA} \rightarrow \text{RNA} \rightarrow \text{Proteine}
\]

Rosalind Franklin, tramite la diffrazione ai raggi X, riesce a
determinare la struttura del DNA nel 1953, ma Watson e Crick rubano il
merito della scoperta e sono insigniti del premio Nobel. Da qui in
avanti vengono effettuate delle nuove scoperte, come i cromosomi, il
genoma e il codice genetico.

Nella scienza attuale, non si è ancora determinato il ruolo esatto del
DNA; nel DNA umano, solo il 2--3 \% codifica le proteine, mentre il resto
non ha una funzione nota. Il DNA non codificante viene chiamato ``DNA
spazzatura''.

La complessità di un organismo è data dall'estensione del DNA non
codificante e non dal numero di geni; questo si vede nel riso, che ha
più o meno gli stessi geni dell'uomo.

La prima proteina ad essere scoperta è la mioglobina e in seguito viene
scoperta anche l'emoglobina.

La biochimica presenta diverse sfide per il futuro, come la
manipolazione genetica-biotecnologie, l'ingegneria proteica, la
manipolazione del sistema nervoso, l'evoluzione e altre cose

Nel 1953 Miller e Urey provano a creare la vita nello stesso modo in che
si pensa sia nata, ovvero tramite l'energia solare e i fulmini, in una
soluzione di acqua contenente \ce{NH3}, \ce{CH4} e \ce{H2}. Si è
riuscito a dimostrare la produzione delle prime molecole che sono la
base della vita.



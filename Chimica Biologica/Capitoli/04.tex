\part{Proteine}

\chapterpicture{header_03}

\chapter{Le proteine}

\ChangePicturesFolder{4}

Le proteine sono polimeri lineari, non ramificati, composti almeno da 20
amminoacidi. L'informazione genetica è in gran parte espressa in termini di proteine

Ogni proteina è definita dalla sequenza di amminoacidi che la
compongono. Questi si riorganizzano (in un processo che si vedrà più
avanti, detto folding) dando luogo ad una struttura tridimensionale. La
struttura è di fondamentale importanza in quanto è questa che determina
la funzione. È molto più determinante quindi la forma delle proteine
piuttosto che la loro composizione chimica, infatti le proteine possono
differire di qualche amminoacido (anche molti), ma avere la stessa
funzione. Gli amminoacidi che sono diversi devono avere quasi le stesse
proprietà chimico-fisiche.

Durante l'evoluzione diverse specie hanno sintetizzato le stesse
proteine in maniera diversa, con diversi amminoacidi aventi proprietà
simili. Le proteine che sono simili sono dette ``proteine omologhe''.

Le funzioni delle proteine sono praticamente infinite e molto diverse
tra loro. Le proteine hanno una funzione specifica ed andrebbero
studiate e analizzate caso per caso, indagando sulla struttura e le
proprietà. È difficile generalizzare la loro funzione, si può però fare
un elenco di massima che comprende le maggiori funzioni:
\begin{itemize}
\item
Contribuiscono alla struttura perché, ad esempio, nella matrice
extracellulare c'è il collagene che contribuisce a mantenere la forma
(proteina strutturale).
\item
Molte fanno da catalizzatori (la maggioranza) e catalizzano le
reazioni del metabolismo.
\item
Nelle cellule che sono compartimentalizzate è necessario il trasporto
(di sostanze e informazioni con flussi di materia e di energia) dentro
e fuori la cellula e anche tra organelli o tra cellule (sistema
circolatorio).
\item
Sono fondamentali nella risposta immunitaria (emoglobine, anticorpi).
\item
Servono per l'immagazzinamento, come ad esempio la ferritina, la mioglobina per il trasporto di ossigeno
\item
Il segnale di attivazione di un movimento viene portato tramite
proteine
\item
Controllo della crescita e del differenziamento
\item
Movimento coordinato ma entrano anche in piccolo come nei flagelli
delle cellule
\end{itemize}

Ci sono quindi molte proteine, alcune piccole, altre enormi. Le proteine possono assumere forme diverse:
\begin{itemize}
\item
\emph{Globulari:} possono trovarsi nel citoplasma
(solubili) o incastonate nella membrana (idrofobiche) e questo impone
una diversa strategia di disposizione degli amminoacidi.
\item
\emph{Fibrose:} sono di forma allungata e tendono a fare fibre e possono
essere presenti nel citosol o nelle matrici extracellulari.
\item
\emph{Lipoproteine:} dove si possono avere legati covalentemente (dopo la
sintesi) dei lipidi singoli (sangue o linfa che servono al trasporto
lipidico)
\item
\emph{Glicoproteine:} si possono avere legati covalentemente degli zuccheri
in modo sporadico ad aminoacidi (gruppi sanguigni A, B).
\end{itemize}

Le proteine sono formate da \alpha-amminoacidi e il numero di questi può
variare da 50 fino a 5000 in una proteina. Gli amminoacidi sono 20

\autofullpicture*{Alcune proteine} %1

\section{Amminoacidi}

\automarginpicture*{Amminoacido}

Le proteine sono polimero costituiti da monomeri. Questi monomeri, come
detto prima, sono detti aminoacidi e sono venti in tutto.

La forma di struttura è molto semplice, e dato che il carbonio è
tetraedrico si possono avere diversi enantiomeri e si trova che, nel
caso specifico degli amminoacidi, si ha che hanno tutti conformazione di
tipo L (in riferimento all'L-gliceraldeide).

Gli amminoacidi sono tetraedrici e chirali. Esiste solo la forma L, e questo
consente alle proteine di avere una struttura chirale. Questo consente
anche di avere delle reazioni stereoselettive

\marginbox{Chiralità degli amminoacidi}{
Questa selettività al lato pratico è utile perché conferisce agli enzimi
stereospecificità per le reazioni ma non si capisce perché sia stata
selezionata solo una forma degli amminoacidi.\\
Di fatto si è visto che anche nei meteoriti si sono trovati degli
amminoacidi dove si ha un eccesso enantiomerico di tipo L.
}

I 20 amminoacidi variano molto per le loro proprietà chimico fisiche:

\begin{itemize}
\item
Polarità/idrofobicità
\item
Acidità/basicità
\item
Aromaticità
\item
Flessibilità conformazionale della catena laterale
\item
Ingombro sterico
\item
Capacità di formare legami ad idrogeno
\item
Capacità di formare legami a ponte
\end{itemize}

Queste capacità che possono avere gli amminoacidi permettono la
classificazione e il riconoscimento della funzione e di come sarà poi la
proteina dato che quando si legano danno proprietà specifiche. Se si sostituisce un amminoacido con uno con proprietà simili, la
funzione della proteina rimane la stessa, mentre se si sostituisce con
uno diverso la funzione cambia.

\begin{fullpaper}

\begin{table}
\begin{tabular}{lcccccc}
Amminoacido & Simbolo & Struttura & Idrofobicità & $pK_a$ \ce{COOH} & $pK_a$ \ce{NH3} & $pK_a$ \ce{-R}\\
Aspartato & Asp (D) & \tabfigure{width=0.2\textwidth}{Asp} & $-3.5$ & $1.99$ & $9.90 $ & $3.90$\\
Glutammato & Glu (E) & \tabfigure{width=0.2\textwidth}{Glu} & $-3.5$ & $2.10$ & $9.47 $ & $4.07$\\
Lisina & Lys (K) & \tabfigure{width=0.2\textwidth}{Lys} & $-3.9$ & $2.16$ & $9.06 $ & $10.54$\\
Arginina & Arg (R) & \tabfigure{width=0.2\textwidth}{Arg} & $-4.5$ & $1.81$ & $8.99 $ & $12.48$\\
Istidina & His (H) & \tabfigure{width=0.2\textwidth}{His} & $-3.2$ & $1.80$ & $9.33 $ & $6.04$\\
Tirosina & Tyr (Y) & \tabfigure{width=0.3\textwidth}{Tyr} & $-1.3$ & $2.20$ & $9.21 $ & $10.46$\\
Triptofano & Trp (W) & \tabfigure{width=0.25\textwidth}{Trp} & $-0.9$ & $2.46$ & $9.41 $ & \\
Fenilanalina & Phe (F) & \tabfigure{width=0.2\textwidth}{Phe} & $2.8$ & $2.20$ & $9.31 $ & \\
Cisteina & Cys (C) & \tabfigure{width=0.2\textwidth}{Cys} & $2.5$ & $1.4$ & $10.70 $ & $8.37$ \\
\end{tabular}
\caption{Amminoacidi. L'\emph{aspartato} e il \emph{glutammato} sono amminoacidi acidi carichi, mentre la \emph{lisina} e l'\emph{arginina} sono amminoacidi basici carichi. La \emph{tirosina}, il \emph{triptofano} e la \emph{fenilalanina} sono ammonoacidi aromatici.}
\end{table}

\clearpage

\begin{table}
\begin{tabular}{lccccc}
Amminoacido & Simbolo & Struttura & Idrofobicità & $pK_a$ \ce{COOH} & $pK_a$ \ce{NH3}\\
Metionina & Met (M) & \tabfigure{width=0.2\textwidth}{Met} & $1.9$ & $2.13$ & $9.28$\\
Serina & Ser (S) & \tabfigure{width=0.18\textwidth}{Ser} & $-0.8$ & $2.19$ & $9.21$\\
Treonina & Thr (T) & \tabfigure{width=0.18\textwidth}{Thr} & $-0.7$ & $2.09$ & $9.10$\\
Asparagina & Asn (N) & \tabfigure{width=0.2\textwidth}{Asn} & $-3.5$ & $2.14$ & $8.72$\\
Glutammina & Gln (Q) & \tabfigure{width=0.2\textwidth}{Gln} & $-3.5$ & $2.17$ & $9.13$\\
Glicina & Gly (G) & \tabfigure{width=0.1\textwidth}{Gly} & $-0.4$ & $2.35$ & $9.78$\\
Alanina & Ala (A) & \tabfigure{width=0.15\textwidth}{Ala} & $1.8$ & $2.35$ & $9.87$\\
Valina & Val (V) & \tabfigure{width=0.18\textwidth}{Val} & $4.2$ & $2.29$ & $9.74$\\
Leucina & Leu (L) & \tabfigure{width=0.2\textwidth}{Leu} & $3.8$ & $2.33$ & $9.74$\\
Isoleucina & Ile (I) & \tabfigure{width=0.2\textwidth}{Ile} & $4.5$ & $2.32$ & $9.76$\\
Prolina & Pro (P) & \tabfigure{width=0.15\textwidth}{Pro} & $1.6$ & $1.95$ & $10.64$\\
\end{tabular}
\caption{Amminoacidi. La \emph{cisteina}, la \emph{metionina}, la \emph{serina} e la \emph{treonina} sono amminoacidi debolmente polari, mentre l'\emph{asparagina} e la \emph{glutammina} sono amminoacidi fortemente polari. Infine, la \emph{glicina}, l'\emph{alanina}, la \emph{valina}, la \emph{leucina}, l'\emph{isoleucina} e la \emph{prolina} sono ammonoacidi apolari idrofobici.}
\end{table}

\end{fullpaper}


\subsection{Tipologie di amminoacidi}

Gli \emph{amminoacidi polari} sono circa metà e metà e si hanno
ulteriori distinzioni come ad esempio carico o non, tanto o poco, se
dissocia in funzione al pH.

Nel gruppo degli \emph{amminoacidi acidi e carichi} si hanno l'aspartato e il glutammato, i quali sono molto simili in
struttura. L'acidità viene data dal gruppo carbossilico e hanno una catena
carboniosa più o meno lunga, che comporta un ingombro sterico diverso.

Invece, per gli \emph{amminoacidi basici e carichi} si hanno la lisina e l'arginina.
Sono basici con gruppi protonati, a pH fisiologico.

La lisina è molto importante e ha un braccio lungo e lineare con un
gruppo basico \ce{NH3}.
L'arginina contiene tre \ce{CH2} e il gruppo funzionale carico
è \ce{NH2+}, che comporta una $pK_a$ maggiore
rispetto alla lisina.

Restando su amminoacidi carichi, si può avere l'istidina, che può essere
carico ma ha una pka di 6, che è un valore vicino al neutro e quindi lo
si può trovare sia carico che non.


Gli \emph{amminoacidi aromatici} hanno una catena laterale con elevata delocalizzazione
elettronica e sono planari.

La tirosina ha il gruppo \ce{OH} mentre la fenilalanina no, quindi è meno
polare.
Il triptofano ha un doppio anello condensato con un eteroatomo (indolo)
con geometria piatta molto estesa.

Gli \emph{amminoacidi debolmente polari} sono simili tra loro tranne per lo scambio ossigeno-zolfo che li rende
debolmente polari.

La cisteina e la serina sono molto simili a parte lo scambio e spesso
hanno funzioni simili.
La metionina ha uno zolfo legato al \ce{CH3} mentre la
treonina ha un \ce{OH} ma in una posizione diversa.
La cisteina (sotto forma \ce{SH}) in vicinanza di un'altra cisteina e in
presenza di un ambiente ossidante può dar luogo al ponte disolfuro
(legame covalente tra 2 cisteine).

Gli \emph{Amminoacidi fortemente polari} non sono carichi e sono l'asparagina e la glutammina.
La glutammina è l'analogo dell'asparagina ma più lungo.

Gli \emph{amminoacidi non polari, o idrofobici} hanno meno proprietà degli altri e hanno catena laterale più o meno
ingombrante fatta sostanzialmente di C e H.

La glicina è un jolly perché sta bene in tutte le posizioni
che sarebbero critiche per ingombro sterico (perché ha solo un \ce{H} come
catena laterale).

\marginbox*{
    Il collagene ha una glicina ogni 3 amminoacidi e se non c'è questo amminoacido, il collagene non si forma.
}

La prolina ha catena laterale che si chiude con legami covalenti con
il gruppo amminico della parte ``comune'' e non è alfa amminoacido ma è
un imminoacido. Anche la prolina è una specie di jolly, ma proprio per la ciclicità serve
in zone in cui si hanno piegamenti stretti (gomiti) della catena
polipeptidica

\subsection{Equilibri acido-base}

Gli amminoacidi si presentano in diverse forme in base al \(pH\)
dell'ambiente in cui si trovano. In ambiente fisiologico, la forma
prevalente è quella con il gruppo amminico protonato e quello
carbossilico deprotonato. Questa forma è detta zwitterione, o ione
dipolare

\autoherepicture{0.8} %6

\autofullpicture*{Forme degli amminoacidi in funzione al $pH$} %6

Il punto isoelettrico è definito come il valore di pH alla quale non vi
è una carica netta. Questo punto, sperimentalmente, si trova misurando
il pH al quale un amminoacido, in un campo elettrico, non tende a
migrare verso il catodo o l'anodo. Il punto isoelettrico è definito come
\[
pI = \frac{1}{2} (pK_i + pK_j)
\]
dove \(K_i\) e \(K_j\) sono le costanti di dissociazione delle due
reazioni di ionizzazione, che avvengono a carico delle specie neutre. Le
costanti di dissociazione dipendono anche dalle catene laterali.

Tutto questo consegue che anche le proteine hanno un punto isoelettrico.
Questa proprietà può essere un'altra per separare le proteine,
utilizzando quindi l'elettroforesi.

\marginbox{Elettroforesi}{
L'elettroforesi è un processo di separazione di composti polari basato sulla loro mobilità in un supporto
solido.\\
L'aspetto di capacità acido base e l'influenza del gruppo R sullo
zwitterione influenza la carica netta e ogni proteina ha un punto
isoelettrico diverso.\\
Se si svolge un gradiente di pH e si applica un campo elettrico, si
possono separare le proteine; la raggiunta della neutralità e quindi la
fine della corsa dipende dal pH e dalla tipologia di proteine.\\
In questo modo si sfrutta il diverso R che varia il punto isoelettrico.
}

Alcuni amminoacidi (quelli aromatici) danno assorbimento nella regione
dell'UV-Vis, quindi si può usare la legge di Lambert-Beer per
determinare la concentrazione di tali amminoacidi.

Altri amminoacidi sono detti essenziali, perché non vengono sintetizzati
nell'organismo, e quindi devono essere assunti con la dieta. Per l'uomo,
questi amminoacidi sono: Phe, Ile, Hi, Leu, Lys, Met, Trp, Val, Thr.

Alcuni amminoacidi possono subire modifiche post-traduzionali per creare
delle proteine un po' particolari. Quindi questi amminoacidi non sono
codificati dal DNA.{}

Nelle proteine, gli amminoacidi possono essere modificati in maniera
reversibile, ad esempio tramite la fosforilazione. Questi processi hanno
spesso la funzione di inibire o attivare enzimi. Quindi controllando
questi processi di modifica, è possibile controllare l'attività di una
proteina.

Un altro caso sono gli amminoacidi che non compongono le proteine, che
possono svolgere svariate funzioni nell'organismo, tra queste avviamo la
serotonina.

I peptidi sono catene di amminoacidi non troppo lunghe e svolgono
numerose funzioni. Ci sono ormoni e feromoni, come l'insulina e
l'ossitocina, neuropeptidi, antibiotici e peptidi che svolgono una
funzione protettiva.

\paragraph{Metabolismo degli amminoacidi}

Dalla digestione, si ottengono, tra le varie cose, anche gli
amminoacidi. In genere questi vengono utilizzati per la sintesi di
proteine, ma in certi casi, questi possono essere convertiti in zuccheri
e in acidi grassi per ottenere energia. Poi ci sono anche gli
amminoacidi provenienti dalla decomposizione di proteine, che hanno
finito il ciclo vitale e quindi vengono degradate.

\section{Legame peptidico}

La polimerizzazione di un amminoacido viene raggiunta per eliminazione
di una molecola d'acqua tra il gruppo carbossilico e un gruppo amminico.
Tuttavia, questo processo costa energia e quindi non è un processo
spontaneo. Nelle cellule, il legame viene creato nei ribosomi ed è
catalizzato da enzimi con reazioni che coinvolgono l'ATP.{}

Anche se il legame non è termodinamicamente stabile, la cinetica della
reazione di scissione di un legame peptidico è molto lenta.

Per convenzione il gruppo amminico si mette a sinistra e il gruppo
carbossilico a destra. Quindi la catena ha una direzione. Se si inverte
la direzione, il peptide ottenuto non riesce a svolgere la stessa
funzione

\autoherepicture{0.8} %4

\marginbox*{
È importante saper disegnare la struttura di un peptide in dipendenza dal pH (a pH acido i gruppi carbonilici sono protonati).
}

I polipeptidi sono molecole metastabili, infatti la reazione di idrolisi di un peptide è rapida solo in presenza di acidi forti o di
enzimi proteolitici, dette proteasi

Nella formazione di proteine, gli amminoacidi sono attivati dal tRNA (o
RNA-transfer), in quanto gli amminoacidi vengono attivati tramite legame con questo e possono poi
fare un attacco nucleofilo spontaneo verso l'altro amminoacido.

\autofullpicture*{Legame peptidico} %4

\paragraph{Previsione della struttura di una proteina}

Per prevedere la struttura bisogna capire se ci sono vincoli
chimici/elettronici.

Noti gli atomi si può prevedere come si legano e in questo caso si hanno
tanti atomi e bisogna vedere le caratteristiche del legame peptidico.

Il legame peptidico ha un legame \ce{C-N} e si ha una certa condivisione
elettronica tra \ce{O-C-N} quindi è un legame molecolare delocalizzato che
riguarda gli elettroni negli orbitali pz di simmetria pi greco.

Deriva dal fatto che C e N sono ibridati sp\ap{2} e hanno
elettroni negli orbitali pz che possono formare un legame delocalizzato.

In questo legame, proprio per l'ibridazione sp\ap{2} il
legame \ce{C-N-C-O} giace sullo stesso piano e questo comporta che sullo
stesso piano ci siano sei atomi (\ce{C_{\alpha}}, \ce{C}, \ce{H}, \ce{N}, \ce{O} e \ce{C}\ped{\alpha}') dove gli elettroni \pi{} stanno sopra e sotto questo
piano e sono condivisi.

\autoherepicture{0.8} %4

Ogni legame peptidico individua un piano dove ci sono 6 atomi che stanno
sullo stesso piano ma questo non significa che i piani sono coplanari.

Sul C\ped{\alpha} si può ruotare un piano rispetto all'altro perché
ci sono legami singoli attorno ai quali si possono fare rotazioni,
quindi ogni C\ped{\alpha} è uno snodo che fornisce diverse
possibilità di conformazione.

Questa cosa di avere i 6 atomi coplanari vincola la forma del peptide e il backbone è vincolato ad avere certe conformazioni.
Nella formazione del legame concorrono gli orbitali p\ped{z} di C, O, N.
\automarginpicture*{Legame \pi{} planare. Si nota che nella seconda forma è presente un momento di dipolo permanente.} %4

Gli atomi coinvolti nel legame peptidico, oltre ai legami singoli ci
sono gli elettroni \pi{} (2p\ped{z}) che partecipano all'orbitale molecolare.
Si hanno due elettroni in coppia solitaria dell'N che appartengono al 2p\ped{z} dell'N, il 2p\ped{z} del C con un elettrone spaiato e idem per l'O.

Si hanno quindi quattro elettroni che possono essere accomodati negli orbitali
molecolari; il legame \pi{} che vincola sul piano gli atomi è determinato
dal riempimento degli orbitali di simmetria \pi{} greco.
Si hanno quattro elettroni condivisi in tre orbitali e si ha un 40\% di sharing degli elettroni con l'ossigeno.
Questa configurazione determina le proprietà geometriche e stereochimiche del peptide che si forma.

Nei peptidi si forma un momento di dipolo permanente che deriva dal
fatto che l'ossigeno è molto elettronegativo con conseguente parziale
carica negativa su di esso e positiva sull'azoto.
Questo dipolo va a dare un ulteriore vincolo per la struttura tridimensionale della molecola.

I peptidi potrebbero assumere conformazione cis o trans ma si è
verificato che assumono solamente conformazione trans perché permette
una maggiore stabilità e minori problemi di ingombro sterico tra le
catene laterali che porterebbero a una destabilizzazione della molecola.

L'unico amminoacido per il quale si può osservare una conformazione cis è la prolina, nella quale la catena laterale si chiude sull'N e le due conformazioni sono pressochè equivalenti.

Una catena polipeptidica è quindi costituita da una catena principale
con struttura ripetitiva e da catene laterali variabili
R\ped{i}.
Lo scheletro di una proteina è una sequenza di gruppi peptidici planari e rigidi legati

Come si vede, il legame peptidico è caratterizzato da due angoli, ovvero
l'angolo \(\Phi\) che è l'angolo intorno al legame \ce{C_{\alpha}-N},
mentre l'angolo \(\Psi\) è l'angolo intorno al legame \ce{C_{\alpha}-C}.
Non tutti i valori sono permessi; alcune coppie di valori sono più
probabili

\autofullpicture*{Angoli del legame peptidico. Si vede che la conformazione cis non è favorita.} %4

\section{Struttura primaria}

La struttura primaria di una proteina è la sequenza degli amminoacidi
della sua catena polipeptidica. È dunque il livello strutturale
fondamentale su cui sono basate le strutture superiori. Ogni residuo
amminoacidico è unito a quello successivo attraverso un legame
peptidico. I livelli strutturali superiori delle proteine si riferiscono
alla conformazione tridimensionale delle catene polipeptidiche ripiegate

\autofullpicture*{Struttura primaria di una proteina.} %4

La prima proteina per la quale è stata determinata la struttura primaria
è l'insulina, nel 1953, che possiede 51 residui. È infatti possibile
determinare la struttura primaria delle proteine con diverse tecniche in
laboratorio.

Ogni proteine è unica in una data specie di organismo. Sono presenti
similitudini nella struttura primaria; questo perché le proteine
evolvono il volo per modifiche della sequenza primaria. Queste proteine
sono dette ``proteine omologhe''.

\marginbox*{
Le proteine omologhe presentano una correlazione evolutiva; il contorno tra di esse permette di capire quali sono residue essenziali per la funzione della proteine, e quelli residui assumono una rilevanza minore.
}

Le proteine possono avere piccole
differenze di struttura primaria ma avere la stessa struttura spaziale e
quindi la stessa funzione.
Quindi si può dire che:
\begin{itemize}
\item
Se si seguono le sequenze di una stessa proteina in diverse specie si
capisce anche quanto sono vicine diverse specie nell'albero
filogenetico (Specie evolutivamente vicine hanno sequenze più simili).
\item
Una stessa proteina in specie diverse può avere una sequenza diversa
ma le strutture devono essere simili altrimenti non possono fare la
stessa funzione.
\item
Quando ci sono errori di trascrizione si alterano le proteine e
l'ambiente seleziona le mutazioni.
\end{itemize}

\marginbox*{
È molto complesso passare dalla sequenza alla struttura tridimensionale delle molecole ma è possibile determinare la struttura se si isolano le proteine ma la previsione non è facile.
}

\automarginpicture*{Angoli di legame $\Psi$ e $\Phi$} %4

\section{Strutture tridimensionali}

Per una struttura primaria si potrebbero avere infinite conformazioni
diverse. Questo comporta una considerevole complessità e varietà delle
possibili strutture, cosa che diventa ovvia quando si vede che le
proteine svolgono una grande varietà di funzioni. Eppure dal confronto
di 110\,000 strutture si è potuto osservare che le proteine mostrano un
considerevole grado di regolarità strutturale. Si definisce quindi:

\begin{itemize}
\item
\emph{Struttura secondaria:} conformazione spaziale assunta a livello locale
degli atomi di uno scheletro polipeptidico, senza tenere in
considerazione la disposizione delle sue catene laterali
\item
\emph{Struttura terziaria:} distribuzione tridimensionale degli atomi di un
intero polipeptide, compresa quella delle sue catene laterali
\item
\emph{Struttura quaternaria:} indica la disposizione spaziale delle catene
polipeptidiche (o subunità) nel caso una proteina sia formata da più
di una. In genere questo assemblaggio comporta dei vantaggi alla
proteina
\end{itemize}

\autofullpicture*{Struttura primaria} %5

La proteina passa dalla struttura primaria alle strutture
tridimensionali per mezzo del folding. La struttura primaria di una
proteina non ha nessuna attività, mentre le strutture tridimensionali ne
hanno.

\automarginpicture*{Formazione delle strutture secondarie} %

Le conformazioni dello scheletro polipeptidico possono essere descritte
dai loro angoli di torsione \(\Phi\) e \(\Psi\). Questi angoli si
possono diagrammare in un grafico che prende il nome di ``Diagramma di
Ramachandran''. Questo diagramma è anche una mappa delle conformazioni
permesse. Le conformazioni non permesse sono quelle in cui le distanze
interatomiche tra gli atomi non legati è inferiore ai raggi di van der
Waals.

\autofullpicture*{Mappa di Ramachandran. Le zone blu sono quelle permesse senza conflitti sterici, mentre le zone verdi hanno conflitti ma sono comunque permesse. \\Le zone bianche non sono permesse o sono poco probabili. I pallini rossi corrispondono a coppie che sono quelle che ricorrono
in residui che si susseguono a definire le strutture secondarie più
probabili.}

Le diverse coppie di angoli vengono ritrovate e riassunte nel diagramma
di Ramachandran (mappa delle conformazioni permesse) dove si
identificano diverse zone.
Le zone dei grafici in cui si ha la maggiore concentrazione sono zone in
cui si presentano strutture particolarmente stabili e identificano le
strutture secondarie.

\marginbox*{
La glicina è particolare perché presenza zone possibili molto maggiori rispetto a quello visto precedentemente e questo perché come catena laterale ha sono H e quindi non ci sono particolari problemi di ingombro
sterico (punti neri).
}

\autofullpicture*{Mappa di Ramachandran, con le struttura secondarie. } %4

Le combinazioni di angoli accettabili rappresentano le ``sottoclassi di
strutture''. Le strutture secondarie più ricorrenti sono l'\alpha-elica
e il \beta-sheet.

\clearpage

\section{Alfa-elica}

L'\alpha-elica è un elica destrorsa. Gli angoli di torsione che
caratterizzano questa struttura sono \( \Phi = -57^{\circ} \) e \( \Psi = -47^{\circ} \).
Questa struttura è particolare perché consente la formazione di legami
ad idrogeno tra i gruppi \ce{NH} e \ce{C=O}. Il passo dell'elica,
ovvero la distanza di avanzamento dell'elica per ogni giro, è di 5.4
\AA. L\alpha-elica possiede inoltre una lunghezza media di circa 12
residui.

\begingroup \autoherepicture{0.7}\captionof{figure}{\alpha-elica} \endgroup

Nell'\alpha-elica, i legami a idrogeno dello scheletro covalente sono
disposti in modo tale che il gruppo \ce{C=O} del legame peptidico
formato dal residuo n-disposto parallelo all'asse dell'elica viene a
trovano vicino al gruppo \ce{N-H} del legame peptidico formato dal
residuo n+4. In questo modo il legame ad idrogeno è molto forte perché
dista solo 2.8 \AA. Le catene laterali degli amminoacidi dell'elica sono
proiettate verso l'esterno e verso il basso per ridurre le interazioni
steriche con lo scheletro peptidico e tra di loro. Il nucleo dell'elica
è molto compatto in quanto le distanze tra gli atomi sono vicine ai
raggi di Van der Waals.

Nell'\alpha-elica, il legame ad idrogeno chiude un'ansa formata da 13
atomi, e prende il nome di \alpha-elica (3,6\ped{13} elica). Gli altri
casi possibili sono la 2,2\ped{7} elica, 3\ped{10} elica e la \pi-elica
(4,4\ped{16} elica).

Le eliche sono la forma di struttura secondaria più abbondante nelle
proteine globulari.
La mioglobina e l'emoglobina sono formate dal 70\% circa di
\alpha-eliche.

L'\alpha-elica ha un momento di dipolo dato da
\(n \times 3,5 \: \text{Debye}\), dove \(n\) è il numero di residui.
Quindi l'\alpha-elica presenta tanti momenti di dipolo orientati
parallelamente, che generano un momento di dipolo netto.

\automarginpicture*{Momento di dipolo dell'\alpha-elica}

Ci sono delle restrizioni sugli amminoacidi che formano un \alpha-elica.
I residui con catene laterali ingombranti staranno molto difficilmente
vicini (Val, Ile, Thr), e lo stesso vale per i residui con una catena
laterale avente la stessa carica.

Le catene laterali polari (Ser, Asp, Asn) possono formare legami ad
idrogeno con il gruppo peptidico del backbone ed interferire con i
legami ad idrogeno dell'elica
Le interazioni favorevoli tra le catene laterali possono stabilizzare
ulteriormente l'elica (Arg-Asp).
La prolina non è compatibile con l'\alpha-elica, in quanto la rompe
poiché rinchiudendosi su sé stessa costringe un angolo di legame che non
è compatibile con l'\alpha-elica. Per questo motivo si trova spesso alla
fine.

Teoricamente è possibile l'esistenza di un elica sinistrorsa, ma questa
struttura è molto destabilizzata dalle catene laterali, in quanto
risultano essere più vicine al backbone rispetto a quella destrorsa.

\autofullpicture*{Altre eliche possibili} %6

\clearpage

\section{Strutture beta}

Così come l'\alpha-elica, anche il foglietto \beta, o \beta-sheet,
utilizza la capacità dello scheletro polipeptidico di formare legami ad
idrogeno. Nel \beta-sheet, i legami ad idrogeno si instaurano tra catene
polipeptidiche vicine e non all'intero di una sola catena, come
nell'\alpha-elica

\begingroup \autoherepicture{0.8}\captionof{figure}{Foglietto \beta{} antiparallelo e parallelo} \endgroup

I \beta-sheet sono denominati anche foglietti pieghettati (per
ottimizzare i legami ad idrogeno). Le catene laterali sono disposte
perpendicolarmente al piano, alternativamente sopra e sotto, quindi se
si ha un abbinamento alternato tra amminoacidi polari e apolari, una
parte del foglietto sarà idrofobica, mentre l'altra sarà idrofilica.

I \beta-sheet sono formati da minimo due catene, se antiparallelo, o da
cinque catene, se parallelo. In media si hanno sei catene. La lunghezza
media delle catene è di sei residui ed il massimo è di 15 residui.

Le catene dei \beta-sheet mostrano un pronunciato avvolgimento
destrorso. Il \beta-sheet ideale dovrebbe essere planare, tuttavia nelle
proteine i \beta-sheets hanno una curvatura destrorsa probabilmente a
causa dell'interazione tra il backbone e le catene laterali. Quando si
hanno delle coppie di \beta-sheet, queste si impacchettano in maniera
allineata o ortogonalmente.
Non è possibile creare pattern di intercalazione, tranne in casi
particolari, come nella seta.

\autofullpicture*{Ripiegamento del foglietto che ottimizza i legami ad idrogeno.}

\autofullpicture*{Le catene laterali stanno alternetivamnete sotto e sopra il piano delle catene \beta. \\
Nel \beta-foglietto, ogni residuo forma un legame \beta{} con altri due residui.
}

\automarginpicture*{Avvolgimento delle strutture \beta}

\subsection{Superstrutture beta}

Quattro \beta-strands antiparalleli adiacenti sono detti \emph{arrangiamento
a chiave greca}. Questo motivo si trova nelle strutture \beta-barrel.

\automarginpicture*{Motivo a chiave greca}

\automarginpicture*{Altre superstrutture \beta}

Ci sono altre strutture supersecondarie che devono essere ricordate,
come il motivo \beta\alpha\beta{} in cui un \alpha-elica unisce due
filamenti paralleli di un \beta-foglietto. Questo motivo si trova spesso
in proteine che legano nucleotidi ed è detto ``Rossman fold''. Il motivo
\beta-\alpha-\beta{} contribuisce al barile \alpha/\beta, e può essere
visto come un \beta-barrel avvolto da \alpha-eliche. Questo
arrangiamento si chiama ``Tim fold'', che prende il nome dalla struttura
supersecondaria dove è stato visto per la prima volta, ovvero l'enzima
Trioso Fosfato Isomerasi.

\automarginpicture*{\beta-turn}

Ci sono anche altre strutture supersecondarie, come gli \alpha-\alpha{}
corner, oppure un fascio di quattro eliche, che fornisce una stabilità
della proteina

Le proteine possono essere divise in

\begin{itemize}
\item
\emph{All \alpha:} costituite esclusivamente da \alpha-eliche separate da
brevi tratti di collegamento. Le strutture supersecondarie che possono
formarsi sono il fascio di eliche e \alpha-\alpha{} corner
\item
\emph{All \beta:} non sono presenti \alpha-eliche, ma solo strutture \beta,
come il \beta-barrel e le chiavi greche.
\item
\emph{\alpha/\beta:} sono costituite prevalentemente da strutture
supersecondarie del tipo \alpha/\beta, come il Rossman fold e il TIM
barrel
\item
\emph{\alpha+\beta:} sono un misto delle due strutture \alpha{} e \beta{}.
Contengono una quantità significativa di entrambe le strutture
\alpha{} e \beta{} isolate, ovvero non interagenti tra loro
(altrimenti la struttura sarebbe \alpha/\beta).
\end{itemize}

\automarginpicture*{Superstrutture più comuni}

\automarginpicture*{Impilamento di \beta-sheets}

\clearpage

\section{Riempimenti e gomiti}

\automarginpicture*{Fascio di \alpha-eliche} %8

I segmenti polipeptidici contraddistinti da una struttura regolare, come
le \alpha-eliche oppure i \beta-strand, sono uniti da tratti di catena
polipeptidica che cambia improvvisamente direzione. Questi ripiegamenti
\beta{} chiamati così poiché collegano filamenti successivi di foglietti
\beta{} anti-paralleli coinvolgono nella maggior parte dei casi quattro
residui amminoacidici disposti principalmente in due modi (tipo I e tipo
II), che si differenziano tra loro per una rotazione di 180° dell'unità
peptidica che unisce i residui 2 e 3. Entrambe sono stabilizzate da un
legame a idrogeno.

Le strutture \alpha{} e \beta{} rappresentano in media la metà della
proteina; le rimanenze possono assumere forme a coil, loop o essere
tratti disordinati (random coil).

\begingroup\autoherepicture{0.7}\captionof{figure}{Ci vogliono circa 3.5 residui per fare un giro (passo).
Ci sono quindi sette posizioni nella ruota (coperta in due giri).} \endgroup

Spesso capita che la struttura ad \alpha-elica sia formata da
pseudo-ripetizioni di sette residui (a-b-c-d-e-f-g) in cui i residui non
polari prevalgono in corrispondenza delle posizioni ``a'' e ``d''.
Poiché un \alpha-elica possiede 3.6 residui per giro, i residui ``a'' e
``d'' si allineano lungo un lato di ciascuna \alpha-elica. La striscia
idrofobica lungo un'elica si associa con quella idrofobica di una
seconda elica.

\autofullpicture*{Varie tipologie di avvolgimento avvolto}

Le due eliche si avvolgono in quanto la ripetizione, ovvero 3.5 residui
nella sequenza amminoacidica, è leggermente inferiore al numero di
residui per giro, ovvero 3.6 residui. L'incrocio avviene a circa 30° per
massimizzare le interazioni tra le due catene.

Questo avvolgimento tra due \alpha-eliche viene chiamato ``avvolgimento
avvolto'', oppure ``coiled coil''. Questo è il caso
dell'\alpha-cheratina.

\autofullpicture*{Le eliche si avvolgono una sull'altra per ridurre al minimo l'esposizione delle catene laterali amminoacidiche, che sono idrofobe, all'amibiente acquoso.}

Sono permesse anche interazioni tra più di due eliche, come nel caso del
collagene, che è formato da una tripla elica.




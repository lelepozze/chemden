\chapter{Proteine fibrose}

\ChangePicturesFolder{5}

Non sono stati definiti elementi per la struttura terziaria. In realtà,
la struttura secondaria svolge un ruolo determinante nella struttura
terziaria.

Le proteine fibrose hanno la tendenza di formare strutture secondarie e
sono in genere utilizzate dalle cellule, o dai sistemi cellulari, per fornire
proprietà meccaniche (resistenza).

Questo è dovuto alle strutture secondarie che si estendono per tutta la
molecola. Le interazioni sono del tipo di foglietto.

Sono insolubili in acqua, e quindi sono adatti per fare strutture che
crescono in comparti acquosi. Hanno molti amminoacidi idrofobici.
Tendono a formare strutture sopramolecolari. Si trovano sia nella
matrice extracellulare sia nelle cellule. Negli spazi extracellulari si
trova il collagene; si trova in tendini in cornee. È una componente
importante dell'osso, dell'epidermide, nelle unghie.
Alcune proteine sono il collagene, l'\alpha-fibroina e l'\alpha-cheratina.

Le altre proteine si trovano nelle cellule, come la tubulina, che forma i filamenti che
danno forma alle cellule (citoscheletro). L'\alpha-cheratina può assumere
strutture più o meno rigide a seconda della struttura più o meno rigida.
A seconda della destinazione assumono sequenze caratteristiche.

\section{Cheratina}

La cheratina è presente nei corni, nei capelli e nella lana. Si trova anche
all'interno dei filamenti del citoscheletro.

Tutte le \alpha-cheratine, hanno in comune l'adozione della \alpha-alicha e
formazione di coil (eliche che tendono a formare sovrastrutture).

Le \alpha-eliche si sovrappongono per interazioni idrofobiche per
minimizzarle. Si usano \alpha-eliche destrorse, che si super-avvolgono in
senso sinistrorso.

Nel capello sono presenti \alpha-eliche, che formano microfibrille e
microfibrille, il cui elemento è l'\alpha-elica.

Le fibrille sfruttano come fattore di resistenza allo sfaldamento il
ponte disolfuro. Sono presenti molte cisteine che formano legami S-S. A
seconda delle proprietà meccaniche richieste sono presenti più o meno
ponti disolfuro.

I ponti disolfuro non resiste alla riduzione, ma all'ossidazione si. Le tarme riescono
a rompere i ponti disolfuro (tramite mercaptani che riducono i ponti
s-s), quindi riescono a mangiare la lana

Le disposizioni dei dimeri (coils) formano i protofilamenti, con un
interazione test\alpha-testa tra diversi coils. Le interazioni fibrose sono
formate dai dimeri (si formano legami di disolfuro).
I protofilamenti si associano a dare microfibrille e i ponti disolfuro fanno unire le microfibrille.

\automarginpicture*{Formazione dei dimeri, protofilamenti e microfibrille.}

I parrucchieri sfruttano la rottura dei ponti disolfuro nativi e
costringendo (ossidando) in posizioni diverse riescono a mantenere la
forma, che risulta permanente.

\autoherepicture{0.8}

Il ponte disolfuro si forma anche nel dimero, a seconda delle
caratteristiche richieste.

\section{Fibroina}

È l'elemento dominante nella seta. È anche componente nelle ragnatele,
in combinazione con altre proteine.
La fibroina dona molte proprietà meccaniche alla seta.

La fibroina presente presenta vari \beta-sheet, che in realtà si impilano.
Si formano strati di foglietti \beta, che resistono dal punto di vista
meccanico (sono molto estesi), però sono flessibili, perché sono tenuti
insieme da legami deboli.
Questo comportamento è consentito dalla sequenza, in quanto è una sequenza ricorrente.

\autofullpicture*{La struttura è formata da \beta-sheet antiparalleli.}

La sequenza primaria è (-Gly-Ser-Gly-Ala-Gly-Ala-)\ped{n}, quindi è presente una glicina ogni due, alternata da alanina o serina, che sono simili per dimensione.

Una caratteristica del \beta-sheet è che le catene laterali dello strand
puntano alternativamente sopra e sotto il foglio.
L'alternanza di amminoacidi piccoli consente di fare l'intercalazione,
ovvero consente di abbassare il limite di vicinanza dei beta sheet; si
incastrano e si intercalano tra di loro.
Questa possibilità di intreccio consente di stretchare il \beta-sheet e
consente anche la flessibilità perché ci sono interazioni deboli.

\autofullpicture*{Intercalazione}

Se in sequenza ci sono amminoacidi più ingombranti, come tirosina, la
zona diventa amorfa; questo è previsto dalla sequenza. Queste zone diventano
più sfaldate e più flessibili e quindi consentono zone di maggiore
flessibilità. Questo consente di avere delle parti disordinate
intervallate a parti ordinate.

\clearpage

\section{Collagene}

Il collagene è una struttura dominata dalle interazioni secondarie. È una proteina che si
trova nello spazio extracellulare, per organismi pluricellulari.

Il collagene è una proteina importante per i vertebrati, in quanto dona resistenza alle ossa dei denti,
ai tendini e ai legamenti. Dona anche elasticità alla pelle, ai vasi sanguigni, alla cornea e
alle cartilagini. Quindi fornisce sia resistenza sia elasticità
Costituisce 1/4 della massa totale di un generico animale

L'unità di base non è una \alpha-elica (si veda mappa di Ramachandran), ma una elica sinistrorsa tripla.
Questa elica è stabilizzata dai legami ad idrogeno, quindi possiede un elemento di
struttura secondaria per tutta la lunghezza della proteina.

Si tratta la proteina come tripla elica, che ha passo molto elevato (3 residue
per giro). Queste eliche sono sinistrorse ma si avvolgono in forma destrorsa.
Si formano fasci di triple eliche, che formano fibre macroscopiche.

\begingroup\autoherepicture{0.5}\captionof{figure}{Tripla elica del collagene} \endgroup

Dal punto di vista della struttura è necessario avere una glicina ogni
tre amminoacidi (Gly-X-Y-Gly-etc). Anche le X e Y non sono a caso. X è
spesso una prolina e la Y è spesso un amminoacido modificato, come l'4-idrossi-prolina. Non è codificato dal DNA, ma viene sintetizzata
dopo.
La Gly è fondamentale affinché la tripla elica si formi.

\autofullpicture*{Sequenza della tripla elica}

Nella tripla elica si formano i legami ad idrogeno H. La \alpha-elica invece non ha
legami H tra le due catene.
Ci sono vari tipi di collagene, cambia X e Y, ma non Gly.

\autofullpicture*{Rappresentazione della tripla elica.}

Quando viene sintetizzato il collagene viene
sintetizzato con la prolina. Si necessitano delle reazioni che formano
l'idrossiprolina.

\automarginpicture*{4-idrossi-prolina}

La reazione prevede che la prolina in catena prevede un consumo di
\alpha-chetoglutarato che reagisce con una reazione con \ce{O2} e di \ce{Fe^{2+}}. La reazione
prevede che del ferro si ossidi; un prodotto è la \ce{CO2} con l'ossidazione
del carbonio del gruppo ossidrilico. Il \ce{Fe^{2+}} viene ripristinato da \ce{Fe*{3+}} tramite l'utilizzo di
ascorbato.

\autoherepicture{0.8}

Questo spiega perché i marinai che non assumevano vitamina C (acido
ascorbico/ascorbato) la reazione non poteva andare; si ha una deficienza di
idrossiprolina. Si prendeva lo scorbuto. La carenza del ripristino di
\ce{Fe^{2+}} comporta lo scorbuto.

L'idrossiprolina ha i gruppi \ce{OH}, che servono per fare legami H tra
catene (triple eliche). È un amminoacido modificato che serve a dare
robustezza alla tripla elica.

All'interno delle catene si hanno dei legami ad idrogeno tra la glicina, con il gruppo \ce{NH}, e il gruppo carbossile di X. La glicina punta verso l'intero della tripla elica, le proline
puntano verso l'esterno. La tripla elica si forma grazie alle
interazioni della glicina.

Le idrossiproline legano le tre eliche (più triple eliche insieme) tra
loro, mentre le glicine legano la tripla elica insieme.
La singola elica non è molto stabile, ma la tripla elica è stabilizzata, quindi non si riesce a strecharla, perché è poco flessibile.

La Gly è necessaria; se ci fosse dell'alanina, la formazione della
tripla elica non è consentita (causa una malattia).

Le interazioni principali sono tra più triple eliche. Se si aumentano le
idrossiproline, si ottiene una struttura più stabile (solida). Non serve
solo questo, però.
Le triple eliche sono sintetizzate come pre protocollagene, così come è
codificata dal DNA.

In seguito cominciano le idrossilazioni. Si può idrossilare anche la
lisina e si possono aggiungere anche degli zuccheri.
Una volta che queste reazioni avvengono, si ha il protocollagene, questo
esce dalla membrana cellulare, dove verrà ulteriormente modificato.

L'enzima per l'idrossilisina funziona allo stesso modo. Si forma la
allisina. La lisina idrossilata può essere essere
glicosilazione; serve perché la proteina diventa più solubile. Può anche
avere la stessa funzione della idrossi.

\vfill
\pagebreak

\autoherepicture{0.8}

Si possono avere interazioni tra le triple eliche per fare legami
covalenti.
Si forma l'allisina; serve perché si può formare (per due
vie) un legame covalente tra lisina o allisina (cambia la reazione).

In questo modo le triple eliche sono tenute insieme in modo
indissolubile. Questi legami sono chiamati cross-link.
Questo consente al fascio di essere un tutt'uno e di non essere
separabile.
Queste danno rigidità e a seconda del tessuto, il numero dei legami
viene controllato.
Un numero maggiore di questi legami comporta l'invecchiamento, in quanto
aumenta la rigidità, quando in realtà servirebbe l'elasticità.

Le tre catene possono essere dello stesso tipo, o può cambiare a seconda
della necessità. A seconda del tessuto, si hanno diverse composizioni
della tripla elica.
Le proprietà non guardano la singola elica, ma più la tripla elica.

Il procollagene, una volta uscito, viene tagliato alle estremità. Una
volta posizionato, comincia l'interazione tra le triple eliche.
Quindi si vede che la struttura secondaria è estesa a tutta la proteina.

\marginbox*{
Il collagene si trova anche nei tendini. Ci sono delle
sindromi che sono legate a questi. Le articolazioni sono difettose, sono
troppo elastiche. Si pensa che Paganini soffrisse di questa malattia.
}

Il collagene viene utilizzato dalle ossa. Nelle ossa è presente il
calcio; sono presenti anche delle cellule.
Si ha una matrice extracellulare, che è il collagene. Le cellule si
devono depositare attorno alla matrice extracellulare.
C'è anche tutta la parte dei cristalli di calcio, che si depositano
sulla matrice, e forniscono la durezza dell'osso. I cristalli sono di
idrossiapatite.

Il collagene è la matrice dove cresce l'osso. Ci sono delle malattie che
intaccano la formazione dell'osso, e le rendono fragili. Riguardano
difetti genetici del collagene. (Anche di una sola glicina).
Si genera un difetto locale, ma se è esteso a tutte le catene diventa un
difetto devastante per le ossa.

\clearpage

\autofullpicture*{Sintesi del collagene all'interno della cellula.}

\autofullpicture*{Sintesi del collagene fuori dalla cellula}

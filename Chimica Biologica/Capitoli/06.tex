\chapter{Classificazione dell'architettura proteica}

\ChangePicturesFolder{6}

La struttura primaria è vista come la successione di amminoacidi nella
catena.
Le strutture secondarie sono particolarmente stabili per via della
continuità di alcuni amminoacidi.

Le strutture secondarie possono assemblare in super strutture secondarie
(viste in questa lezione: struttura quaternaria e con più monomeri di
proteine).
La funzione vera e propria viene data dalla struttura terziaria.
La somma delle strutture terziarie forma la struttura quaternaria;
questa struttura ha più funzioni rispetto alla struttura terziaria.
Ci sono anche dei domini.

La struttura primaria è quella che esce dai ribosomi, poi viene
ripiegata e viene assemblata in una struttura secondaria e terziaria.
La struttura primaria indica alla proteina come foldare, quindi la
proteina ``sà'' come deve ripiegarsi.
Ci sono tanti tipi di proteine, dalle globulari alle tubolari.

Tutte le strutture sono state determinate sperimentalmente; gli
strumenti utilizzati sono la diffrazione a raggi X (sui monocristalli
delle proteine)
Oggigiorno, si usa la cryo-electron microscopy, che permette di fare, a
bassa temperatura, un'indagine e di ottenere una mappa di densità
elettronica e permette di ricostruire la struttura.
Si può anche usare l'NMR, che però ha dei limiti sulla grandezza
determinabile.
Questo è importante per correlare la struttura alla funzione.

La struttura dipende dalla funzione, che a sua volta dipende
dall'ambiente. Quindi c'è una correlazione tra struttura e intorno
chimico.
Una proteina è codificata geneticamente per ripiegarsi e svolgere una
funzione ad una determinata condizione. Se mancano le condizioni, la
funzione e la struttura vengono a mancare.
La struttura terziaria viene raggiunta quando la proteina assume una
conformazione che è determinata dalla termodinamica del processo e dalla
cinetica del processo, per quanto riguarda il ripiegamento.

Spontaneamente, la struttura raggiunge quella conformazione,
(termodinamica) e lo raggiunge in un breve periodo (cinetica).

L'entropia e l'entalpia di questo processo sono favorevoli. La struttura
porta vicine parti della catena distanti, in quanto si instaurano delle
interazioni deboli tra la catena Le interazione sono il legame H, come
per la struttura secondaria. Anche le strutture secondarie svolgono una
funzione nel ripiegamento.

\autofullpicture*{Si possono anche avere legami covalenti, ponti \ce{S-S} e anche interazioni
ioniche.}

L'assunzione della struttura terziaria è legata al confinamento
all'interno delle parti idrofobiche (nelle proteine idrosolubile). Si ha
quindi un core ben schermato dal solvente, mentre le catene polari sono
molto distribuite sulla superficie.

\autoherepicture{0.8}

Le proteine possono anche avere dei cofattori o leganti. Questi
instaurano legami con i residui della proteina.
Come si vede nell'immagine \ref{img:cofattore}, il cofattore ha accesso alla tasca, e quindi possono scambiare. Altri
sono legati durante il folding.

\autofullpicture{Esempio di una proteina che lega una molecola di cAMP.{} Il cofattore legato alla catena carboniosa. La molecola ha un gruppo fosfato, una base l'adenina e uno zucchero. Questi atomi possono
formare legami H, con i residui, o anche interazioni di carica ad
esempio, del fosfato con l'arginina.}{img:cofattore}

\autofullpicture*{Le proteine possono essere rappresentate in diversi modi.\\ (a) Si visualizzano le parti delle strutture secondarie o (b) talvolta si guarda attraverso una sezione.{} (c) Superficie con le cavità. (d) catene laterali. Le proprietà specifiche e locali vengono dalle catene; il carattere locale di una macromolecola è conferito dalle catene laterali, ad esempio un enzima presenta un sito attivo. (e) struttura reale}

Lo stato nativo di una proteina è lo stato conformazionale, che permette alla proteina di svolgere la
sua funzione biologica.
Lo stato nativo è legato alla funzione ed è necessario mantenerlo.
È caratterizzato da avere una temperatura di fusione, ovvero lo stato
nativo si conserva per un range di temperatura, però ad una certa
temperatura, lo stato cambia in modo cooperativo, insieme. Si perde
quindi la struttura.
Esiste anche un punto isoelettrico e una risposta del pH. Quindi la
funzione dipende anche dal pH. Se il pH non è consono, la proteina non
funziona, a seconda dell'ambiente e della funzione

Solo se la proteina è allo stato nativo svolge la sua funzione. Ad
esempio, un enzima detto papaina, la funzione viene svolta nel sito
attivo.
Ad esempio, il sito attivo della papaina è vicino ad una cisteina. La
proteina dopo la sua funzione deve ripristinarsi, senno non è un enzima
(catalizzatore).

Un altra funzione è quella di trasportare qualcosa. Ad esempio, il
canale potassio; le \alpha-eliche sono immerse nella membrana. La specificità
è data dalla dimensione della terna di amminoacidi, lascia passare solo
potassio per le dimensioni (G-Y-G)
Dalla struttura si può riconoscere la funzione della proteina.

Nei recettori, ci sono due parti, visto che la proteina è intermembrana.
Si producono quindi delle variazioni esterne che poi vanno a modificare
la struttura (allosteria) che si ripercuote in modifiche interne.

Generalmente, nelle proteine idrosolubili, gli aa idrofobici sono nel
core. Gli aa non carichi o polari, sono presenti all'interno per fare
una funzione specifica, catalitica o da cofattore.
Ad esempio, l'anidrasi carbonica, l'istidina polare, serve a coordinare
lo zinco per la catalisi.
Gli amminoacidi possono anche essere distanti, perché il folding le avvicina.

Date alcune definizioni:
\begin{itemize}
\item \emph{Omologo:} a livello di sequenza, sono proteina che hanno un'identità
uguale almeno al 25 \%, quindi si può identificare un progenitore
comune. Un'omologia tale, consente di dire che le proteine avranno
molte similitudini a livello di struttura, per via dell'ingombro
sterico simile. A volte si assomigliano più le strutture delle
sequenze. Perfino a livello della stessa specie, ci sono delle
proteine che sono codificate in modo leggermente diverso, però fanno
la stessa funzione (polimorfismi).
\item \emph{Domini:} Strutturalmente si possono definire delle zone diverse, che fanno parte di due parti
diverse della proteina. I domini hanno funzioni diverse, ad esempio, possono legare specifiche
molecole.Il backbone è lo stesso. In proteine diverse, i domini possono ricorrere
in diverse proteine. Ad esempio, se si lega il nad, si avrà un dominio
che lega il nad. Non sono catene distinte, ma sono inserzioni di diverse
proteine.
Se un dominio è ricorrente, allora anche la parte di DNA codificante è
ripetuta nel gene.
\end{itemize}

\autofullpicture*{Esempio di una proteina con tre domini}

\section{Struttura quaternaria}

Talvolta le proteine si mettono insieme; al posto di mettere insieme dei
domini, si mettono catene diverse, quindi si ottiene la struttura
quaternaria (o complesso proteico.)
Si mettono insieme delle subunità, dette protomeri.
Le subunità possono essere identiche (omodimero, etc) o diverse.

Se succede, ci deve essere un vantaggio biologico. Se si mettono insieme
acquisiscono delle proprietà migliori. Si possono mettere insieme anche
più sensori, per avviare una reazione (negli enzimi).
Le catene sono unite da interazioni deboli.
Se c'è la funzione quaternaria, si hanno funzioni elevate, sono molto
cooperative (tra le parti).

\autofullpicture*{Esempio di interazioni deboli nell'alcol-deidrogenasi. Si possono formare \beta-sheet con
parti di proteine diverse. C'è anche lo stacking. Nella sezione si vede il sito di complementarietà tra le due subunità.}

\autofullpicture*{Proteina con molti monomeri, ovvero una chaperonina. È una proteina che
permette il folding di altre proteine. La tasca serve a inglobare altre proteine all'interno}

\autofullpicture*{In questa proteina si hanno diversi monomeri, alternati. Questo è un multimetro, si
ha quindi una macchina molecolare di ATP.}

\chapter{Folding delle proteine}

Per capire il processo di folding è necessario guardare la termodinamica.
Il $\Delta G$ deve essere negativo, ed inoltre la cinetica è molto veloce.

Per molti anni, uno dei sogni dei biochimici, è quello di prevedere la
struttura in base alla sequenza.
Non tutte le proteine si sà la struttura, anche si sa la sequenza. In
parte perché vengono modificate in dopo, perché è più facile da
riconoscere, rispetto alle proteine.

Nel DNA si possono riconoscere le proteine (zone codificanti), che però
non è detto che vengano codificate subito o nel corso di una breve fase.
Si utilizzano degli algoritmi, per codificare la struttura delle
proteine. Il salto c'è stato dal programma AlphaFold, che è un programma
basato su AI, che sfrutta le proprietà di apprendimento della macchina.

Le capacità sono state testate alla cieca. Su alcuni campionamenti, non
totalmente noti, si possono fare delle previsioni sulla struttura. È
molto usato. Si usa un database come base di dati, per fare il machine
learning. In seguito, con questi dati, si possono prevedere delle
strutture e quindi delle funzioni.

Quindi è necessario guardare il minimo di energia.
Dai dati si vede che il risultato netto è che ci sono diversi
contributi (anche se il $\Delta G$ è negativo, però è piccolo).
Molte proteine devono anche variare la conformazione, quindi se si
avesse un minimo profondo non si potrebbe spostare.
I due contributi sono molto simili tra di loro, e per entrambi sono
presenti termini sfavorevoli e favorevoli Ci sono due contributi

\begin{itemize}
\item
\emph{Entropico:} è un contributo svantaggioso, in quanto la proteina si
``riordina'', quindi c'è una perdita di entropia nella proteina. Il
termine è controbilanciato dal rilascio di molecole d'acqua, che
causano disordine nel sistema.
\item
\emph{Entalpico:} i termini favorevoli sono le interazioni che si sviluppano
nella proteina una volta foldata. Il termine sfavorevole è dato
dall'interazione della proteina dall'acqua.
\end{itemize}

L'effetto che si raggiunge è $\Delta G$ negativo, che come detto prima è
piccolo.

Il comportamento deve essere flessibile, però può essere anche un
vincolo. Operando a condizioni diverse, il $\Delta G$ può cambiare, che
comporta la denaturazione della proteina.

Il fatto che esista la spontaneità del processo è visto sia in vivo che
in vitro. Il che significa che si può passare da una forma denaturata ad
una forma corretta, spontaneamente. Il processo di folding è il processo
dove la proteina passa da una forma disordinata ad una forma funzionale.
Il processo inverso è definito una denaturazione, che non è l'idrolisi o
degradazione (taglio degli amminoacidi).

I fattori denaturanti possono essere fisici T e p, o fattori biochimici,
pH, e l'ambiente (se si usano solventi), in quanto la struttura decade e
la proteina si denatura.
Alcuni denaturanti competono per il legame H, come l'urea.
La proteina in questo senso conosce la sua struttura.
Ad esempio, RNasi A, si può denaturare per l'utilizzo di
2-mercaptoetanolo, che rompe i legami \ce{SS}, o si può far perdere il legame
ad idrogeno, tramite l'urea.

Se si prende la proteina denaturata (dopo aver utilizzato un agente
denaturante) e si mette la proteina in ambiente fisiologico (si mette
quindi la proteina in un ambiente denaturante), la proteina riprende la
sua struttura nativa.
Se si ha un'interazione in un ambiente, la proteina instaura le
interazioni che si devono instaurare, e quindi la proteina riprende la
sua struttura.

Per i ponti disolfuro sono più casuali, e si instaurano quando le
cisteine sono sufficientemente vicine. In vivo, i ponti \ce{SS} si possono
instaurare in posti sbagliati, però esistono le chaperonine che tagliano
e controllano la forma della proteina.
Non sempre si possono recuperare le proteine, nel senso che a volte si
possono instaurare altre interazioni con altri frammenti di proteine, 
che a volte sono irreversibili. Le aggregazioni sono molti difficili da
separare.
Sono necessarie le condizioni corrette, con i giusti enzimi o il giusto
ambiente e, in teoria, si potrebbe sempre tornare indietro dalla
denaturazione.

Nel processo di denaturazione, esiste una certa temperatura in cui il
processo di denaturazione è cooperativo. Il processo va a sigmoide, nel
punto di flesso c'è il punto di fusione. L'intervallo è stretto.
Il processo è cooperativo (in entrambi i sensi) e cioè avviene la
denaturazione in certi centri più che in altri. Le forze agiscono in
modo cooperativo e vengono meno allo stesso modo.

\autofullpicture*{La temperatura di fusione dipende dalle forze deboli della proteina.}

\section{Cinetica del ripiegamento}

Per considerare il processo di ripiegamento, si deve considerare il
paradosso di Levinthal. Se una proteina non foldata dovesse trovare la
sua forma causalmente, servirebbe un tempo enorme, perché le
conformazioni possibili sono moltissime, statisticamente parlando.
Su una proteina con 100 amminoacidi si considerino due gradi di libertà
possibili (\psi{} e \phi). Ogni angolo ha tre conformazioni accessibili.
Le conformazioni possibili sarebbero \(3^{2 \cdot 100}\).

Se le conformazioni possono essere esplorate ad una velocità di $10^{13}$
al secondo, il tempo richiesto sarebbe $8 \cdot 10^{74}$ anni per
trovare la conformazione giusta.
Quindi è necessario che sia presente un cammino preferenziale.
Questo è stato preso in considerazione dal punto di vista sperimentale.
Anche questo è un processo cooperativo.

Si hanno quindi quatto fasi:
\begin{itemize}
\item \emph{Nucleazione:} si osservano dei processi di nucleazione, ovvero si ha una zona che folda
prima e queste zone consentono di velocizzare la foldatura di altre
zone. Le zone di nucleazione sono elementi di struttura secondaria. Ad
esempio, l'\alpha-elica serve per nucleare.
\item \emph{Burst iniziale:} buona parte delle strutture viene foldata entro pochi
millisecondi. La sequenza stessa causa questo
\item \emph{Collasso idrofobico:} nel folding c'è il collasso idrofobico, che causa il core a sprofondare
al centro. Questo causa la formazione del molten globule, il globulo o core centrale,
che è già una buona formazione
\item \emph{Formazione dei legami:} in seguito si formano tutte le altre interazioni, come ponti disolfuro o
altro
\end{itemize}
Si ha un cammino di ripiegamento ad imbuto.
In vivo, questo può avvenire all'uscita della proteina dai ribosomi.

\autofullpicture*{Cammino ad imbuto}

\section{Fasi del folding}

Si è visto che ci sono diverse fasi del folding. Ci sono diversi
percorsi, che raggiungono un minimo di energia. In una prima fase, c'è
la nucleazione del folding, sulle strutture secondarie. È la fase veloce
del processo.
In seguito, si ha il collasso idrofobico, dove le parti idrofobiche si
avvicinano al centro. In questa parte, la proteina espelle acqua.
In seguito, si ha la produzione del molten globule, che è la struttura
iniziale del folding; in seguito, si ottengono le proteine foldate.

C'è una propensione di formare elementi di struttura secondaria, a
seconda degli amminoacidi.
Quindi alcuni amminoacidi diventano importanti per fare il primo
processo di formazione dell'elemento secondario.

\autofullpicture*{Propensione degli amminoacidi a ricorrere nelle strutture secondarie}

Durante il folding, ci sono processi più lenti, a seconda della
proteina. Come ad esempio, la formazione di ponti disolfuro nel modo
corretto ed ad esempio l'isomerizzazione della prolina.
Alcune volte serve la struttura cis, altre la struttura trans.

Una volta finiti i processi lenti, si possono instaurare le interazioni
per la struttura quaternaria.
Si ha un restringimento del panorama energetico lungo il ripiegamento.
Si ha un minimo alla fine, però si possono anche avere minimi locali,
nel corso del folding; i minimi locali possono essere superati per
ottenere la struttura finale.

Il folding può essere fatto sia in vivo che in vitro. Nelle simulazioni
si simula un ambiente in vitro.
In vivo avviene la stessa cosa, solo che avviene subito dopo la sintesi
nel ribosoma. Gli elementi di struttura secondaria possono anche
formarsi durante la sintesi, come ad esempio, \alpha-elica.

In vitro, la concentrazione, non è troppo elevata. Si ha un buffer, ma
non altri componenti. Ben diverso è il folding in vivo, dove vicino ai
ribosomi ci sono un sacco di altri elementi, come altre proteine, acidi
nucleici e lipidi. Il materiale presente nel citoplasma è \(\sim\) 340
mg/mL. Questo ambiente può interferire con il processo di folding; questo
si vede in particolare se si hanno zone idrofobiche, che non consentono
l'appaiamento di acqua.

Ci sono vari modi per l'assemblaggio in vivo.
Le proteine piccole, si foldano immediatamente dopo la sintesi.
Altre, più grandi vengono coinvolte le Hsp, che proteggono le zone
idrofobiche durante la sintesi, in modo da non consentire l'interazione
con zone non volute.
Circa l'85\% delle proteine vengono protette.
Il 15\% hanno bisogno di catalizzatori, serve un aiuto cinetico, se le
proteine sono grandi. Più sono grandi, più possono subire il processo di
misfolding.

\autofullpicture*{Reazioni schematiche per il ripiegamento delle proteine}

Le proteine in questione utilizzano i chaperon molecolari, che aiutano i
processi.
La struttura dei chaperon molecolari è formata da un cappello e da un
cilindro; si ha un confinamento spaziale, che protegge ed aiuta il
folding della proteina.
I chaperon si aprono e fanno entrare le proteine. In seguito, si
chiudono e foldano la proteina.
Quando l’ATP non serve più, fa da timer per consentire l'uscita della
proteina. L'ATP serve perché serve dell'energia.

Il processo può anche portare al misfolding, però può essere
riarrangiata. Se la proteina in un numero di tentativi finito, non
riesce a foldarsi correttamente, allora viene eliminata. In primis la
proteina viene marcata (tramite ubiquitina) e poi viene distrutta (nel
proteosoma). C'è anche questo livello di controllo del misfolding.

Ci possono essere più o meno aggregati (di proteine misfoldate). Alcune
forme di misfolding, si possono avere delle malattie, se la zona di
misfonding è estesa. Le proteine misfoldate danno
luogo a delle interazioni specifiche, che possono dare luogo a
interazioni estese.
Ad esempio, le placche del morbo di Alzheimer sono proteine precipitate, misfoldate.
Si creano dei processi di iniziale assunzione di conformazione
sbagliate, che a loro volta fanno catalisi di proteine misfoldate.

\autofullpicture*{
La proteina PrP ha una conformazione. Se è misfoldata, si ottiene la
PrPc, che catalizza la formazione di altre PrPc. Questo diventa un
processo a cascata. Le proteine in seguito tendono ad aggregare.\\
Al posto della formazione dell'\alpha-elica, si ha la formazione di un
\beta-sheet. La zona esposta all'esterno della proteina è idrofobica e
darà interazioni con altri \beta-sheet di altre proteine idrofobiche.\\
Quindi si ha la formazione di fibrille (zone estese); le fibrille poi
invadono le cellule e causano il morbo.}

I processi di controllo con il tempo, diminuiscono e questo causa
l'accumulo delle proteine misfoldate.
La struttura della proteina non è ferma, ma è mobile. Le immagini
derivate dalla diffrazione dei raggi-X sono immagini ferme. Man mano che
la le molecole crescono, si hanno più gradi di libertà, che consente
alla proteina di muoversi.

Alcune zone di loop, o disordinate, sono le più mobili. Si ha la
possibilità di moti diversi, che individuano diversi movimenti.
Si possono avere:
\begin{itemize}
  \item Le vibrazioni atomiche
  \item Nelle proteine si possono avere strutture secondarie che si muovono. 
  \item anche le truttura terziarie (subunità) si possono muovere.
\end{itemize}
Quando si arriva a questo livello di movimento, si ha un movimento di
assemblaggio di subunità, o interazione con partner di reazione.

Le scale temporali sono diverse, a seconda del moto e a seconda della
complessità della proteina. A seconda del tempo, si può avere un legame
con un substrato.
Ad esempio, nella mioglobina, si ha un movimento che permette di
rilasciare ossigeno, causato da un istidina. Il moto prevede dele
istantanee, che consentono di visualizzare il movimento delle parti
delle molecole. Queste sarebbero le immagini corrette.

Come si è visto, le conformazioni hanno un minimo di energia. Quando la
proteina è legata ad un cofattore, si possono avere delle variazioni
strutturali importanti, come ad esempio nella calmodulina, che regola il
calcio libero.


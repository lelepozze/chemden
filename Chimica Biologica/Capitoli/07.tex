\chapter{Rapporto struttura-funzione}
\markboth{Rapporto struttura-funzione}{Rapporto struttura-funzione}

\ChangePicturesFolder{7}

Qual'è il \emph{rapporto struttura-funzione} delle proteine?
La funzione di una proteina è legata alla sua funzione. Quindi è
necessario andare per esempio, per capire la funzione di una proteina.
Un esempio particolare è tra l'emoglobina e la mioglobina Hanno
struttura abbastanza simile, ma non identica. Sono trasportatori di
ossigeno; la differenza strutturale comporta il comportamento diverso
delle due.

L'ossigeno è importante nel metabolismo aerobico. L'ossigeno viene
impiegato nella fosforilazione ossidativa (sintesi di ATP)(respirazione
cellulare). O meglio, serve alla catena di trasporto di elettroni,
quindi come accettore finale di elettroni.
La solubilità di ossigeno nel sangue non è elevata, e quindi non si può
contare su questo. È necessario aumentare la solubilità dell'ossigeno,
quindi si utilizzano queste proteine per legare l'ossigeno. Si trovano
nel sangue e in gran quantità, per consentire di avere una grande
disponibilità.

Queste proteine si caricano di ossigeno nei polmoni, e attraverso la
circolazione, l'ossigeno viene trasportato attraverso i tessuti. Quindi
le proteine devono legare l'ossigeno e poi lo devono rilasciare.
In seguito, devono anche trasportare l'anidride carbonica.

La mioglobina ha una funzione di immagazzinamento, quando serve più
ossigeno, ad esempio quando c'è uno sforzo muscolare.
La \ce{CO2} è il risultato della glicolisi, del metabolismo dei carboidrati.

L'emoglobina è composta da quattro subunità, mentre la mioglobina ne ha
solo una. Le quattro subunità sono simili alla mioglobina.
La parte importante del trasporto viene effettuata dai globuli rossi,
cellule che sono piene di emoglobina.
Il cuore dell'emoglobina è il gruppo eme. Il gruppo eme è un gruppo prostetico,
un cofattore esterno, che lega il ferro ad una protoporfirina. L'eme
è all'interno di una tasca della proteina.
Il ferro è legato ai quattro azoti della protoporfirina. La struttura è
planare. Il ferro si coordina e complessivamente l'eme è neutro.

\autohalfpicture*{Protoporfirina e gruppo eme}

Il ferro deve essere Fe(II) ed è necessario che non si ossidi. Questa è una cosa
interessante, in quanto il ferro si lega all'ossigeno. Il ferro è
coordinato dall'altra parte ad un istidina.
Perché il ferro non si ossida? Il Fe(III) non ha la possibilità di
legarsi con una coordinazione tetraedrica.

\autohalfpicture*{Coordinazione del ferro da parte di un istidina}

Per ossidare il ferro è necessario essere in ambiente acquoso. L'eme è
all'interno di una tasca idrofobica. La zona è protetta da una valina e
da una fenilalanina.
L'ossigeno è piccolo e passa attraverso queste zone. Questo garantisce
che il ferro non venga ossidato.

\marginbox*{A lungo andare anche l'emoglobina si può ossidare, in quanto viene
ossidato il ferro. Per questo la carne imbrunisce se esposta all'aria.}

\automarginpicture*{Tasca idrofobica che protegge il Fe(II) dall'ossidazione.}

Il ferro è circondato da due istidine, una distale e una prossimale.
Quella distale è legata all'ossigeno, mentre quella prossimale
direttamente al ferro. Sono parte di due eliche diverse (subunità
diverse). Le eliche sono E e F.
La preferenza delle strutture delle emoglobine è per l'\alpha-elica.

\autohalfpicture*{Istidina prossimale e istidina assiale.}

Anche altre molecole possono legate al ferro sono il monossido di
carbonio, che forma un legame molto più forte. Questo causa l'elevata
tossicità del \ce{CO}.
La mioglobina ha solo un sito di legame.

Come si comporta la mioglobina in base alla pressione parziale di
ossigeno?
Dipende dalle proprietà della proteina e dell'intorno chimico, quindi si
guarda la pressione parziale di ossigeno.
Si fa uno studio di funzione in base all'ossigeno legato alla proteina
e alla pressione di ossigeno.

Si guarda lo spettrofotometro, perché lo spettro della forma senza
ossigeno è diverso dalla forma con l'ossigeno. Si sa quindi la
concentrazione di ossigeno legato, in base a quanta emoglobina presente.
Da qui si ricava quanto ossigeno è disciolto nel sangue.

\autohalfpicture*{Spettro UV-Vis per la forma ossigenata e deossigenata della mioglobina.}

La costante di dissociazione è data dal rapporto delle concentrazioni di ossigeno
legato diviso i reagenti, ovvero mioglobina libera e pressione parziale di ossigeno.
\[
K = \frac{\ce{[MbO2]}}{\ce{[Mb][O2]}}
\]

Si va a vedere la frazione di siti occupati, contro i siti totali, al
variare della concentrazione di ossigeno.{} \theta{} è un frazione molare,
tra le due forme.
\[
\theta = \frac{\text{siti occupati}}{\text{siti totali}}
\]

Si opera una sostituzione con la prima formula.
\[
\theta = \frac{\ce{[MbO2]}}{\ce{[Mb]} + \ce{[MbO2]}} = \frac{K \ce{[Mb][O2]}}{\ce{[Mb]} + K\ce{[Mb][O2]}}
\]

Si trova che la frazione di siti occupati dall'ossigeno è data dalla
costante di associazione (in seguito si trasforma in costante di
dissociazione).
\[
\theta = \frac{K \ce{[O2]}}{1 + K\ce{[O2]}} = \frac{\ce{[O2]}}{\frac{1}{K} + \ce{[O2]}}
\]

Si definisce $\ce{[O2]_{1/2}}$ ovvero la concentrazione dove i siti sono metà riempiti.
Si ricava la relazione per trovare la dipendenza della concentrazione a
metà rispetto alla costante di associazione.
\[
\theta = 0.5 = \frac{K \ce{[O2]_{1/2}}}{1 + K\ce{[O2]_{1/2}}} = \frac{\ce{[O2]_{1/2}}}{\frac{1}{K} + \ce{[O2]_{1/2}}}
\]
Quindi la costante di associazione, per una concentrazione di ossigeno tale che $\theta{} = 0.5$ 
\[
K = \frac{1}{\ce{[O2]_{1/2}}}
\]
Si può quindi sostituire $K$ sulla funzione \theta.
\[
\theta = \frac{\ce{[O2]}}{\ce{[O2]_{1/2}} + \ce{[O2]}}
\]

Si può sostituire le pressioni parziali di ossigeno al posto della
concentrazione di ossigeno.
\[
\theta = \frac{p\ce{O2}}{P_{50} + p\ce{[O2]}}
\]

Si può plottare questa funzione e si riconosce una curva iperbolica, come nell'immagine \ref{img:CurvaMio}

\autohalfpicture{\theta{} in funzione della pressione parziale}{img:CurvaMio}

Se si guarda la pressione parziale di ossigeno, nei polmoni è molto elevata (circa 100 mmHg). La zona violetta è la pressione parziale nei polmoni. La frazione è 1, tutta la mioglobina è legata all'ossigeno.
La zona resting muscle ha 40 mmHg. La mioglobina tende a non rilasciare
ossigeno.
La mioglobina rilascia ossigeno ad una concentrazione inferiore di 20
mmHg, ovvero ottenuta con il muscolo a lavoro.
La p50 è pari a 4 mmHg.

Più bassa è la pressione parziale, e più la mioglobina si scarica di
ossigeno. La mioglobina quindi rilascia ossigeno in condizioni di tanta
richiesta.

In condizioni di riposo, l'emoglobina tende a alimentare i tessuti,
mentre la mioglobina viene utilizzata in condizioni di stress.
L'emoglobina è una proteina più raffinata, ha un meccanismo diverso.

La mioglobina quindi entra in azione quando c'è bisogno di ossigeno, ovvero quando l'emoglobina non è più efficiente.
La popolazione di mioglobina lega tutto l'ossigeno presente (nei polmoni) e riesce a
scaricarsi quando l'ossigeno manca.

Serve quindi una proteina che consenta di rilasciare ossigeno al metabolismo basale. Non deve essere troppo semplice, perché si carica
poco nei polmoni, per poter rilasciare nel punto giusto. Serve una
proteina che si comporti come una sigmoide, che quindi leghi bene
\ce{O2}, però che lo rilasci facilmente.

\autohalfpicture*{Andamento iperbolico, della mioglobina, e sigmoide, dell'emoglobina.}

Un andamento a sigmoide non è previsto da un semplice binding. Serve un
cambio di affinità per l'ossigeno dell'eme, al variare della pressione
parziale di \ce{O2}.

La grandezza della proteina cambia la sua affinità con l'ossigeno, a livello chimico/molecolare.

Serve una transizione di stato, ovvero una variazione conformazionale della proteina.

Nel passaggio della struttura dalla mioglobina, a quello della
emoglobina si ha un cambiamento di comportamento. Legando quattro
subunità, si acquisisce una proprietà aggiuntiva. Le subunità lavorano
in modo sinergico.

\automarginpicture*{Emoglobina e mioglobina}

Nella struttura dell'emoglobina, le istidine che legano il ferro sono conservate. La catena della
emoglobina non si allinea esattamente. L'emoglobina quindi presenta una
\alpha-elica in meno.

\autofullpicture*{Differenze degli amminoacidi nella mioglobina e nell'emoglobina. I tratti blu rappresentano i redidui identici nelle catene \alpha{} e \beta{} della mioglobina. I tratti rosa sono i residui identici nelle catene dell'emoglobina e della mioglobina. I tratti viola invece sono i residui identici nell'emoglobina e nella mioglobina di tutti i vertebrati. }

Le interazioni più forti sono quelle tra \alpha{}1 e \beta{}1 e \alpha{}2 e \beta{}2, questo
comporta che l'emoglobina è un dimero di eterodimeri. Queste (le
interazioni tra le unità) formano due eterodimeri

Il nome \alpha{} e \beta{} non indica la struttura secondaria, ma indica il
nome delle subunità. Esiste anche l'emoglobina \gamma, che si vedrà più
avanti.

I siti dell'emoglobina, sono cooperativi, quindi non sono possibili
legami intermedi, si legano o zero o quattro molecole di ossigeno. È un
caso di estrema cooperatività

La reazione è

\begin{center}
\ce{E + n L <=> EL_n}
\end{center}

Si definisce la costante di equilibrio

\[
K = \frac{[EL_n]}{[E][L]^n}
\]

Bisogna capire sperimentalmente la curva a sigmoide, in base ai siti
occupati

\[
\theta = \frac{[\text{siti occupati}]}{[\text{siti totali}]} = \frac{[L]^n}{[E] + K[E][L]^n}
\]

La frazione dei siti occupati può essere riarrangiata come

\[
\theta =\frac{K[L]^n}{1 + K[L]^n} =\frac{[L^n]}{\frac{1}{K} + [L]^n}
\]

Si calcola la concentrazione di legante \([L]_{\nicefrac{1}{2}}\), per
metà dei siti occupati, ovvero con \(\theta = 0.5\). Quindi la \(K\),
per \([L]_{\nicefrac{1}{2}}\) vale

\[
K = \frac{1}{[L]^n_{1/2}}
\]

Si può quindi ridefinire \theta{} in base alla \([L]_{\nicefrac{1}{2}}\).
\[
\theta = \frac{[L]^n}{[L]^n_{\nicefrac{1}{2}} + [L]^n}
\]

Questa funzione può essere riscritta come
\[
\frac{\theta}{1 - \theta} = \frac{[L]^n}{[L]^n_{\nicefrac{1}{2}}}
\]

Sostituendo il legante generico con l'ossigeno, la formula risulta
\[
\frac{\theta}{1 - \theta} = \frac{[\ce{O2}]^n}{[\ce{O2}]^n_{\nicefrac{1}{2}}}
\]

Si passa ai logaritmi, si linearizza la funzione.

\[
\log \frac{\theta}{1-\theta} = n\log [\ce{O2}] - n\log[\ce{O2}]_{\nicefrac{1}{2}}
\]

La curva sigmoide è data da \theta{} in funzione dell'ossigeno. In questo
caso, n=4, ovvero i leganti legati sono quattro, coordinati
simultaneamente.

\autofullpicture{Curva della mioglobina e dell'emoglobina}{fig:MioEEmo}

Questi passaggi possono essere fatti anche per la mioglobina. Se il
legame è totalmente cooperativo, si avrebbe una pendenza quattro volte
più alta rispetto alla mioglobina.

In figura{} \ref{fig:MioEEmo} è rappresentata la curva della mioglobina e dell'emoglobina.
Sperimentalmente, si ottiene una curva sigmoide. Si vede che si ha
un cambio di pendenza. Il cambio di pendenza avviene in una zona
particolare; la pendenza è il coefficiente di Hill.
La pendenza è pari a 3.6, quando l'ossigeno legato è pari a quattro.

Il comportamento della emoglobina è diverso a seconda della
\(p\ce{O2}\). Nel punto dove si ha un aumento della pendenza, la
proteina si comporta in modo cooperativo, anche se non lo fa sempre
(basse pressioni).

L'emoglobina si comporta come la mioglobina per zone a bassa concentrazione ed a
alta concentrazione

Il coefficiente di Hill viene valutato per molte proteine, noi lo
vediamo solo per l'ossigeno. Però può essere visto per enzimi
catalitici, a seconda della quantità di substrato presente. Il
comportamento del coefficiente di Hill, dà un'idea della cooperativa
dell'enzima. È una cosa molto generale.
L'emoglobina non arriva mai ad una pendenza uguale a quattro.

\automarginpicture*{Conformazioni dell'emoglobina}

\clearpage

Come fa l'emoglobina a diventare cooperativa? Si cambia la $K$ di equilibrio.
È stato possibile capire il comportamento in quanto si è studiata
l'emoglobina con e senza l'ossigeno.
Si sono visti quindi due comportamenti (strutture) limite.

\autoherepicture{0.8}

Sono presenti due conformazioni T (tight) e R (relaxed). Dalla struttura di
van der Waals si vedono delle variazioni strutturali.
Ci sono vari tipi di interazioni incrociate. Le interazioni importanti per il cambio di struttura sono quelle incrociate.

\autofullpicture*{Subunità dell'emoglobina}

Se la proteina cambia stato, si ha un movimento relativo \alpha{}1\beta{}2 e \alpha{}2\beta{}1.
Si ha quindi una variazione di 15° di un dimero rispetto ad un altro.

\autofullpicture*{Cambio conformazionale nelle subunità della mioglobina}

In seguito al legame con \ce{O2}, si vede una variazione
conformazionale. Questo si vede per ogni legante possibile (\ce{CO}, \ce{CN},
etc). I contatti che si vedono nell'immagine {}\ref{fig:Emo15}, dopo il
movimento cambiano la posizione relativa.

\autofullpicture{Cambio dei punti di contatto in seguito al cambio conformazionale}{fig:Emo15}

L'ossigeno è in grado di indurre questa variazione conformazionale.
L'ossigeno si lega all'atomo di ferro. Quando non c'è \ce{O2},
l'istidina prossimale coordina il ferro. Il ferro è pentacoordinato,
fino a che non lega \ce{O2} e questo causa uno spostamento, che sposta
il ferro più vicino all'istidina prossimale e fuori dal piano
dell'anello porfirinico.
Il ferro tende a tornare sul piano, quando lega \ce{O2}, in quanto il
ferro diventa esacoordinato.

\automarginpicture*{In assenza di \ce{O2}, l'anello porfirinico è deformato per spostamento del ferro coordinato all'istidina prossimale.}

\autofullpicture*{Effetto del legame dell'ossigeno, che abbassa il ferro sul piano.}

Quando l'ossigeno attrae il ferro dall'altra parte, l'istidina viene
trascinata dal ferro (è mobile). L'istidina prossimale (F8) trascina
l'elica e quindi l'ossigeno è responsabile del cambiamento
conformazionale. L'ossigeno triggera il cambio di affinità.

Per effetto cooperativo, il primo \ce{O2} lega e fa cambiare l'affinità
degli altri siti, quindi si ha un legame cooperativo
L'elica comunica lo spostamento al gomito, che quindi trasporta lo
spostamento di tutta catena (e anche di altre catene)

\autofullpicture*{Il cambio conformazionale  si trasfersice all'elica F.}

Questa variazionecomporta una variazione (riduzione) della cavità centrale (9-31). Le
variazioni quindi si trasmettono a tutta la struttura quaternaria. Per
effetto della simmetria, in quanto il dimero si sposta in modo solidale,
si crea un altro spostamento dalla parte opposta.

\autofullpicture*{Variazione conformazionale}

Anche legando un solo ossigeno, si verifica il passaggio di
configurazione (da relaxed a tight). L'affinità cambia a seconda
dell'ossigeno legato; cambia dal primo \ce{O2} legato.
Gli stati comunque sono stabili.

Se l'emoglobina è scarica, ed è presente \ce{O2}, questo si lega al
primo eme, trascina l'eme e fa cambiare la conformazione.
L'effetto è quello di predisporre l'eme come se fosse legato (Fe sul
piano porfirinico). Tutti i gruppi eme sono uguali, a livello di
preferenza e di conformazione.
Questo avviene in funzione della probabilità che l'emoglobina trovi
\ce{O2}. L'evento molecolare può avere più o meno probabilità.
Più l'emoglobina lega \ce{O2} e più affine diventa per l'ossigeno.

Questo effetto si chiama \emph{effetto allosterico}, che dice che, una volta legato
l'ossigeno, si ha un effetto altrove nella proteina (ovvero negli altri
eme). Anche molti enzimi sono allosterici, come quando un enzima ha un
substrato e la sua attività è regolata dal substrato presente.
Il massimo di probabilità della variazione della conformazione si ha
quando si legano una o più molecole di \ce{O2}.

\paragraph{Effetto Bohr}

Questa cosa succede anche con altri fattori allosterici, quindi si ha un
controllo anche in base ad altri fattori.
La \ce{CO2} viene prodotta dalla glicolisi e poi viene eliminata. La
\ce{CO2} viene trasportata sia dall'emoglobina, sia da bicarbonato.
La \ce{CO2} è catalizzata dall'enzima anidrasi carbonica, per essere
solubilizzata in acqua e poi viene catalizzata la gassificazione.
La \ce{CO2} è acida, e quindi è anche un buon fattore di controllo sul
rilascio di ossigeno (tramite il pH).

La conformazione (tight -- relaxed) può cambiare anche in base al pH.
effetto Bohr.
Quando c'è produzione di \ce{CO2}, si tende ad abbassare il pH;
l'acidificazione induce l'effetto allosterico che favorisce il rilascio
di \ce{O2} (forma tight).

\autofullpicture{Modifica dell'affinità dell'emoglobina per l'ossigeno, in seguito all'abbassamento del pH.}{fig:Bohr}

L'effetto dell'acidificazione, abbassa il coefficiente di Hill, che
causa la curva sigmoide {}\ref{fig:Bohr}. Quindi per la stessa pressione, si hanno
meno siti occupati; si libera \ce{O2}.
Questo è un modo per fare un fine-tuning della proteina.
Quindi più si acidifica, più si sta consumando \ce{O2}, e più ossigeno viene
liberato.

Oltre all'acidificazione, anche \ce{CO2} può legarsi all'emoglobina.
Questo induce la conformazione T, che consente un ulteriore rilascio di
\ce{O2}.
Quindi se si produce tanta \ce{CO2}, si ha un rilascio maggiore di
\ce{O2}.

Questi effetti sono fatti da effettori diversi dall'ossigeno, che è
definito come omo-effettore. Tutti questi effetti sono dati dalla
struttura quaternaria dell'emoglobina.
Questi meccanismi non avvengono per la mioglobina.

\vfill
\pagebreak

\automarginpicture*{BPG}
C'è un altro effettore, ovvero il 2,3 bisfosfoglicerato, che è importante per il rilascio di
\ce{O2}. È un composto a due fosfati (piccola molecola), che ha un
importante effetto biologico.
Questo abbassa l'affinità dell'emoglobina per l'ossigeno, che consente
di far scaricare l'ossigeno in misura maggiore.
Il BPG viene prodotta per lo scopo di regolare l'affinità
dell'emoglobina.

\autohalfpicture*{Effetto degli effettori verso l'emoglobina.}

Il BPG si lega nella cavità della emoglobina. Nella forma R non può
entrare, mentre nella forma T si. Questo consente di spostare
l'equilibrio verso la forma T.
Solo \ce{O2} promuove verso la forma R, mentre gli effettori (tutti)
verso la T.

\autoherepicture{0.5}

La cavità consente di legarlo in quanto la cavità
è ricca di cariche positive. Gli amminoacidi creano una zona carica,
l'interazione avviene facilmente.
L'unica cosa che lo ferma è la conformazione.

Il BPG è importante nella fase di gestazione. La placenta è una zona di
scambio. Il feto viene nutrito dalla madre; l'emoglobina del feto deve
essere molto più affine per l'\ce{O2} rispetto a quello della madre.

\autofullpicture*{Effetto del BPG nella fase della gestazione.}

L'emoglobina fetale ha una subunità \gamma{} al posto di una beta. La
madre usa il BPG che permette all'emoglobina di scaricarsi nella
placenta, quindi l'emoglobina fetale (che non può legare BPG) diventa
più affine.
La subunità gamma serve a non poter legare il BPG

\autofullpicture*{Differenze tra il legame del BPG con le subunità \beta{} e \gamma.}

Il feto riesce a legare più ossigeno rispetto alla madre. La subunità
\gamma{} non ha la zona positiva, quindi non può legare.


L'emoglobina è un esempio di proteina allosterica. La regolazione allosterica viene mediata attraverso un legame con un sito specifico di un effettore, che regola l'attività dell'enzima.

Si possono avere due effetti:
\begin{itemize}
   \item \emph{Effetto omotropico:} il legando è uguale dall'effettore.
   \item \emph{Effetto eterotropico:} il legando è diverso dall'effettore.
\end{itemize}

Ci sono due modelli teorici che trattano questo aspetto, ovvero il modello sequenziale e il modello concertato.

\paragraph{Modello sequenziale}

In questo modello, le subunità cambiano conformazione, una alla volta. Il legame di un legando ad una subunità cambia la conformazione della sola subunità a cui si lega e solo le subunità vicine ne risentono.

\autoherepicture{0.8}

\paragraph{Modello concertato}

In questo modello, tutte le subunità cambiano conformazione insieme (o T o R, ma mai TR). I legandi hanno bassa affinità per la forma T, ma elevata per R. Il legame aumenta le probabilità che tutte le subunità siano nella forma R. L'equilibrio si sposta verso la forma R sempre di più, man mano che più siti sono occupati.
Nell'emoglobina, lo stato predominante è R, con circa due molecole di ossigeno.
Per avere la curva perfetta, si dovrebbe tenere conto di tutte le costanti del modello concertato.

\automarginpicture*{Reazioni nel modello concertato}

Le mutazioni possono alterare la struttura e la funzione dell'emoglobina. Alcune mutazioni che alternano la struttura della proteina possono essere la causa di alcune patologie. Una di queste, per essere precisi, è la sostituzione di una glicina con una valina nella posizione A3(6)\beta{}, passando quindi da un amminoacido polare a uno apolare.
Questa mutazione viene detta \emph{anemia falciforme} e l'emoglobina viene indicata come Hbs.

Infatti, questa sostituzione amminoacidica comporta che la forma deossi dell'emoglobina tenda ad aggregarsi (polimerizzare), formando forme rigide che si estendono per la lunghezza dell'eritrocita, dandole una forma a falce.
Questi filamenti sono insolubili e possono ostruire i capillari, causando la morte cellulare.

\automarginpicture*{Aggregazione dell'emoglobina}

Gli individui portatori di due coppie del gene dell'emoglobina S sono affetti da questa malattia, che è grave e se non è tenuta sotto controllo, porta alla morte. Al contrario, i portatori sani sono colpiti da una forma blanda di anemia e, anzi, mostrano una resistenza maggiore alla malaria. Da qui si vede una correlazione, infatti nelle zone dove è presente la malaria, prevale il gene dell'anemia falciforme.

Altri esempi possono essere l'emoglobina M;{} in questa mutazione, l'istidina prossimale è sostituita da una tirosina, che stabilizza la forma \ce{Fe (III)}. L'eme non può più legare \ce{O2}.

Le talassemie sono errori genetici con alterazioni delle subunità \alpha{} o \beta. Le talassemie caratterizzano una ridotta o assente sintesi dell'emoglobina; questo può evitare alle malattie che necessitano di questa proteina per diffondersi.



\chapter{Contrazione muscolare}

\ChangePicturesFolder{8}

La miosina è una proteina motoria che subisce dei cambiamenti conformazionali nel momento in cui idrolizza l'ATP.
Il modello a scorrimento dei filamento della contrazione muscolare descrive il movimento dei filamenti spessi rispetto ai filamenti sottili.

La proteina globulare actina è in grado di formare delle strutture come i microfilamenti filamenti sottili del muscolo.
Ora verranno esaminate le basi strutturali e chimiche del movimento nel muscolo striato, uno dei sistemi di mobilità meglio conosciuti e compresi.

Il muscolo striato è costituito da filamenti proteici spessi e sottili che interagiscono tra loro.
Le fibre muscolari hanno un diametro di circa 20--100 \mu m, sono spesso sono lunghe quanto il muscolo stesso e contengono circa 1000 miofibrille
L'unità ripetitiva della miofibrilla è il sarcomero. Ogni miofibrilla è rivestita da un reticolo sarcoplasmatico che è quello che dà lo stimolo di aumentare la concentrazione di calcio, che fa contrarre il sarcomero.

\marginbox{Sarcomero}{Il sarcomero è l'unità ripetitiva, che si accorcia durante la contrazione.}

\autofullpicture*{Sarcomero}

I filamenti spessi e sottili scivolano l'uno sull'altro durante la contrazione. I filamenti spessi contengono principalmente miosina, la cui struttura quaternaria consiste in due catene avvolte, terminando con due teste.
La testa della miosina è un ATPasi, ovvero un enzima che idrolizza l'ATP.{}

I filamenti sottili sono costituiti principalmente da polimeri di actina. Nella sua forma monomerica, l'actina è anche detta actina G ed è a forma di sfera, in quanto è una proteina globulare.
In forma polimerizzata è invece detta actina F.

Ogni subunità di actina ha un sito di legame per uno ione \ce{Ca^{2+}} e diversi siti di legame per l'ATP.{}
Il filamento è avvolto da filamenti sottili di tropomiosina e da troponina, che interagiscono con le teste e sono sensibili al calcio

\autofullpicture*{Filamenti di actina e miosina}

\paragraph{Miosina}

È stato possibile separare in vitro le parti della miosina, sono state cristallizzate ed è stata determinata la struttura di questa proteina.

La miosina è formata da una coda lunga 150 nm formata da due \alpha-eliche superavvolte, che poi terminano nella testa, dove assumono una forma globulare.
Queste sono le due catene pesanti, tuttavia sono presenti anche altre quattro catene leggere.

Le catene leggere sono divise in due catene regolatorie (RLC) e due catene essenziali (ELC).
Sono posizionate tra la coda e la testa di ciascuna catena leggera (una per tipo).
Studiando la testa (globulare), si è notato un sito dell'attività ATPasica e un sito che lega l'actina

\autofullpicture*{Struttura del filamento.}

\paragraph{Actina F}

L'actina F è costituente dei filamenti sottili
Il monomero actina G (42 kDa) polimerizza e diventa actina F, i filamenti sono direzionati.
Affinché avvenga la polimerizzazione, l'actina G lega dell'ATP, che idrolizza. Nella forma polimerizzata, si ha quindi ADP legato. In realtà l'actina G, può polimerizzare sia con ADP che con ATP.{}

L'actina F può polimerizzare, per rimozione di ADP.{}
La miosina si attacca sull'actina con la testa, che funge da sito di riconoscimento

\autofullpicture*{Struttura della troponina}

Come anticipato, nei filamenti sottili ci sono altri due componenti, ovvero la tropomiosina e la troponina. La tropomiosina è un \alpha-elica a doppio filamento, un monomero di tropomiosina lega sette monomeri di actina. La troponina invece è un complesso a tre subunità, con una diversa specificità.
La TnC lega lo ione \ce{Ca^{2+}}, la TnT lega la tropomiosina, mentre la TnI lega l'actina.

Lo stimolo per contrarre il muscolo fa aumentare la concentrazione di \ce{Ca2+}. La troponina lega lo ione calcio e cambia conformazione. Questo cambio conformazionale si trasmette alla tropomiosina, che si sposta e permette alla testa S1 della miosina di attaccarsi sull'actina

\autofullpicture*{Il cambio conformazionale si trasmette alla tropomiosina, che permette alla testa S1 della miosina di attaccarsi all'actina.}

\clearpage

I filamenti di actina scorrono sui filamenti di miosina. In condizioni di riposo, il sito di legame per la miosina è coperto e, quando lo ione calcio aumenta, si scopre e la miosina è libera di legare l'actina. A questo punto, l'ATP si può legare alla testa della miosina.

La miosina che ha legato ATP rilascia l'actina. Nel sito attivato della miosina, viene idrolizzato l'ATP appena legato. A seguito dell'idrolisi avviene un cambio conformazionale della testa della miosina (ADP e P vengono associati alla testa)

\autofullpicture*{Si ha un cambio di conformazione della miosina, a seguito dell'idrolisi di ATP.{} In seguito, la miosina si riattacca al filamento di actina e rilascia P. Infine, avviene un ``colpo di forza'' dovuto alla miosina che torna alla conformazione iniziale e trascina il filamento di actina per 100 \AA.}

La miosina inizia ad attaccarsi debolmente a un’actina più in là. L'uscita del gruppo fosfato fa in modo che si leghino con forza e avviene il ``colpo di forza'', dove la miosina torna alla conformazione iniziale, trascinando il filamento di actina di 100 \AA. Infine esce l'ADP e il ciclo può ripetersi.

\automarginpicture*{Trasmissione al braccio}

In sintesi, l'ATP produce una variazione conformazionale della miosina che viene trasmessa al braccio, causando il movimento

\autoherepicture{0.7}

\clearpage

\paragraph{Giunzione neuromuscolare}
Il nervo ha delle terminazioni, ovvero i bottoni sinaptici, che si attaccano al sarcolemma (membrana che avvolge le fibre muscolari). L'impulso nervoso si traduce con l'aumento della concentrazione di un neurotrasmettitore (acetilcolina), che funge da primo messaggero.

Quindi delle vescicole sinaptiche si aprono e fanno uscire l'acetilcolina, che, in qualche modo, fà aprire i canali dello ione calcio (secondo messaggero).

I segnali nervosi sono fatti in modo da durare istantaneamente, infatti quando esce l'acetilcolina, questa viene prontamente idrolizzata. Quando il segnale si interrompe, la concentrazione di \ce{Ca^{2+}} torna rapidamente a quella iniziale. Ci sono infatti delle pompe, che pompano lo ione calcio sotto gradiente, usando ATP.{}

Quando avviene la morte, si smette di produrre ATP ed il reticolo sarcoplasmatico inizia a deteriorarsi, facendo fuoriuscire lo ione calcio. Questo fa contrarre i muscoli e non essendoci ATP, la pompa non è in grado di pompare fuori il calcio.

È per questo che i muscoli rimangono contratti, fino a che i miofilamenti non si decompongono o si rompono.

Un meccanismo simile avviene anche nella cellula procariota, nei flagelli. Inoltre dentro la cellula, il citoscheletro è composto da actina, che fa da ``strada'' sulla quale possano muoversi gli organelli della cellula. La strada è quindi formata da microtubuli rivestiti da filamenti di actina.

\autofullpicture*{Contrazione muscolare }

\begingroup \autoherepicture{0.78}\captionof{figure}{Giunzione neuromuscolare} \endgroup
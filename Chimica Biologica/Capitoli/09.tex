\chapter{Immunoglobuline}

\ChangePicturesFolder{9}

Tutti gli organismi sono soggetti a continui attacchi da parte di altri organismi, compresi i microrganismi che provocano malattie e virus. Negli animali superiori, i patogeni che riescono ad oltrepassare la barriera fisica formata dalla pelle e dalle membrane mucose possono essere identificati ed eliminati dal sistema immunitario.

Gli anticorpi fanno parte di un arsenale, che serve per scatenare la
risposta immunitaria. Noi andremo a guardare la parte di riconoscimento.
La risposta è complessa: vengono liberate delle immunoglobuline (o
anticorpo) come risposta umorale, mentre come risposta cellulare,
vengono liberati delle proteine.
La risposta umorale mette nel sangue delle proteine.

Ci sono delle cellule specializzate linfociti B, che liberano le
immunoglobuline nel sangue. I linfociti T eliminano direttamente le
cellule estranee (o tumorali). Sono abbastanza specifiche.
Ci sono altre cellule T, che servono ad aiutare queste due risposte.

Il sistema immunitario dunque, distingue ciò che è estraneo, tramite l'utilizzo di antigeni. La risposta immunitaria è specifica, versatile e dotata di memoria. Specifica perché ogni anticorpo è in grado di legare un solo antigene. Versatile perché sono presenti numerosissimi antigeni. Inoltre è dotata di memoria, in quanto le cellule B che hanno già incontrato un antigene, producono una risposta immunitaria più rapida e intensa. Si distinguono due tipi di immunità:
\begin{itemize}
\item \emph{Risposta immunitaria cellulare:} mediata dalle cellule dette linfociti T.
\item \emph{Risposta immunitaria umorale:} è mediata da un gruppo numeroso di proteine estremamente diverse, dette anticorpi o immunoglobuline, e sono prodotte dai linfociti B
\end{itemize}

La risposta immunitaria è indotta dalla presenza di una macromolecola estranea, che può essere una proteina, un carboidrato o un acido nucleico, detta antigene o immunogene. L'antigene è capace di scatenare la risposta immunitaria. Il determinante antigenico viene chiamato epitopo, ed è il sito dell'antigene di affinità per l'anticorpo.

Tutte le immunoglobuline contengono almeno quattro subunità, che sono composte da due catene leggere L identiche (23 KDa circa) e da due catene pesanti H identiche (53--75 KDa). Queste subunità si associano tramite ponti disolfuro e interazioni non covalenti; le subunità consentono una struttura simile ad una Y.

Le cinque classi di immunoglobuline (Ig) differiscono in base al tipo di catena pesante che contengono o, in alcuni casi, alla struttura delle loro subunità. Le IgG, o \gamma-globuline, sono le più abbondanti, sono monomeri e sono presenti nel sangue e nel fluido extravascolare. Le IgA sono presenti sottoforma di monomeri, dimeri o trimeri, sono presenti nelle secrezioni e agiscono sulle superfici cellulari. Le IdD sono monomeri presenti in grandi quantità sulla membrana di molti linfociti B circolanti. Le IgE sono monomeri presenti in piccole quantità nel sangue e sono coinvolte nelle risposte allergiche. Le IgM sono le più grosse, sono formate da cinque molecole a forma di Y, disposte intorno ad una subunità J centrale e sono le prime ad essere secrete.

\section{IgG}

\autofullpicture*{Rappresentazione di un ummonoglobulina G}

IgG è costituita da quattro catene, due leggere e due pesanti. Ogni catena contiene dei domini compatti e resistenti. In particolare, ciascuna catena leggera contiene una regione variabile, detta V\ped{L}, e una regione costante, detta C\ped{L}, mentre le catene pesanti contengono una regione variabile V\ped{H} e tre regioni costanti, CH\ap{1}, CH\ap{2} e CH\ap{3}. Ciascun dominio contiene dei ponti solfuro intracatena e le quattro catene sono legate da essi.

Le anse ipervariabili determinano la specificità antigenica. L'enzima proteolitico papaina scinde IgG in corrispondenza della regione cerniera, delimitata dalla linea tratteggiata, producendo due frammenti, ovvero F\ped{ab} e F\ped{c}. I frammenti F\ped{ab} rappresentano i bracci della molecola, contengono un'intera catena L e la metà N-terminale di una catena H.
Questi frammenti contengono i siti di legame dell'antigene; infatti la denominazione ``ab'' indica che è presente l'antigen binding.

\autofullpicture*{Domini costanti e domini variabili}

Il frammento F\ped{c} è formato dalla metà C-terminale dalle due catene H. F\ped{c} e F\ped{ab} sono collegati da una regione detta \emph{cerniera} flessibile, quindi la forma ad Y non è sempre simmetrica, ma si muove. Nel frammento F\ped{c}, i domini costanti delle catene pesanti alla base segnalano ai macrofagi di attaccare, indirizzando le diverse immunoglobuline nei tessuti o per la secrezione.
C'è anche un carboidrato a metà della Y, che influisce sulla stabilità strutturale e nel riconoscimento. La specificità di un anticorpo è stabilita dalla sequenza di residui nei domini variabili delle catene leggere e pesanti.

\autofullpicture*{Struttura tridimensionale di un immonoglobulina G}

Le interazioni antigene-anticorpo sono forze deboli, come interazioni di van der Waals, interazioni ad idrogeno e interazioni ioniche. L'epitopo è la zona dell'antigene che viene riconosciuta dall'anticorpo.

\autofullpicture*{Per generare un adattament oottimale con l'antigene, i siti di legame subiscono piccole variazioni conformazionali.}

Negli anticorpi multimerici, come le IgM, le subunità sono legate da ponti disolfuro e da una catena di unione J. Quando un antigene presenta più determinanti antigenici, si forma un reticolo esteso formato da anticorpi e antigeni, che consente la precipitazione.

Le differenze risiedono nelle regioni delle catene leggere (V\ped{L}, N-terminali 1--108) e delle catene pesanti (V\ped{H} 1--125). Ci sono tre tratti ipervariabili.

Le unità di analogia delle immunoglobuline hanno tutte lo stesso caratteristico ripiegamento delle immunoglobuline, ovvero un sandwich composto da tre o quattro foglietti \beta{} antiparalleli, collegati tra loro da ponti disolfuro. Nei domini variabili sono presenti tre loop ipervariabili e sono denominati CDR1, CDR2 e CDR3. La sigla sta ad indicare la funzione di questi domini, infatti CDR indica \emph{Complementary Determining Regions}.

Il frammento F\ped{C} può essere riconosciuto dai recettori di questo frammento, ad esempio quando si lega ai recettori di un macrofago.

\autofullpicture*{Fagocitosi di un virus, legato ad anticorpi IgG da parte di un macrofago}

\subsection{Immunoblotting}

L'immunoblotting è una tecnica che permette di identificare una determinata proteina in una miscela di proteine mediante il riconoscimento da parte di anticorpi specifici. Una di queste tecniche è chiamata \emph{ELISA}, che è un acronimo per \emph{Enzyme-Linked Immunosorbent Array} ed è utilizzata per riconoscere alcune malattie, quali l'HIV, il morbo della mucca pazza, il COVID, oltre a droghe e steroidi. Inoltre viene usato anche nei test di gravidanza.

\marginbox*{Nei test di gravidanza si va a ricercare la presenza di hCG-Gonadotropina Corionica Umana, che è un ormone secreto dal trofoblasto.}

Il funzionamento è il seguente:
\begin{itemize}
\item La superficie è rivestita di campioni antigeni.
\item I siti grigi alla base del test sono dei siti che sono occupati da una proteina non specifica.
\item L'anticorpo primario si lega all'anticorpo secondario, che lega l'antigene.
\item Il complesso anticorpo-enzima si lega all'anticorpo primario.
\item Si aggiunge il substrato, che consente agli anticorpi di viaggiare.
\item La formazione di un prodotto colorato indica la presenza dell'antigene specifico.
\end{itemize}

\autohalfpicture*{Rappresentazione del funzionamento dei test di gravidanza o per il COVID-19.}

Il SARS-CoV-2 ha una sequenza molto simile al coronavirus che ha originato la SARS nel 2003 e penetra all'interno delle cellule epiteliali del tessuto respiratorio, grazie allo stesso recettore, ovvero Ace2.

Sulla superficie del virus, si trova una proteina chiamata \emph{spike}, che è una sorta di chiave che si adatta a una delle serrature presenti sulla superficie delle cellule del sustema respiratorio unamo, ovvero il recettore Ace2

\automarginpicture*{Proteina Spike del SARS-COVID-19.}

Una volta legato il recettore Ace2, il virus è in grado di penetrare all'interno della cellula e si forma una fagocitosi nella membrana che assorbe la particella virale.
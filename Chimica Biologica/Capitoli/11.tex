\part{Lipidi}

\chapterpicture{header_04}

\chapter{Membrane}

\ChangePicturesFolder{11}

Le membrane sono strutture organizzate di lipidi e proteine che delimitano le cellule e i comparti della cellula, tipo gli organelli. Le membrane servono a isolare l'interno della cellula dall'ambiente.

Le membrane organizzano i processi biologici, compartimentalizzandoli.

Le membrane possiedono una certa permeabilità, che viene chiamata \emph{permeabilità selettiva}. Le membrane controllano il flusso delle informazioni, facendo passare gli stimoli chimici e/o elettrici.

Le membrane guidano processi di conversione di energia tramite gradienti ionici e voltaici, dai quali, ad esempio, si può ottenere ATP.{} Le membrane non sono polimeri, ma composti organizzati in una struttura supramolecolare di lipidi e proteine.

\section{Lipidi}

I lipidi si distinguono in lipidi di riserva, che vengono utilizzati con uno scopo energetico, e in lipidi di membrana, ovvero quelli che vengono utilizzati con uno scopo strutturale.

I lipidi sono sostanze insolubili in acqua e possono avere una testa idrofila e una coda idrofoba; per questo possono essere definite \emph{sostanze anfipatiche}. Le sostanze di questo tipo tendono a formare aggregati in acqua, come micelle o doppi strati.

I lipidi comprendono una grande varietà di composti, infatti si possono trovare:
\begin{itemize}
\item Acidi grassi
\item Grassi e olii
\item Cere
\item Fosfolipidi
\item Steroli
\item Vitamine solubili
\end{itemize}

Alcune delle funzioni e delle proprietà dei lipidi sono:
\begin{itemize}
\item Sorgente metabolica
\item Riserva di energia come trigliceridi nel tessuto adiposo
\item Isolamento del corpo dal freddo
\item Isolamento protettivo meccanico di organi vitali
\item Formazione di membrane
\end{itemize}

I lipidi possono essere scomposti in:
\begin{itemize}
\item Lipidi semplici, ovvero acidi grassi, grassi neutri (come i mono, di o trigliceridi) e cere
\item Lipidi complessi, ovvero fosfolipidi, glicolipidi e lipoproteine
\end{itemize}

\autofullpicture*{Esempi di acidi grassi}

\clearpage

\section{Acidi grassi}

Gli acidi grassi sono acidi carbossilici con lunghe catene idrofobiche. Possono essere scomposti in:
\begin{itemize}
\item Acidi grassi saturi
\item Acidi grassi insaturi
\end{itemize}

In natura gli acidi grassi hanno un numero pari di carbonio. 
Gli acidi grassi sono acidi deboli, e hanno una $pK_a$ circa pari a 4.5.

Alcuni acidi grassi sono detti \emph{essenziali}, in quanto è necessario assumerli con la dieta e non è possibile sintetizzareli. Essi sono:
\begin{itemize}
\item Acido linoleico
\item Acido linolenico
\item Acido arachidonico
\end{itemize}

I grassi saturi sono molto flessibili, ma comunque la conformazione a energia minore è quella distesa. Hanno punti di fusione più alti e questi aumentano con l'aumentare della lunghezza della catena.

I grassi insaturi saranno preferibilmente nella struttura cis. Hanno punti di fusione più bassi.

\paragraph{Cere}
Le cere sono esteri derivati da acidi grassi a catena lunga e un alcol a catena lunga. Le cere sono una riserva di energia, ma in realtà fanno da cuscinetti, o da rivestimento superficiale. Infatti costituiscono il sottile strato che copre le foglie delle piante o forniscono un rivestimento superficiale per insetti o la pelle degli animali. Non si trovano nelle membrane.

\paragraph{Trigliceridi}
La funzione biologica è quella di immagazzinare l'energia negli adipociti, e sono anche degli isolanti termici. Sono degli esteri del glicerolo e di tre acidi grassi.
Il glicerolo diminuisce il carattere idrofilico della testa polare, quindi i trigliceridi sono insolubili in acqua.

\autoherepicture{0.75}

\clearpage

I lipidi di membrana sono:
\begin{itemize}
\item \emph{Glicerofosfolipidi:} hanno le teste polari e una doppia coda idrofobica, quindi tendono a formare doppi strati
\item \emph{Sfingolipidi}
\item \emph{Glicoesfingolipídi}
\end{itemize}

I glicerofosfolipidi o fosfogliceridi sono i principali costituenti lipidici delle membrane biologiche. Sono formati da glicerolo-3-fosfato, dove le posizioni \ce{C1} e \ce{C2} sono esterificato con acidi grassi e il gruppo fosforico è legato ad un altro gruppo polare \ce{X}. Sono i lipidi più abbondanti nelle membrane

\automarginpicture*{Glicerofosfolipidi}

\paragraph{Fosfatidilcoline}

\automarginpicture*{Fosfatidilcoline}

Tutte queste varietà di lipidi garantiscono una membrana che può essere assemblata in tanti modi diversi. A volte il foglietto esterno e quello interno possono avere gruppi diversi per avere una direzionalità di proprietà, ad esempio, in una membrana si può avere una differenza di potenziale

\paragraph{Sfingolipidi}

Sono derivati della sfingosina, in figura, che ha una coda idrofobica lunga e un dominio polare con un gruppo amminico.
Il gruppi amminico può essere legato ad un acido grasso per dare una ceramide

\automarginpicture*{Sfingolipidi}

La sfingomielina è un ceramide con fosfocolina nella testa. È un costituente comune delle membrane plasmatiche. Sono detti anche sfingofosfolipidi.

\automarginpicture*{Sfingomielina}

\automarginpicture*{Cerebroside}

Un cerebroside è un ceramide con un monosaccaride nella testa polare. Sono quindi chiamati glicosfingolipidi. Sono i primi di gruppi fosforici, quindi non sono ionici.
Un ganglioside è una ceramide con un oligosaccaride nella testa polare.

Il fatto che i lipidi di membrana si trovino nel foglio esterno delle membrane plasmatiche con le catene di zucchero fuori dalla superficie permettono loro di comportarsi da recettori di determinati ormoni, ma anche da recettori per determinate tossine proteiche dei batteri.

\autofullpicture*{I lipidi di membrana sono distribuiti in modo asimmetrico. Per fare un esempio, le glicoproteine e i glicolipidi di membrana sono orientati con le loro porzioni oligosaccaridiche all'esterno della cellula.}

Le membrane non sono strutture rigide, bensì fluide e sono in continuo movimento. È stimato che un singolo lipide può coprire le lunghezza di un nanometro in un secondo, per questo il doppio strato lipidico viene considerato come un fluido bidimensionale.
L'interno del doppio strato lipidico è in movimento costante a causa delle rotazioni intorno ai legami \ce{C-C} delle code lipidiche. Mentre il trasferimento di un lipide attraverso un doppio strato, detto \emph{diffusione trasversale} o \emph{flip-flop}, è un evento estremamente raro.
Ci sono,però, degli enzimi, detti \emph{flippasi}, che favoriscono questo processo. Le catene idrocarburiche non stanno dritte, ma sono tutte attorcigliate tra loro.

\autofullpicture*{Azione delle \emph{flippasi}}

I lipidi di membrana sono distribuiti in modo asimmetrico. Questa asimmetria si stabilisce durante la biogenesi nel reticolo endoplasmatico\footnote{I lipidi di membrana vengono fabbricati nel reticolo endoplasmatico} e viene mantenuta grazie alla presenza di traslocatori di membrana. I traslocatori di membrana consumano ATP e lavorano contro concentrazione, le flippasi invece catalizzano il flip-flop da alte concentrazioni a basse concentrazioni.

L'asimmetria causa una differenza di carica netta che genera un campo elettrico. In una cellula morta, le traslocasi dei fosfolipidi non funzionano più e si ha una perdita dell'asimmetria. Quando è presente questa condizione, la fosfatidilserina è esposta sulla superficie e questo dà il segnale che una cellula è morta.

\autofullpicture*{Disposizione dei lipidi di membrana. La fosfatidilserina è l'unica con carica negativa, gli altri sono neutri}

A pH neutro, solo la fosfatidilserina ha una carica negativa, mentre gli altri fosfolipidi sono neutri. I lipidi agiscono come solvente bidimensionale per le proteine di membrana. Alcune proteine di membrana possono funzionare solo in presenza di specifiche teste di fosfolipidi.

Esistono enormi differenze in composizione tra i due foglietti del doppio strato. In particolare i glicolipidi, la sfingomielina e la maggior parte della fosfatidilcolina, con le loro teste cariche positivamente, sono localizzate sul foglietto endoplasmatico . I lipidi con teste polari caricate negativamente o neutre, come la fosfatidiletanolammina e la fosfatidilserina sono localizzate prevalentemente sul foglietto citoplasmatico. I glicolipidi sono assolutamente asimmetrici, in quanto la parte dello zucchero sta sempre all'esterno. Il colesterolo è distribuito uniformemente.

Gli animali utilizzano l'asimmetria dei fosfolipidi delle membrane plasmatiche per distinguere tra cellule vive e cellule morte. Infatti, quando, una cellula animale subisce la morte cellulare programmata, detta \emph{apoptosi}, la fosfatidilserina, che normalmente è confinata nel monostrato citosolico della membrana plasmatica, viene rapidamente traslocata al monostrato extracellulare.

La fosfatidilserina esposta sulla superficie cellulare serve come segnale per indurre le cellule vicine, ad esempio i macrofagi, a fagocitare la cellula morta e digerirla.

\marginbox{Modello a mosaico fluido}{
Gli elementi lipidici e proteici sono in costante movimento, che viene chiamato \emph{diffusione laterale veloce}.\\
Come avviene il flip-flop? Per vederlo si rende fluorescente la membrana, attaccando un gruppo fluorescente, e si eccita la membrana resa fluorescente con un laser. Poi si guarda il tempo di rilassamento.\\
Misurando l'intensità della fluorescenza nella zona colpita dal laser, si dovrebbe vedere un grafico del genere\\
Il recupero della fluorescenza può essere spiegato solo con il movimento di flip-flop che avviene nella membrana.}

\section{Colesterolo}

\automarginpicture*{Colesterolo}

Il colesterolo è un componente di membrana nelle cellule eucariote. Ha una coda idrofobica e una testa polare. Ha una notevole rigidità molecolare dovuta agli anelli, che non sono planari. Può essere biosintetizzati, ma anche assunto nella dieta. Oltre ad essere un componente di membrana. è utilizzato come precursore metabolico degli ormoni steroidei, ma anche nella sintesi della vitamina B. Gli steroidi sono derivati del ciclopentanoperidrofenantrene.

Il colesterolo biosintetizzano non crea problemi; questi arrivano quando se ne assume troppo con la dieta. Infatti un po' viene metabolizzato, ma se è troppo, si accumula e si deposita nelle pareti interno delle arterie, ostruendole.

Il colesterolo, in quanto rigido, consente di diminuire la mobilità delle code dei fosfolipidi. La presenza del colesterolo interferisce con il packing delle code degli acidi grassi nello stato cristallino e così inibisce la transizione allo stato cristallino stesso.
Le membrane con alta concentrazione di colesterolo hanno una fluidità intermedia tra il cristallo liquido e lo stato cristallino.

Quando un doppio strato lipidico viene raffreddato al di sotto di una temperatura caratteristica, detta \emph{temperatura di transizione}, va incontro ad una sorta di cambiamento di fase in cui passa da cristallo liquido\footnote{Questo stato viene detto cristallo liquido in quanto le molecole sono ordinate in alcune direzioni, ma non in altre} ad uno stato gelatinoso.
È fondamentale per il funzionamento della membrana che la temperatura sia superiore alla temperatura di transizione.

La temperatura di transizione aumenta con la lunghezza delle catene e con il grado di saturazione di queste. La maggior parte delle membrane ha una temperatura di transizione compresa tra 10 e 40 °C.
Il colesterolo amplia l'intervallo di temperatura di transizione, fino ad eliminarla se presente in alte concentrazioni.

Le membrane costituite da molti lipidi con acidi grassi a catena lunga, come le plasma-membrane, devono avere una concentrazione di colesterolo elevata per evitare di cristallizzare alle temperature fisiologiche.
La membrana mitocondriale interna, invece, non ha bisogno di una concentrazione elevata di colesteroli, in quanto è formata da fosfolipidi con acidi grassi aventi uno o più doppi legami

Riassumendo, i fattori che influenzano la fluidità della membrana sono:
\begin{itemize}
\item Fisici, come temperatura, pressione e potenziale di membrana
\item Chimici, come la presenza di teste polari nei fosfolipidi, la lunghezza degli acidi grassi, il grado di insaturazione e la presenza di colesterolo
\item Indiretti, come la presenza di ormoni, i cicli cellulari e l'adattamento allo stress
\end{itemize}

\begin{table}
\begin{tabular}{cccc}
Membrane & Proteine & Lipidi & Carboidrati \\
Plasmatica: Fegato di topo & 46 & 54 & 2--4\\
Plasmatica: Eritrocita umano & 49 & 43 & 8\\
Plasmatica: Ameba & 52 & 42 & 4\\
Nucleare: Fegato di ratto & 59 & 35 & 2\\
Mitocondiale esterna & 52 & 58 & 2--4\\
Mitocondriale interna & 76 & 24 & 1--2\\
Mielina & 18 & 79 & 3\\
\end{tabular}
\caption{Composizione delle membrane in diversi tipi di cellule}
\end{table}

La membrana mitocondriale interna ha molte proteine, che sono necessarie in quanto sono coinvolte nella fosforilazione ossidativa. Inoltre, in questo ambiente avvengono molte reazioni.

\autofullpicture*{Effetto del colesterolo nelle membrane}

\clearpage

\section{Proteine di membrana}

\autofullpicture*{Proteine di membrana}

Le \emph{proteine integrali di membrana} contengono una struttura transmembrana formata da \alpha-eliche e da \beta-barrel, la cui superficie è idrofoba.
Le \emph{proteine legate ai lipidi} hanno un gruppo preniloco o un gruppo formato da un acido grasso o da un gruppo glicosilfosfatidilico legato covalentemente.
Le \emph{proteine periferiche di membrana} interagiscono in modi non covalenti con altre proteine o lipidi a livello della superficie della membrana

Le proteine di membrana catalizzano reazioni chimiche, mediante il flusso dei nutrienti e di sostanze di scarto, attraverso la membrana e partecipano al trasferimento delle informazioni riguardanti l'ambiente extracellulare a varie componenti intracellulari

Le proteine di membrana vengono classificate in:
\begin{itemize}
\item Proteine intrinseche, o integrali
\item Proteine periferiche
\item Proteine ancorate
\end{itemize}

La maggior parte delle proteine di membrana attraversano il doppio strato lipidico, e sono dette \emph{proteine integrali}, come singola elica o come eliche multiple. Altre proteine sono legate covalentemente al doppio strato attraverso legami con gli acidi grassi oppure sono legate ad un oligosaccaride, anche se meno frequentemente; sono dette \emph{proteine ancorate}. Inoltre, alcune proteine sono attaccate alla membrana attraverso interazioni non covalenti con altre proteine di membrana, e sono dette \emph{periferiche}.

\subsection{Proteine periferiche}

Possono essere dissociate dalla membrana con procedure relativamente blande, che lasciano intatte le membrane. Non legano lipidi e, una volta purificate, sono molecole solubili in acqua.
Si associano alla superficie delle membrane, legandosi attraverso interazioni elettrostatiche e legami ad idrogeno con le proteine integrali, o più difficilmente con lipidi specifici.

Il \emph{citocromo C} è una proteina periferica e si muove sulla superficie esterna della membrana interna dei mitocondri. Ha la funzione di trasportare elettroni, infatti a pH fisiologico è un catione e può interagire con i fosfolipidi carichi negativamente.

\autofullpicture*{Proteine periferiche}

\subsection{Proteine ancorate}

Contengono lipidi legati covalentemente, che ancorano la proteina alla membrana. I lipidi collegati possono avere funzioni aggiuntive, oltre a quella di ancorare la proteina, come ad esempio mediare interazioni proteina-proteina o modificare la struttura e l'attività della proteina a cui è attaccato.

Per essere isolate, è sufficiente separare i due domini, ovvero quello idrofobico e quello idrofilico.

Possono essere ancorate tramite un'\alpha-elica composta da residui idrofobici, senza lipidi legati covalentemente.

Un esempio di proteina ancorata è il \emph{citocromo B\ped{5}}, un enzima di trasporto elettronico coinvolto nel metabolismo dei lipidi nella superficie della membrana del reticolo endoplasmatico.

Dalla sua struttura, si nota subito un dominio idrofilico, che è enzimaticamente attivo e sta sulla superficie della membrana, e un segmento idrofobico, che è formato da due \alpha-eliche che ancorano la proteina alla membrana

\autofullpicture*{Proteine ancorate}

\subsection{Proteine integrali}

Sono più difficili da isolate, bisogna usare metodi hard, come i detergenti ad alta concentrazione, che entrano in competizione con la formazione delle membrane; in ogni caso è difficile isolare senza distruggere la membrana.
Sono in buona parte immerse nella membrana e sono saldamente legate a questa. Per essere legate alla membrana, queste proteine posseggono in alta percentuale di amminoacidi idrofobici, e sono presenti molto frequentemente \alpha-eliche transmembrana o \beta-barrels

Sono proteine anfifiliche, quindi i segmenti proteici immersi nella parte interna della membrana hanno in prevalenza residui idrofobici, mentre quelli esterno hanno una maggioranza di residui idrofilici.
Sono strutture asimmetriche e le proteine sono orientate in un'unica direzione.

\autofullpicture*{Proteine integrali}

Alcuni esempi di queste proteine sono le \emph{porine}. Sono proteine della membrana mitocondriale esterna e della membrana esterna dei batteri Gram-negativi. Possiedono circa 300 residui amminoacidici che formano 16 \beta-sheets organizzati a barilotto (\beta-barrel). Il diametro del poro è di due nm.

Queste proteine sono in grado di far passare attraverso la membrana piccole molecole polari, come l'acqua. Questo avviene grazie alle proprietà fattorizzate dei \beta-sheet che formano una superficie esterna idrofobica e una superficie interna polare. In questo modo si crea un canale idrofilico transmembrana

Altre proteine sono le \emph{prostaglandine H\ped{2} sintetasi 1}. Le prostaglandine sono mediatori dei processi infiammatori; l'inibizione della sintesi delle prostaglandine è il meccanismo di azione di una classe di farmaci, ovvero i FANS.

La \emph{prostaglandina H\ped{2} sintasi 1} è ancorata tramite \alpha-eliche con amminoacidi idrofobici. All'esterno c'è il dominio solubile in acqua. L'ancora, oltre a tenere ancorata la proteina, ha una funzione catalitica, mentre la parte esterna forma un canale idrofobico per il substrato.

Un altro esempio di questo tipo di proteine è la \emph{Glicoforina}, che è una proteina caratteristica degli eritrociti. Questa proteina presenta oligosaccaridi per aumentare la solubilità degli eritrociti. Questa proteina presenta tre domini:
\begin{itemize}
\item \emph{Esterno:} il dominio esterno possiede 72 residui e 16 catene di carboidrati
\item \emph{Transmembrana:} il dominio transmembrana possiede 19 residui idrofobici
\item \emph{Interno:} il dominio interno presenta 40 residui, di cui una buona parte sono carici o polari
\end{itemize}

\autofullpicture*{Glicoforina}

Infine, come ultimo esempio, si consideri la \emph{batteriorodopsina}, che è una proteina simile alla rodipsina, una proteina che sta nell'occhio e consente la visione in bianco è nero.

La \emph{batteriorodopsina} ha una proteina integrale di membrana utilizzata dai microrganismi \emph{Halobacteria}. Agisce come una pompa protonica, ovvero cattura la luce e la utilizza per spostare protoni attraverso la membrana. Il gradiente proteinico generato è utilizzato per produrre ATP.{}
È formata da un fascio di 7 tratti a forma di bastoncini \alpha-elica transmembrana. Il gruppo che assorbe la luce è costituito da una molecola di \emph{retinale} legata covalentemente alla proteina

La luce, eccitando la molecola, è in grado di isomerizzare l'untimo doppio legame da cis a trans. Questa isomerizzazione passa attraverso uno stato eccitato. Quando la molecola passa da trans a cis, alcuni protoini possono passare verso l'esterno della membrana,

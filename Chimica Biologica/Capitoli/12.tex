\chapter{Permeabilità della membrana}


\ChangePicturesFolder{12}

\autofullpicture*{Permeabilità della membrana}

Le cellule sono separate dal loro ambiente dalle membrane plasmatiche. Le cellule eucariote sono suddivise in compartimenti tramite le membrane intracellulari, che formano i confini e le strutture interne dei vari organelli.

Le membrane biologiche sono barriere straordinarie che regolano il passaggio di qualsiasi cosa. Sono poche le molecole che possono oltrepassarla per semplice diffusione; tra queste ci sono i gas come \ce{O2}, \ce{CO2}, \ce{NO} e piccole molecole non cariche, come l'etanolo. Anche l'acqua riesce a passare, però per diffusione ne passa molto poca.

Tutte le altre molecole hanno bisogno di specifiche proteine di trasporto. In questo modo, le membrane regolano le concentrazioni interne ed esterne della cellula. Il motivo per cui le molecole polari hanno difficoltà a oltrepassare la membrana è che questa è formata da un doppio strato fosfolipidico, quindi al suo interno è presente una zona idrofobica.

Quindi gli ioni come \ce{H+}, \ce{Na+}, \ce{K+}, \ce{Ca^{2+}} e \ce{Cl-} e i metaboliti, come piruvato, amminoacidi, zuccheri e nucleotidi necessitano di proteine di trasporto.

Alle proteine di trasporto sono associati tutti i fenomeni elettrochimici di membrana, come ad esempio i segnali nervosi.

Esistono due forme di processi di trasporto:
\begin{itemize}
\item Trasporto non mediato, che avviene per diffusione semplice, senza l'ausilio di proteine. Questo trasporto avviene secondo il gradiente di concentrazione.
\item Trasporto mediato, che ha luogo a seconda dell'azione specifica delle proteine
\end{itemize}

La forza che dirige il flusso non mediato di una sostanza è il suo gradiente chimico. Il composto si diffonde nella direzione che porta all'annullamento del gradiente ad una velocità proporzionale alla sua entità. La velocità, inoltre, dipende dalla solubilità della sostanza chimica nella parete interna, pertanto molecole non polari come gli steroidi e \ce{O2} diffondono agevolmente attraverso la membrana.

Il trasporto mediato si differenzia in:
\begin{itemize}
\item Trasporto mediato passivo, o diffusione facilitata. La proteina di membrana facilita la diffusione nel verso del gradiente di concentrazione.
\item Trasporto attivo, che necessita di energia, fornita dall'ATP, che deve essere accoppiata ad una reazione che renda il $\Delta G$ complessivo negativo. La proteina trasporta una sostanza contro il gradiente di concentrazione
\end{itemize}

\autofullpicture*{Tipologie di trasporti}

La termodinamica del trasporto segue questa equazione
\[
N_{\text{A}} = N_{\text{A}}^0 + RT \ln \ce{[A]}
\]

Il $\Delta G$ transmembrana è
\[
\Delta G = RT \ln \frac{\ce{[A_{\text{in}}]}}{\ce{[A_{\text{out}}]}}
\]

Se si hanno degli ioni, è necessario aggiungere il potenziale elettrico generato dalla differenza di cariche elettrostatiche ai due lati della membrana, che viene indicato come $\Delta \Psi$
\[
\Delta G = RT \ln \frac{\ce{[A_{\text{in}}]}}{\ce{[A_{\text{out}}]}} + z_{\text{A}} F \cdot \Delta \Psi
\]
dove $z_{\text{A}}$ è la carica elettrica di dello ione \ce{A}, $F$ è la costante di Faraday, ovvero 96\;500 C/mol e $\Delta \Psi$ è solitamente pari a -100 mV.

Questo comporta che per specie ioniche, l'equilibrio termodinamico non corrisponde alla situazione dove le due concentrazioni sono uguali.

Per il trasporto attivo, si può dire che
\[
\Delta G = RT \ln \frac{\ce{[A_{\text{in}}]}}{\ce{[A_{\text{out}}]}} + \Delta G'
\]
dove il termine $\Delta G'$ viene da una reazione accoppiata e deve garantire che il $\Delta G$ complessivo sia negativo.

Per quanto riguarda la velocità, il trasporto per diffusione la seguente equazione cinetica
\[
J_{\text{A}} = P_{\text{A}} \cdot D \cdot \Delta \ce{[A]}
\]
dove $J_{\text{A}}$ è il flusso, $P_{\text{A}}$ è il coefficiente di permeabilità e $\Delta \ce{[A]}$ è la differenza di concentrazione tra interno ed esterno della membrana. Il coefficiente di permeabilità è di solito basso.

\autofullpicture*{I trasportatori specifici velocizzano, agendo da catalizzatori, abbassando le barriere energetiche. Controllando i trasportatori, si controlla il flusso. Con i trasportatori, la dipendenza non è più lineare}

\section{Trasporto facilitato}

È operato da pori, canali, quali gli ionofori, le porine, i canali ionici e le acquaporine, e da proteine di trasporto, come ad esempio la GL1. Gli ionofori possono trasportare ioni o fare dei canali. Le porine forniscono una via di passaggio per ioni o soluti non polari.

I canali ionici sono altamente selettivi e possono essere regolati. Le aquaporine mediano il passaggio transmembrana dell'acqua. Le proteine di trasporto possono mediare l'uniporto, il simporto e l'antiporto.

\autofullpicture*{I trasportatori hanno solitamente due conformazioni}

\automarginpicture*{Tramite il trasporto mediato, l'energia di attivazione $\Delta G^{\ddagger}$ viene abbassata notevolmente.}

Un esempio è la \emph{valinomicina}, che è un trasportatore dello ione \ce{K+}. Il controllo degli ioni quali \ce{Na+}, \ce{Cl-} e \ce{K+} è estremamente importante per il mantenimento dell'equilibrio osmotico, per la trasduzione del segnale e per generare le variazioni nel potenziale di membrana necessarie alla trasmissione di impulsi nervosi.

Dunque, la \emph{valinomicina} è un canale estremamente specifico per lo ione \ce{K+}. Infatti lo ione potassio si adatta perfettamente alla cavità.

Questo trasportatore è uno ionoforo ed è usato come antibiotico, infatti esso annulla il gradiente di concentrazione del potassio, che è estremamente utile in molti processi, come ad esempio la sintesi di ATP. Infatti, contribuisce al mantenimento del potenziale elettrochimico necessario per compiere lavoro per la sintesi di ATP.

Un altro esempio sono le \emph{acquaporine}, che sono delle proteine che contengono un \beta-barrel e che regolano il passaggio dell'acqua. Il \beta-barrel ha una superficie interna idrofilica e una esterna idrofobica.

Il trasporto facilitato mediato da proteina ha le seguenti caratteristiche:
\begin{itemize}
\item \emph{Velocità:} il passaggio avviene molto velocemente
\item \emph{Specificità:} le proteine sono altamente specifiche per far passare sono certe molecole o certi ioni
\end{itemize}

La proteina può essere inattivata chimicamente, oppure può essere inibita tramite inibizione competitiva.
Nei globuli rossi, il glucosio non riesce a entrare da solo. È necessario l'intervento della \emph{GLUT1}, che è un trasportatore del glucosio localizzato nella membrana dell'eritrocita.

\autofullpicture*{Legge cinetica per il glucosio negli eritrociti. Si nota l'andamento simile a quello di Michaelis-Menten.}

La legge cinetica non è più lineare, ma è più simile alla legge di Michaelis-Menten
\[
J = \frac{J_{max} \ce{[A]_{\text{out}}}}{K_M + \ce{[A]_{\text{out}}}}
\]

\automarginpicture*{GLUT1}

Questa proteina è soggetta ad inibizione competitiva da parte di qualsiasi molecola di forma simile al glucosio che si lega alla proteina, formando un buon legame. Il funzionamento di questa proteina è diviso in quattro passaggi:
\begin{enumerate}
\item Il glucosio si lega alla proteina su un lato della membrana
\item Una variazione conformazionale chiude il primo sito di legame ed espone quello dall'altro lato della membrana. Si ha quindi il trasporto.
\item Il glucosio si dissocia dalla proteina
\item Il ciclo di trasporto si completa grazie al ritorno di GLUT1 alla sua conformazione iniziale in assenza di glucosio legato. Si ha quindi il recupero
\end{enumerate}

Il GLUT1 è una glicoproteina da 12 \alpha-eliche, che pesa 55 kDa. I domini IC1, IC2 e IC3 regolano l'apertura e la chiusura verso l'interno.
L'esempio appena visto è un \emph{trasporto in uniporto}

\autofullpicture*{Legge cinetica per il glucosio in presenza e in assenza di trasporto facilitato.}

\autofullpicture*{Funzionamento del GLUT1}

\section{Trasporto attivo}

Il trasporto attivo è più complesso perché è necessario convertire l'energia in trasporto chimico. Le pompe utilizzano l'energia libera dell'ATP per trasportare gli ioni contro il loro gradiente di concentrazione.

Questo è un processo endoergonico, che quindi richiede l'accoppiamento con un processo esoergonico.
Le pompe ioniche consentono il trasporto tramite l'idrolisi di ATP.{}

I gradienti ionici così ottenuti sono associati a determinati processi:
\begin{itemize}
\item Controllo del volume cellulare
\item Eccitabilità delle membrane nelle cellule nervose e muscolari
\item Guida del trasporto attivo secondario di zuccheri e di amminoacidi
\item Mantenimento del gradiente \ce{Na+}/\ce{K+} attraverso la membrana plasmatica degli eucarioti superiori
\end{itemize}

Le concentrazioni di \ce{Na+} e \ce{K+} sono controllati dalla pompa sodio-potassio, ovvero (\ce{Na+}-\ce{K+})-ATPasi. La pompa sodio-potassio è accoppiata all'idrolisi di ATP.{}

\marginbox*{Dal 30\% al 50\% di ATP prodotta è utilizzata per pompare ioni}

\autohalfpicture*{Differenza tra trasporto in uniporto, trasporto in simporto e trasporto in antiporto}

La reazione che avviene è la seguente
\[
3 \ce{Na+_{\text{in}}} + 2 \ce{K+_{\text{out}}} + \text{ATP} \ce{->} 3 \ce {Na+_{\text{out}} + 2 \ce{K+_{\text{in}}} + \text{ADP} + \text{P}}
\]

\paragraph{Pompa sodio-potassio}

\autofullpicture*{Diagramma schematico della pompa sodio-potassio della membrana plasmatica dei mammiferi}

L'idrolisi dell'ATP avviene sul lato citoplasmatico della membrana. Gli ioni \ce{Na+} vengono trasportati fuori dalla cellula e gli ioni \ce{K+} vengono trasportati all'interno. La stechiometria del trasporto è che per ogni molecola di ATP idrolizzata, 3 ioni \ce{Na+} escono, mentre 2 ioni \ce{K+} entrano.
L'ouabaina ed altri glicosidi cardiaci inibiscono la pompa sodio potassio, legandosi sulla superficie extracellulare della pompa.

Sono steroidi cardiotonici, si chiamano così perché hanno effetti cardiotonici. Sono inibitori della pompa. Queste molecole si legano saldamente sul sito attivo, inibiscono la pompa bloccando il passaggio S. In questo modo la concentrazione di sodio aumenta. Questo stimola una pompa scambiatore antiporto \ce{Na+}-\ce{K+} che pompa \ce{Na+} fuori ma fa entrare \ce{Ca^{2+}} nel reticolo sarcoplasmatico. Lo ione calcio è il mediatore della contrazione muscolare.
L'ouabaina era usata come veleno nelle freccie, viene da una pianta in Africa

La fase fondamentale è la fosforilazione di un sito specifico residuo di Asp. L'ATP fosforila il trasportatore solo in presenza di \ce{Na+}, mentre il risultante residuo, detto Aspartil Fosfato, è sottoposto a idrolisi solo in presenza di \ce{K+}. Ciò indica che la pompa sodio-potassio possiede solo due stati conformazionali, detti E\ped{1} e E\ped{2}, dotati di struttura differente, attività catalitiche diverse e differenti specificità per il ligando. La proteina sembra agire nel modo seguente
\begin{enumerate}
\item E\ped{1}-ATP, che ha ricevuto la suo molecola di ATP all'interno della cellula, lega ioni \ce{Na+} per dare luogo al complesso ternario E\ped{1}-ATP-3 \ce{Na+}
\item Il complesso reagisce per formare un intermedio ad alta energia E\ped{1}$\sim$P-3 \ce{Na+} e ADP che viene rilasciato all'interno della cellula
\item Questo intermedio si rilassa nella pompa conformazionale a ``bassa energia'' e rilascia fuori dalla cellula \ce{Na+} che in precedenza era legato all'intermedio. Si ricordi che E\ped{1} ha più affinità per \ce{Na+}, mentre E\ped{2} ha più affinità per \ce{K+}
\item E\ped{2}-P si unisce a due ioni \ce{K+} provenienti dall'esterno della cellula per dare origine ad un complesso E\ped{2}-P-2 \ce{K+}
\item Il gruppo fosforico è idrolizzato producendo E\ped{2}-2 \ce{K+}
\item E\ped{2}-2 \ce{K+} cambia conformazione diventando E\ped{1}, lega ATP e rilascia i suoi due ioni \ce{K+} dentro la cellula e li sostituisce con tre ioni \ce{Na+}
\end{enumerate}

Anche se le reazioni elencate sono tutte reversibili, il processo può avvenire solo in questa direzione.

\paragraph{\texorpdfstring{$ \text{Ca}^{\text{2+}} $}{Ca2+} ATPasi}

Agisce spesso da secondo messaggero intracellulare. Un aumento transitorio della concentrazione di \ce{Ca^{2+}} innesca alcune risposte cellulari, tra cui la contrazione muscolare, il rilascio di neurotrasmettitori e la demolizione di glicogeno.

La concentrazione di \ce{Ca^{2+}} è elevata all'esterno delle cellule (1.5 mM), mentre è bassa all'intero (10\ped{-4} mM). Questo crea un gradiente molto elevato, che viene mantenuto dalla pompa per il calcio e da uno scambiatore \ce{Na+}/\ce{Ca^{2+}}.

Inoltre, dentro la cellula, ci sono molti compartimenti intracellulariin cui la concentrazione del calcio è molto più elevata che nel citosol. Questi compartimenti sono i mitocondri, il reticolo endoplasmatico ER, che funge da riserva di \ce{Ca^{2+}} e il reticolo sarcoplasmatico SR del muscolo scheletrico, in quanto è necessario tenere alta la concentrazione di calcio nel muscolo a riposo. Per impulso nervoso, il \ce{Ca^{2+}} viene liberato nel citosol.

Il meccanismo è simile a quello della pompa sodio-potassio.

\autofullpicture*{Pompa \ce{Ca^{2+}} ATPasi}

Il ciclo del trasporto di \ce{Ca^{2+}}-ATPasi coinvolge almeno cinque conformazioni della proteina, che sono: E\ped{1} $\cdot$ \ce{Ca^{2+}}, E\ped{1}$\cdot$ ATP$\cdot$ 2\ce{Ca^{2+}}, E\ped{1}$\cdot$ P-ADP$\cdot$ 2\ce{Ca^{2+}}, E\ped{2}$\cdot$ P\ped{i} e E\ped{2}.

La reazione è uguale a quella della pompa sodio-potassio, e avviene solo in presenza di \ce{Ca^{2+}}.

È una pompa che ha una funzione di trasporto attivo primario, ma può anche sfruttare il gradiente ionico generato dal trasporto attivo primario, quindi ha anche una funzione di \emph{trasporto attivo secondario}.

Nel trasporto attivo guidato da gradienti ionici, l'energia libera immagazzinata in un gradiente di potenziale elettrochimico, può essere utilizzata per favorire trasporti attivi, ovvero senza idrolisi diretta di ATP.

Ad esempio, lo scambiatore \ce{Na+}/\ce{Ca^{2+}} è uno scambiatore in antiporto. Questo scambiatore dà una mano alla pompa \ce{Ca^{2+}}-ATPasi, infatti sfrutta il gradiente di \ce{Na+}, mantenuto dalla pompa sodio potassio, per pompare fuori altro \ce{Ca^{2+}} fuori dalla cellula. In realtà questo scambiatore ha una maggiore capacità di portare fuori lo ione calcio, con 2000 molecole di \ce{Ca^{2+}} contro le 30 della pompa calcio.

Il secondo esempio è il trasporto in simporto di \ce{Na+}-glucosio. Questa pompa sfrutta sempre il gradiente di \ce{Na+} per far entrare nella cellula il glucosio

\autofullpicture*{Il sangue scorre verso il basso e avrà una concentrazione di glucosio sempre minore di quella nella cellula.}

Un altro esempio è lo scambiatore lattosio permeasi nei batteri E. Coli. Questo enzima sfrutta il gradiente protonico attraverso la membrana della cellula per trasportare insieme \ce{H+} e lattosio. Ci sono altri scambiatori, che operano a livello del pH e servono a trasportare i metaboliti, come gli zuccheri.






\chapter{Fosforilazione ossidativa}

\ChangePicturesFolder{13}

La sintesi di ATP è molto importante in quanto è alla base della biologia energetica. Avviene nei mitocondri. L'energia necessaria alla sintesi di ATP deriva da un gradiente protonico.

Il processo si chiama \emph{fosforilazione ossidativa} perché si aggiunge un gruppo fosfato (fosforilando) all'ADP, secondo la reazione
\[
   \ce{ADP} + \ce{P} \ce{->} \ce{ATP}
\]
e perché, per ottenere il gradiente protonico, ci sono dei substrati ridotti derivanti dal metabolismo, che vengono ossidati
\[
   \text{substrati} \ce{->[ossidazione]} \text{energia}
\]

Nei mitocondri, la sintesi di ATP, a partire da ADP e fosfato è catalizzata dall'ATP-sintasi (detta anche complesso V) tramite l'energia ottenuta dal processo di trasporto degli elettroni (processo redox).

La domanda che i biologi si sono posti e che si pongono tuttora è
\begin{quoting}
L'energia libera rilasciata dal trasporto di elettroni attraverso i complessi I-IV deve essere conservata in una forma utilizzabile dall'ATP-sintasi. Qual'è questa forma?
\end{quoting}

La teoria generalmente accettata è la \emph{teoria chemiosmotica}. L'energia libera del trasporto degli elettroni viene conservata nel gradiente elettrochimico protonico attraverso la membrana mitocondriale interna

\autofullpicture*{Membrana di un mitocondrio}

Lo spazio intermembrana è topologicamente equivalente al citosol perché la membrana mitocondriale esterna è permeabile agli ioni \ce{H+}

\autofullpicture*{Membrana mitocondriale interna}

Gli elettroni, che sono rappresentati dalla freccia blu, che si muovono, favoriscono la traslocazione dei protoni attraverso la membrana mitocondriale interna da parte dei complessi I, III e IV  della matrice (con una massa concentrazione di \ce{H+}) allo spazio intermembrana, che invece possiede un'alta concentrazione di \ce{H+}.
L'energia libera del processo è
\[
   \Delta G = 2.3 RT [pH_{\text{in}} - pH_{\text{out}}] + zF \Delta \Psi \qquad \Delta \Psi = 100 \text{ mV} \: \Delta pH = 0.75
\]

Il potenziale di membrana e il pH si oppongono alla traslocazione di \ce{H+} verso l'esterno: il costo energetico è pagato dal trasporto elettronico.

\automarginpicture*{ATP-sintasi}

Ci sono due unità funzionali, F\ped{0} e F\ped{1}, ciascuna composta da varie subunità e da uno stelo, necessario per legare le due unità funzionali.

\automarginpicture*{Subunità dell'ATP-sintasi}



L'unità F\ped{0} è una proteina transmembrana composta da 4--5 subunità, contenente un canale per la traslocazione degli ioni \ce{H+}, mentre l'unità F\ped{1} è una proteina periferica composta da cinque tipi di subunità (\alpha\ped{3}\beta\ped{3}\gamma\delta\epsilon). che si dissocia da F\ped{0}.

Le subunità \alpha{} e \beta{} hanno un 20 \% di sequenza identico, quasi identico e stanno attorno \gamma{} (\alpha-elica). \epsilon{} e \delta{} assieme a \gamma{} uniscono F\ped{1} a F\ped{0}.

Molte subunità a\ped{1}, b\ped{2} e ``c’’ possono variare dal 8 a 15 \%. In ognuna subunità c'è una doppia \alpha-elica. Le due subunità \beta{} e \gamma{} formano uno stelo che collega la subunità ``a'' di F\ped{0} con F\ped{1}. Anche la subunità ``a’’ è composta da \alpha-eliche.

F\ped{1} non è simmetrica perché \gamma{} non lo è, ma soprattutto perché \gamma{}, ruotando su sé stessa, cambia le conformazioni di \alpha{} e \beta{} e quindi le affinità per ATP e ADP.{}
\alpha\ped{3}\beta\ped{3} e \gamma{} costituiscono il sito catalitico o il dominio catalitico. La subunità \gamma{} è un unità rotante che viene messa in rotazione da F\ped{0} a \epsilon. Questa subunità inoltre sta nella matrice.

C\ped{n} e a stanno nella membrana e costituiscono il dominio di membrana.
a, b\ped{2}, \delta{}, e \alpha\beta\alpha\beta\alpha\beta{} sono tutte collegate tra loro e sono la parte statica della proteina, quindi non si muovono.

Ogni subunità c contiene due \alpha-eliche, tutte le subunità c sono uguali.
``a’’ è una subunità idrofobica che contiene 9 \alpha-eliche.
\epsilon{} ha notevoli punti di contatto i c e con \gamma{}.{}

\autofullpicture*{F\ped{0} è un complesso proteico transmembrana composta dalle subunità a, b e c. Conoene un canale per la traslocazione dei protoni.\\ F\ped{1} è un complesso proteico periferico composto da cinque tipi di subunità. Si dissocia da F\ped{0} per trattamento con urea.\\ Lo stelo è formato dalle unità b e \delta{} per legare F\ped{0} e F\ped{1}.}

\clearpage

\section{Meccanismo}

\autofullpicture*{Sito catalitico della sintesi di ATP: \alpha\ped{3}\beta\ped{3} esamero; \gamma{} unità rotante messa in rotazione da F\ped{0} attraverso \epsilon.}

Il meccanismo è legato ad una rotazione meccanica. Il gradiente protonico tende a spingere gli \ce{H+} verso la matrice, quindi trovano un canale di ingresso in un'unità c trammite a (i protoni entrano in un canale idrofilo che si trova nell'unità a, in posizione adiacente a c e si legano all'Asp-61 del c più vicino).

Nel disegno, \ce{H+} entra dal canale sull’unità ``a’’ e si attacca all’unità ``c’’ più vicina. Asp-61 inizialmente è legato (interagisce elettrostaticamente) con Arg-210 di a, che  protonandosi, si indebolisce (si protona subito perché la concentrazione di \ce{H+} è alta).

Allora l'Arg-210 di a tende ad interagire con l'Asp-61 della subsuitò c precedente e così facendo, il protone che era legato alla subunità c è libero di andarsene. Per andarsene, segue un altro canale idrofilo di a per uscire dalla parte opposta. Questo processo è favorito perché dall'altra parte ci sono pochi ioni \ce{H+}. Questo fa girare l'unità c.

Il $\Delta G$ del gradiente protonico viene convertito in energia meccanica. Ma in che modo questa rotazione produce ATP? La rotazione delle unità c è trasmessa alla subunità \gamma. La subunità \gamma{} cambia le conformazioni di \alpha{} e \beta{}; in realtà il movimento di \gamma{} non è continuo ma è a scatti.

Le unità c si muovono in contiguo mentre \gamma{} fa uno scatto ogni quattro protoni, in quanto ci sono 12 unità c, almeno nel caso di E. Coli. La subunità \gamma{} quindi assume tre conformazioni.
Quella del dominio catalitico è detta \emph{pseudorotazione} perché non ruota, ma cambiano solo le conformazioni.{}
\gamma{} è una doppia elica, formata da due \alpha-eliche superavvolte.{}
Ciascuno dei tre complessi \alpha\beta{} possiede una diversa conformazione stabilita da \gamma; queste conformazioni possono essere L (loose), T (tight) e O (open).

\autofullpicture*{Nella prima immagine ADP e P si legano debolmente al sito di legame nella conformazione L. Nella seconda immagine, la subunità \gamma{} ruota e cambia conformazione da L a T. In questa conformazione, la formazione di ATP è catalizzata. Nella terza immagine, ADP  e P nel sito T formano ATP, mentre una molecola di ATP precedentemente sintetizzata viene lasciata dal sito O.}

Ogni rotazione di \gamma{} forma una molecola di ATP, quindi per formare tre molecole di ATP e far tornare la macchina molecolare al suo stato originario, servono 12 ioni \ce{H+}. Questo significa anche che uno ione \ce{H+} produce \nicefrac{1}{4} di molecola di ATP.{}

\autoherepicture{0.7}

È stato anche verificato lo spostamento del rotore \gamma\epsilon-c rispetto allo statore ab\ped{2}-\alpha\ped{3}\beta\ped{3}\delta. Per verificare questo spostamento, sono stati legati sei residui di His consecutivi alle subunità \alpha. L'istidina si attacca ad un rivestimento sulla superficie del vetro, ovvero all' acido \ce{Ni^{2+}}-nitriloacetico. In seguito, il contenitore in vetro è stato capovolto.

Ad una subunità c, si attacca un filamento di actina marcata con dei coloranti fluorescenti.

La ATP-sintasi è stata fatta lavorare a rovescio, quindi è stata usata per pompare protoni idrolizzando ATP. Quindi è stata vista la rotazione.

\marginbox*{L'ATP-sintasi piega un po' la membrana del mitocondrio. Le curve strette del mitocondrio ci sono quando sono presenti molti ATPasi vicini}

Una volta sintetizzato l'ATP, questo deve uscire dal mitocondrio. Quindi esiste una proteina di trasporto in antiporto ATP/ADP che fa uscire l'ATP e fa entrare l'ADP.{} In seguito c'è una proteina che fa entrare gli ioni fosfato in simporto con \ce{H+}

\autofullpicture*{Trasporto di ATP e \ce{H+} nel mitocondrio}


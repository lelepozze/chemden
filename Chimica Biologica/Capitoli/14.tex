\part{Acidi Nucleici}

\chapterpicture{header_05}

\chapter{Il flusso dell'informazione genetica}

\ChangePicturesFolder{14}

\automarginpicture*{Acido nucleico} %1

Gli acidi nucleici DNA e RNA sono direttamente coinvolti nei processi di immagazzinazione e decodificazione delle informazioni genetiche. La forma e le attività delle singole cellule sono determinate, in larga misura, dalle istruzioni genetiche contenute nel DNA, o dell'RNA in alcuni virus.

Il dogma centrale della biologia molecolare è che:
\begin{quoting}
Le sequenze delle basi nucleotidiche contenute nel DNA codificano quelle amminoacidiche delle proteina
\end{quoting}

In definitiva il DNA contiene l'informazione genetica nella sua sequenza di nucleotidi, che viene trascritta nella sequenza dei nucleotidi dell'RNA messaggero, che viene a sua volta tradotta in una proteina, ovvero una sequenza di amminoacidi

Ci sono diversi tipi di RNA:
\begin{itemize}
\item \emph{RNA-messaggero:} dopo essere stato trascritto dal DNA, l'RNA messaggero dirige la sintesi delle proteine sui ribosomi, nel processo di traduzione
\item \emph{RNA-transfer:} durante la sintesi proteica, prota gli amminoacidi sui ribosomi, secondo le istruzioni del mRNA
\item \emph{RNA-ribosomiale:} presente nei ribosomi, costituiti per due terzi da RNA e per un terzo da proteine, che in questo caso hanno un ruolo strutturale e funzionale
\end{itemize}

L'RNA può essere associato a proteine specifiche, ovvero le ribonucleoproteine, che partecipano a processi di modifica post-traduzionale

Ci sono diverse evidenze che il DNA è il vettore dell'informazione genetica:
\begin{itemize}
\item Tecnologia del DNA ricombinante
\item Esperimenti transgenici
\item Mutamento del fenotipo in base al genotipo
\item Esperimento di Griffith
\end{itemize}

\marginbox{Genotipo e fenotipo}{Il genotipo è la costituzione genetica di un organismo, mentre il fenotipo è descritto come le caratteristiche osservabili in un organismo, che sono determinate da caratteristiche genetiche e ambientali}

All'inizio si pensava che l'informazione genetica fosse contenuta nelle proteine, tuttavia, in seguito all'esperimento di Griffith, si è visto che le informazioni genetiche sono mediate dagli acidi nucleici

\section{Struttura e chimica}
Il DNA è un polimero lineare di desossiribonucleotidi, mentre l'RNA è un polimero lineare di ribonucleotidi
La direzione del DNA è $5' \to 3'$, mentre il legame fosfodiesterico è tra il carbonio $3'$ e il carbonio $5'$.
La base azotata può essere una pirimidina o una purina.

\autoherepicture{0.8} %3


\autofullpicture*{Differenza tra RNA e DNA} %3

Si può anche distinguere tra:
\begin{itemize}
\item \emph{Nucleoside:} base azotata legata al ribosio, o al desossiribosio
\item \emph{Nucleotide:} nucleoside 5'-monofosfato, ovvero nucleoside legato ad un gruppo fosfato
\end{itemize}

La base e lo zucchero sono tenuti insieme da un legame glicosidico sul C1; due nucleotidi sono tenuti insieme da un legame fosfodiesterico.

\automarginpicture*{Pirimidine} %4

\automarginpicture*{Purine} %4

\begin{fullpaper}
\begin{table}
\begin{tabular}{lcccc}
Base & Sigla & Struttura & Nucleoside & Nucleotide\\
Adenina & Ade (A) & \tabfigure{width=0.18\textwidth}{A} & Adenosina (Ado) & Adenosina monofosfato (AMP)\\
Giuanina & Gua (G) & \tabfigure{width=0.25\textwidth}{G} & Guanosina (Guo) & Guanosina monofosfato (GMP)\\
Citosina & Cyt (C) & \tabfigure{width=0.18\textwidth}{C} & Citidina (Cya) & Citidina monofosfato (CMP)\\
Uracile & Ura (U) & \tabfigure{width=0.18\textwidth}{U} & Uridina (Ura) & Uridina monofosfato (UMP)\\
Timina & Thy (T) & \tabfigure{width=0.2\textwidth}{T} & Deossitimidina (dThd) &  Deossitimidina (dTMP)\\
\end{tabular}
\caption{Basi azotate. Si nota che l'uracile U è legato solo nell'RNA, mentre la timina è legata solo al DNA.}
\end{table}
\end{fullpaper}

Una catena di zuccheri uniti da un legame fosfodiesterico (3'-5') costituisce lo scheletro covalente dell'acido nucleico. Le parti variabili sono le basi derivate dalla purina e dalla pirimidina.

\marginbox{Basi puriniche}{Sono altamente coniugate, quindi sono quasi planari.\\ Queste basi, in soluzione acquosa sarebbero in equilibrio tautomerico (con il tautomero imminico), tuttavia nel DNA, stanno nella loro forma amminica, che è la meno stabile in acqua}

\marginbox{Basi pirimidiniche}{Sono anche queste delle basi altamente coniugate e, come per le basi puriniche, sono planari. Anche in questo caso la forma predominante nel DNA è la forma amminica}

\section{Proprietà dei nucleotidi}

I nucleotidi sono acidi forti, in quanto contengono un gruppo fosfato. La prima dissociazione ha $pk_a=0.9$, mentre la seconda dissociazione ha $pK_a=6.1$.

La presenza di anelli coniugati nelle basi causa la capacità di assorbire nell'UV-Vis; nel DNA denaturato, questo effetto è più intenso. Questo effetto si chiama effetto ipercromico.

\autofullpicture*{Desossiribonucleotidi e ribonucleotidi.} %5

Come detto in precedenza, la sequenza nucleotidica contiene l'informazione. È importante riconoscere questa informazione, anche per poterla modificare.
È più facile sequenziare il DNA, che non le proteine, almeno per piccoli tratti di DNA. Esistono poi dei sistemi complessi che possono sequenziare genomi interi.
La complessità del DNA comporta diverse funzioni, oltre a codificare la sintesi delle proteine, la maggior parte delle quali resta sconosciuta.

È studiato inoltre per studiare l'evoluzione, riconoscere discendenti o identificare parentele, grazie alla specificità del DNA.
Altre utilità nello studiare il DNA e quello di capire come le cellule possano sviluppare tumori, come funzionano le varie tipologie di RNA e capire cosa viene trasmesso, oltre al patrimonio genetico.
Il campo di ricerca per il DNA è molto vasto.

\section{Legame fosfodiestere}

Il legame fosfodiesterico è metastabile, in quanto il $\Delta G_{\text{formazione}} > 0$. Quindi, anche in questo caso, il DNA è una proteina metastabile, come le proteine.

Ci sono reperti di DNA di milioni di anni fa ancora intatti, quindi la sua cinetica di idrolisi è molto lenta.

Infatti, nelle cellule, ci sono degli enzimi fatti apposta per idrolizzare, ovvero le \emph{nucleasi}. Questi enzimi sono importanti forbici molecolari e vengono studiate molto nella manipolazione genetica.

Si può catalizzare l'idrolisi anche in vitro,con acidi o basi. La catalisi acida causa l'idrolisi sia del DNA, che dell’RNA, mentre la catalisi basica causa l'idrolisi dell'RNA. Questo avviene in quanto nel DNA non c'è il gruppo ossidrile in posizione 2'

\autoherepicture{0.8} %7

Nella sintesi, la reazione avviene con nucleotidi ditrifosfati; in questo modo si fornisce l'energia per far avvenire la reazione di sintesi. In seguito c'è un enzima che velocizza la polimerizzazione, detto \emph{polimerasi}.

\[
\text{\small Nucleoside trifosfato } + \ce{H2O} \ce{<=>} \text{\small Nucleoside monofosfato } + \ce{PP_{i}} \quad \Delta G\standard = -31 KJ/mol
\]
\[
\text{\small Catena}_N + \text{\small Nucleoside monofosfato } \ce{<=>} \text{\small Catena}_{N+1} + \ce{H2O} \quad \Delta G\standard = +25 KJ/mol
\]

Lo zucchero coinvolto nella sintesi di acidi nucleici è il \emph{ribosio}. Il ribosio è quasi planare, mentre il desossiribosio non lo è.

\marginbox*{
    La composizione di basi del DNA di un dato organismo è caratteristica e indipendente dai fattori ambientali.
}

La composizione di basi del DNA è governata dalle \emph{regole di Chargaff}.
Erwin Chargaff infatti scoprì che, nel DNA, il numero di residui di adenina è uguale al numero di residui di timina. Lo stesso discorso vale anche per la citosina e la guanina.
L'RNA, invece, è un filamento singolo e non ha limitazioni appartenenti nella composizione in basi.

\begin{table}[h]
\begin{tabular}{lcc}
& A:T & G:G \\
Essere umano & 1.00 & 1.00 \\
Salmone & 1.02 & 1.02 \\
Grano & 1.00 & 0.97 \\
Lievito & 1.03 & 1.02 \\
E. Coli & 1.09 & 0.99 \\
\end{tabular}
\caption{Rapporto delle basi azotate in alcuni esseri viventi}
\end{table}

La differenza di un genoma rispetto a un altro riguarda la \emph{qualità} e non la \emph{quantità}. In particolare, la complessità di una specie non correla con il numero di cromosomi, né con la grandezza del genoma o con il numero di feni. Per questo si dice che la complessità è data da un DNA di qualità.
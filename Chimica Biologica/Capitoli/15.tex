\chapter{DNA}

\ChangePicturesFolder{15}

Rosalind Franklin fu la prima ad ottenere l'immagine della diffrazione ai raggi X di una fibra di DNA. Il DNA che scoprì fù in seguito definito ``DNA-B''.

Quest'immagine fornì la prova fondamentale per la comprensione strutturale proposta da Watson e Crick, ovvero il \emph{modello strutturale a doppia elica}.
In questo modello, il motivo a croce nell'immagine indica la presenza di una elica, o meglio, di una doppia elica.

Il DNA è quindi caratterizzato da 10 coppie di residui per giro, con 36° di rotazione tra una base e la successiva.
I due filamenti di DNA sono antiparalleli, quindi ciascun filamento forma un'elica destrorsa.

Le basi stanno all'intero e le catene zucchero-fosfato stanno all'esterno, in modo tale da diminuire le repulsioni tra gruppi fosfati caricati negativamente.
La superficie presenta due scanalature di differente ampiezza, che vengono definite \emph{scanalatura maggiore} e \emph{scanalatura minore}.

Ogni base forma dei legami ad idrogeno con la base appaiata nel filamento opposto, dando luogo ad una coppia di basi planari. Gli pagamenti sono A-T e G-C, ovvero tra basi complementari
Le basi sono quasi perpendicolari all'asse e le coppie nello stesso filamento distano 0.34 nm. Il passo della doppia elica è di 3.4 nm.

\automarginpicture*{Struttura del DNA}

\autofullpicture*{Legami a idrogeno tra le basi. Le coppie di basi hanno una lunghezza identica; ogni coppia può quindi essere inserita senza distorsioni. La lunghezza ottimale per il legame a idrogeno è di 2.8 3.0 \AA. È per questo motivo che le basi stanno in questa forma.}

Nell'immagine \ref{img:DNAstacked}, si vedono i gruppi fosfati all'esterno e le basi all'interno. Le basi sono appaiate in modo standard l'una sull'altra, dando un contributo alla stabilità della doppia elica. Le forze coinvolte in queste interazioni sono: le forze di van der Waals, l'effetto idrofobico e i legami ad idrogeno.

\autohalfpicture{Vista dall'alto del DNA}{img:DNAstacked}

I due filamenti sono tenuti assieme solo da forze deboli, quindi sono facilmente separabili, il che è fondamentale per i meccanismi in cui il DNA è coinvolto, come ad esempio la replicazione.
Il DNA appena descritto è detto ``DNA-B'', che è più abbondante in natura. È nota solo un'altra forma di DNA, che viene chiamato ``DNA-Z''.
Le catene complementari agiscono reciprocamente da stampo nella replicazione del DNA.

\autofullpicture*{Appaiamento delle basi azotate}

\clearpage

Watson e Crick pubblicarono l'ipotesi di replicazione in base al loro modello.

\begin{quoting}
Dato l'ordine reale delle basi di una catena, si potrebbe scrivere l'ordine esatto delle basi sull'altra catena, perché l'appaiamento è specifico.
Una catena è quindi, per così dire, il complemento dell'altra ed è questa caratteristica che suggerisce il modo in cui la molecola di DNA si potrebbe duplicare \ldots Ora il nostro modello è, in effetti, una coppia di stampi, ciascuno complementare all'altro.

Noi immaginiamo che prima della duplicazione, i legami ad idrogeno si spezzino e che le due catene si svolgano e si separino. Ciascuna catena agisce quindi da stampo per la formazione di una nuova catena, cosicché alla fine si avranno due doppie di catene, dove prima ne avevamo solo una. Inoltre la sequenza delle coppie di basi sarà duplicata esattamente
\end{quoting}

Le possibili vie di replicazione sono tre:
\begin{itemize}
\item Conservativa
\item Semiconservativa
\item Dispersiva
\end{itemize}

\automarginpicture*{Vie di replicazione del DNA}

È stato dimostrato che la replicazione del DNA è semiconservativa, nell'esperimento di Meselson e Stahl, nel 1958.

Una coltura del batterio di E. Coli è stata fatta prolificare in presenza di azoto pesante \ce{^{15}N} e sono state generate 14 generazioni. La 14-esima generazione aveva il DNA composto solo da \ce{^{15}N}. In seguito, i batteri vengono trasferiti in un nuovo terreno contenente azoto \ce{^{14}N}, facendoli duplicare. In seguito è stata misurata la densità del DNA in funzione del tempo di crescita con il metodo della centrifugazione del campione.

\marginbox{Centrifugazione}{
Questa tecnica consiste nella centrifugazione del campione in una provetta contenente{CsCl}, che riesce a creare un gradiente di densità.
}

A seguito della centrifugazione, nella generazione zero è stata ottenuta una singola banda, con una densità caratteristica del \ce{^{15}N}. Nella prima generazione, è stata ottenuta una banda, con densità pari alla media di \ce{^{15}N} e \ce{^{15}N}. Nella terza generazione. Nella seconda generazione, la banda vista prima si sdoppia, formando due bande con le densità relative.

\autofullpicture*{Esperimento di Meselson e Stahl}

I dati sperimentali quindi confermano la replicazione semiconservativa.

\section{Interazioni nel DNA}

Nella struttura del DNA sono presenti le seguenti forze:
\begin{itemize}
\item Interazioni idrofobiche: l'interno del Dna è idrofobico, mentre l'estero è idrofilo, grazie alla presenza dei gruppi fosfato
\item Legami a idrogeno: i legami a idrogeno si formano tra le basi complementari
\item Forze di van der Waals: nello stacking delle basi, si formano delle forze deboli, ma additiva
\item Interazioni elettrostatiche: le interazioni elettrostatiche sono destabilizzati rispetto ai vari gruppi fosfato, ma sono neutralizzata dalle cariche positiva, che possono essere sia ioni che proteina
\end{itemize}

La struttura del DNA è rilassata, in quanto non ci sono costruzioni conformazionali. Lo scheletro formato da zuccheri-fosfati non è una struttura rigida, ma è libera di muoversi. Infatti, si ripiega per occupare il minor spazio possibile. Quando due filamenti si separano, assumono una conformazione casuale.

\autofullpicture*{Il DNA duplex può essere denaturato reversibilmente.}

\automarginpicture*{DNA parzialmente fuso}

Il DNA può denaturare e ha una sua specifica ``temperatura di fusione", che dipende dal numero di legami ad idrogeno. Il numero di legami ad idrogeno dipende dalla presenza dei nucleotidi, infatti la coppia A-T forma due legami H, mentre la coppia G-C ne forma tre. Più alta è la presenza di G-C e più alta è la temperatura di fusione.
La fusione è un processo cooperativo e reversibilmente.

\autofullpicture*{La temperatura di fusione è determinata misurando l'assorbanza a 260 nm. Infatti, gli stacking delle basi diminuiscono l'assorbanza, tramite l'effetto ipercromico. Questo effetto si riscontra solo se il DNA è nella struttura a doppia elica.}

Nella cellula, alcune proteine dette elicasi sfruttano l'ATP per preparare i due filamenti.
La temperatura di fusione dipende dal pH, forza ionica, peso molecolare e composizione del DNA.
Infatti, quando il DNA fonde, c'è un momento in cui è parzialmente fuso e le zone che si separano prima sono quelle più ricche di A-T.



\clearpage

\section{DNA-A, DNA-B e DNA-Z}

\begingroup
\begin{table}
\allpagewidth{
\begin{tabular}{lccc}  \addlinespace[3ex]
Parametri & DNA-A & DNA-B & DNA-Z\\
Senso dell'elica & Destrorso & Destrorso & Sinistrorso\\
Diametro & $\sim$ 26 \AA & $\sim$ 20 \AA{} & $\sim$ 18 \AA{}\\
Coppie di basi per giro dell'elica & 11 & 10.5 & 12 \\
Avanzamento dell'elica per coppia di basi & 2.6 \AA{} & 3.4 \AA{} & 3.7 \AA{}\\
Passo & 34 \AA{} & 34 \AA{} & 44 \AA{} \\
Inclinazione base rispetto all'asse & 20° & 6° & 7° \\ 
Increspamento dello zucchero & C3'& C2' & C2' (pirimidine) C3' (purine)\\
Conformazione del legame glicosidico & Anti & Anti & Anti (pirimidine) Sin (Purine)\\
\end{tabular}}
\caption{Tipologie di DNA}
\end{table}
\endgroup

Il DNA presente nella cellula è in forma B. Una piccola frazione può essere localmente nella forma Z, dove c'è alternanza di basi puriniche e pirimidiniche. La forma A è presente in RNA-duplex e negli ibridi DNA-RNA.
L'elica Z ha un avvolgimento sinistrorso, struttura zuccheri-fosfato a zig-zag, ha un passo maggiore, ovvero di 44 \AA{} rispetto a 34 \AA.
Questa struttura si ottiene alternando le basi puriniche e pirimidiniche, come ad esempio CGCGCG.{} Questa conformazione è stabilizzata a alte concentrazioni saline, ha 12 basi per giro e un diametro minore di 18 \AA.

\autofullpicture*{Sia nella forma A che nella forma B, la conformazione del legame glicosidico è anti, mentre dell'elica Z, i residui pirimidinici sono in tanti, mentre quelli purinici sono in sin.}

Il DNA A è stabile in concentrazioni di bassa umidità, è una forma più ampia e piatta. Le molecole di RNA-duplex sono della forma A per effetto dell'essoderile. Anche le molecole ibride DNA-RNA sono nella forma A. L'elica ja verso destrorso, il passo e di 28 \AA, si hanno 11 basi per giro, il buco assuale ha diametro 6 \AA{} e i piani delle basi sono inclinati di 20°.

Può essere immaginata come un nastro avvolto in un cilindro di diametro 6 \AA.

Nel DNA A, lo zucchero è nella conformazione a mezza sedia e ha il C3' rivolto verso l'alto. Il C3' è in posizione endo. Il C3 è spostato verso l'anello su cui sta C5'. Nel DNA-B C2' è in posizione endo, quindi C2' è spostato verso l'anello su cui sta C5'. Nel DNA Z, lo zucchero a cui sono state attaccate le basi pirimidiniche hanno il C2' in posizione endo, mentre l'altro zucchero, ovvero quello delle basi puriniche, ha C3' endo.

\marginbox*{Il termine \emph{endo} significa ``dentro'', mentre il termine \emph{eso} significa ``fuori''}

\section{Superavvolgimenti}

Il DNA può adottare forme strutturali complesse, ovvero super avvolgimenti toroidali e interwoven

\autofullpicture*{Nella figura (a) il DNA è avvolto a spirale attorno a un immaginario toroide.{} (b) Il DNA si avvolge e si impacchetta su sè stesso.{} (c) Superavvolgimenti di lunghe strutture lineari di DNA a formare delle anse, le cui estremitò sono impedite dei movimenti.}

Il DNA è lungo 1-.2 metri (patrimonio genetico) e deve stare nei nuclei della cellula, quindi è chiaro che è notevolmente impaccato. Nei batteri è presente il DNA ciclico

Considerando una molecola di DNA a doppia elica in cui ciascuno dei suoi filamenti forma un cerchio chiuso in modo circolare, si ottiene una molecola duplex circolare infatti essendo i due filamenti di DNA antiparalleli, ogni filamento deve chiudersi su sé stesso.

Il numero di avvolgimenti non può essere alterato senza prima concedere almeno uno dei suoi filamenti . Questo fenomeno è matematicamente espresso come
\[
L = T + W
\]
dove $L$ è il numero di legame, ovvero il numero di volte che un filamento di DNA si avvolge intorno all'altro, $T$ è la torsione, ovvero il numero di avvolgimenti intorno all'asse del duplex ed è positivo per giri destrorsi e negativo per giri sinistrorsi. Infine $W$ è il numero di superavvolgimenti e, come per la torsione, è positivo per i giri destrorsi.

Il numero di legame $L$ può essere variato solo rompendo uno o due filamenti e avvolgendoli in modo più o meno stretto e ricollegando le estremità. Questo processo è catalizzato dalle \emph{topoisomerasi}

Le topoisomerasi tagliano il filamento e lo riattaccano per cambiare il numero di legame, consumando ATP.

Si possono esprimere anche altre misure come:
\begin{itemize}
\item Densità della super-elica \sigma: misura l'impacchettamento. $\sigma = \nicefrac{\Delta L}{L}$
\item $\Delta L = L - L_0$, dove $L_0$ è $L$ nella forma rilassata
\end{itemize}

Il DNA superavvolto negativamente si può organizzare in una disposizione toroidale stabilizzata attorno a proteine che fungono da bobine per il nastro di DNA. Questo è importante nella formazione dei cromosomi

\autofullpicture*{Il DNA superavvolto in forma toroidale s'impacchetta facilmente attorno alle proteine che fungono da ``bobina''. Un segmento attorcigliato di DNA lineare con due superavvolgimenti negativi (a) può collassare in una conformazione toroidale se le sue estremità vengono avvicinate (b). L'impacchettamento di un toroide di DNA attorno ad una ``proteina bobina'' stabilizza tale conformazione (c).}

Una tipica cellula umana ha un diametro di 20 micron. Il materiale genetico è costituito da 23 coppie di molecole di DNA, disposte in cromosomi la cui lunghezza media è $3^9$ bp/23, ovvero sono presenti $1.3\cdot 10^8$ coppie di nucleotidi.

La lunghezza media del DNA è di 3.4 n/m/bp e quindi ogni cromosoma è lungo 5 cm. Vi sono quindi 46 molecole di DNA che corrispondono a 2 metri. Questi sono compattati in un nucleo di 5 micron di diametro!

Gli istoni e i nucleosomi sono proteine su cui si avvolgono filamenti di DNA; questo è solo il primo esempio di sovrastrutture per arrivare ai cromosomi.

Gli avvolgimenti sono fatti in modo che lo svolgimento dell'avvolgimento sua funzionale alla replicazione

\autohalfpicture*{Un modello per la struttura del cromosoma: il cromosoma umano 4.}
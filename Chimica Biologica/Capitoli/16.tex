\chapter{Replicazione del DNA}

\ChangePicturesFolder{16}

Dalla struttura si può intuire il meccanismo di duplicazione. Come prima cosa, il DNA perde la sovrastruttura. La replicazione del DNA viene effettuata da enzimi detti \emph{DNA-polimerasi}, che catalizza la formazione del legame fosfodiesterico.

Ogni filamento funziona da stampo. Le DNA-polimerasi possono allungare la catena solo aggiungendo nuovi nucleotidi al gruppo \ce{-OH} 3' libero della catena.

Molte DNA-polimerasi hanno attività nucleasica per correggere eventuali errori.

In modo semplice, il DNA perde la sovrastruttura, la doppia elica si apre e avviene la sintesi dei filamenti complementari tramite la DNA-polimerasi.

C'è una complicazione, in quanto nel DNA duplex, i due filamenti corrono in direzioni opposte e gli enzimi catalizzano solo nella direzione $5' \to 3'$. Ci sono altre DNA-polimerasi che hanno attività esonucleasica, ovvero correggono eventuali errori, idrolizzando il legame. Ciò nonostante, non è garantita una precisione di replicazione del 100 \% e questo consente di avere una variabilità genetica.

Le girasi sono enzimi che introducono dei superavvolgimenti negativi, quindi rimuove quelli già esistenti. Le elicasi separano i due filamenti del DNA utilizzando ATP.
La zona dove i filamenti sono aperti è detta bolla di replicazione e si muove lungo in unica direzione per tutta la durata della sintesi.

Le DNA-polimerasi devono stare dietro alle elicasi. I punti di ramificazione della bolla di replicazione si chiamano forcelle di replicazione ed è qui che avviene la sintesi del DNA. Le elicasi lavorano in coppia, rompendo i legami ad idrogeno

\automarginpicture*{Replicazione del DNA}

Qui è schematizzata la duplicazione sui filamenti $3' \to 5'$. In questo filamento, la DNA-polimerasi può lavorare tranquillamente nella direzione $5' \to 3'$, che è la stessa delle elicasi.

\autoherepicture{0.8}

Come già detto, la reazione avviene con un nucleotide trifosfato e può procedere solo se l'ultimo nucleotide ha un \ce{-OH} in posizione 3' libero.

Le DNA-clamp o sliding clamp (pinze scorrevoli) sono proteine a forma di anello che circondano l'elica del DNA. Sono dette pinze scorrevoli perché consentono alle DNA-polimerasi di rimanere attaccate al proprio stampo di DNA, scorrendo lungo la direzione della sintesi.

In questo modo la DNA-polimerasi può sintetizzare lunghi pezzi di DNA stampo senza abbandonare il filamento.

Le polimerasi allungano il filamento solo nella direzione $5' \to 3'$

\autoherepicture{0.8}

C'è quindi una catena più lenta perché lavora in direzione opposta, e quindi fa un pezzo di DNA alla volta. Questo Pezzo viene detto \emph{frammento di Okazaki} e consente la riproduzione semidiscontinua.

Il filamento guida procede tranquillamente in modo continuo, mentre il filamento lento è sintetizzato in modo discontinuo, sotto forma di frammenti di Okazaki.

I frammenti sono poi uniti covalentemente dall'enzima DNA-ligasi, che lega circa 1\,000 nucleotidi alla volta.

Dato che le polimerasi richiedono un gruppo ossidrilico libero in 3' per funzionare, c i deve essere qualcosa che dia il via alla sintesi. Questo iniziatore è l'RNA-primer, ogni frammento di Okazaki inizia con un pezzo di RNA primer all'estremità 5'. La sintesi della catena richiede molti RNA-primer (uno per ogni segmento), mentre per il filamento guida ne basta uno.

Il DNA maturo, tuttavia, non contiene RNA primer perché alla fine del processo, i segmenti di RNA-primer sono sostituiti dal DNA. Le SSB sono proteine che proteggono il filamento di DNA durante la sintesi.

Le DNA polimerasi hanno inoltre un dominio esonucleasico che idrolizza i legami con le base errate.

L'accoppiamento sbagliato genera una variazione conformazionale nella DNA-polimerasi, che fà entrare in gioco il dominio esonucleasico. La frequenza degli errori è di $10^{-5}$.

LaD DNA-polimerasi ha due domini:
\begin{itemize}
\item \emph{Dominio polimerico:} che catalizza la formazione del legame fosfodiesterico
\item \emph{Dominio esonucleasico:} che idrolizza i legami errati, con le basi sbagliate
\end{itemize}

Le mutazioni puntiformi, le ricombinazione e le trasposizioni sono alla base dell'evoluzione biologica.

\section{Interazione DNA-Proteine}

Il DNA è in grado di interagire con le proteine. Le proteine riconoscono sequenze specifiche su cui posizionarsi per svolgere poi la loro funzione

\autoherepicture{1}

Un esempio di proteina è l'\emph{endonucleasi}, che taglia i legami fosfodiesterici. Ci sono molte tipologie di questa proteina e sono molto specifiche. Per riconoscere il DNA, si sistemano sulle scanalature.

\automarginpicture*{Endonucleasi}

Queste proteine, con funzione di forbice molecolare, sono molto utili nella tecnologia del DNA ricombinante. Infatti, queste proteine consentono il sequenziamento e la manipolazione del DNA.
Le endonucleasi tagliano il DNA in pezzi, e dai pezzi riescono a ricostruire la sequenza

\autofullpicture*{Riconoscimento di sequenze specifiche del DNA duplex}

\autohalfpicture*{Digestione di un ipotetica sequenza di DNA mediante endonucleasi di restrizione e determinazione delle dimensioni dei frammenti e posizionamento dei siti bersaglio. Queste sono dette ``mappe di restrizione''.}

Oltre a far a pezzi il DNA, è necessario anche saperlo sequenziare. Se si mette ad esempio dd-NTP, che è l'analogo del nucleotide usato per la sintesi del DNA, solo che gli manca un \ce{OH} in posizione 3'.
\automarginpicture*{dd-NTP}
Quando la DNA polimerasi incontra questo nucleotide, si blocca perché è privo di \ce{OH} in posizione 3'

\clearpage

\subsection{Sequenziamento a terminazione di catena}

Prima si deve separare i filamenti. In seguito si aggiunge la DNA polimerasi di E.coli con il 2'-2' dideidrossinucleotidi (dd-NTP). Da qui in avanti serve il primer.

\autofullpicture*{Sequenziamento a terminazione di catena}

Si fanno quattro miscele di reazione; in ognuna si aggiunge un d-NTP, ma in più si aggiunge un dd-NTP alla miscela di reazione

In realtà ci sono dei metodi di sequenziamento automatizzato, che sequenziano molte più cose.

Un'altra tecnica importante è quella dell'amplificazione del DNA, detta PCR, tramite una reazione a catena della polimerasi.
La PCR è una tecnica estremamente sensibile, ed è utile perché amplifica una singola molecola di DNA da caratterizzare e/o manipolare.
Questa tecnica è utile:
\begin{itemize}
\item Per la diagnostica medica
\item Per la medicina forense
\item Per gli studi di evoluzione molecolare
\end{itemize}

\clearpage

\section{Sintesi proteica}

Secondo il dogma della centrale della biologia molecolare, si vede che
\[
\text{DNA} \rightarrow \text{RNA} \rightarrow \text{Proteine}
\]

\automarginpicture*{Ribosoma}

La sintesi proteica viene effettuata nei ribosomi, che sono delle macchine molecolari, dove avviene la sintesi proteica. I ribosomi hanno altre attività, infatti coordinano l'attività del tRNA, mRNA, rRNA e delle proteine coinvolte.

La scoperta dei ribosomi ha portato ad un Nobel nel 2009, a Ada Yonath.

La prima fase della sintesi proteica è la trascrizione, ovvero la sintesi del RNA messaggero da parte dell'enzima RNA-polimerasi.

Il DNA ha due filamenti $5' \to 3'$ formerà la catena di mRNA;{} viene detto codificante perché l'RNA sintetizzato avrà la sequenza identica alla catena codificante, tranne per la base azotata uracile, che verrà sostituita al posto della timina.

L'altro filamento di DNA, ovvero il $3' \to 5'$ è invece la catena stampo del DNA, ed è qui che lavora l'RNA-polimerasi.

L'RNA polimerasi è simile in forma, struttura e funzione alla DNA-polimerasi. RNAP non ha bisogno di un primer.

\autoherepicture{0.8} %2

Vi è una differenza tra procarioti ed eucarioti.
Negli eucarioti, i geni contengono parti non codificanti, dette introni ed esoni, che vanno eliminate dal trascritto primario, o splicing.

\autofullpicture*{RNA-polimerasi} %1

I procarioti invece, non hanno questa complicazione, inoltre sono presenti diversi RNAP negli eucarioti, in quanto vi sono diversi tipi di RNA, mentre nei procarioti è presente solo un RNAP è fa tutto lui. Anche i promotori sono un po' diversi, ma neanche troppo

\automarginpicture*{DNA nei procarioti e negli eucarioti} %3

I siti promotori sono sequenze di basi che danno il segnale a RNAP di partire. Nei procarioti ce n'è uno a 10 residui dalla partenza (TATAAT) detto ``prinbow box'' e uno a 35 residui (TTGACA)

L'efficacia dei promotori e proteine regolatrici regolano la frequenza di traduzione.

Dunque l'RNAP cerca sul DNA i siti promotori, srotola il DNA, si forma una bolla di trascrizione e individua il filamento stampo.

A questo punto inizia a catalizzare la formazione del filamento mRNA complementare. Infine RNAP riconosce i segnali di stop per terminare il trascritto.

Nei eucarioti c'è un pezzo in più, l'RNAP interagisce con proteine quali fattori di trascrizione che controllano l'espressione proteica.

Si ricordi che le proteine interagiscono a livello dei solchi del DNA

\autofullpicture*{Bolla di trascrizione} %4

Negli eucarioti è tutto più complicato. La trascrizione avviene nel nucleo, vi sono molti tipi di promotori riconosciuti da proteine specifiche non direttamente da RNA P e poi vi è un processo specifico di maturazione dell'mRNA (alle estremità e per lo splicing). Lo splicing è catalizzato da snRNP, ovvero le \emph{small nuclear RIBONucleo Proteins}.

\automarginpicture*{Bolla di trascrizione negli eucariori} %4

\subsection{Traduzione}

L'informazione contenuta nel mRNA deve essere convertita in una sequenza di amminoacidi e questo avviene nei ribosomi quindi l'mRNA esce dal nucleo, arriva nel ribosoma e incontra vari pezzi di tRNA, che convertono la sequenza di basi

La traduzione avviene da RNA a proteine avviene per mezzo delle triplette. Una tripletta di basi azotate codifica un unico amminoacido. La tripletta di basi è detta \emph{codone}.

Un amminoacido può essere codificato da più codoni.

Alcuni codoni non codificano alcun amminoacido, ma mandano segnali alla sintesi proteica di iniziare o terminare.

Il codice genetico è sovrapposto, nel senso che l'ultima base azotata di un codone codifica anche per il codone successivo.

Le mutazioni puntuali, ovvero di una singola base azotata, possono quindi portare alla mutazione di due residui.



Il codice degenerato è quando un amminoacido è codificato da più codoni. Le sequenze simili corrispondono ad amminoacidi simili per limitare i danni in casi di errori.

La lettura e la traduzione avviene nei ribosomi e il vero protagonista è il tRNA.{}
I codoni che codificano per un amminoacido sono detti \emph{tRNA}. Il codone AUG è il segnale di inizio.

I segnali di inizio negli eucarioti sono più complessi. I codoni di stop sono letti dalle proteine ``fattori di rilascio''.

\autofullpicture*{Sintesi proteica nel ribosoma} %6

La sintesi dei polipeptidi avviene nella direzione N-terminale $\to$ C-terminale. Il ribosoma ha due siti, nell'inizio si lega il tRNA il quale anticodone, ovvero CAU, lega il codone di inizio, ovvero AUG. Poi si lega il secondo tRNA con un amminoacido, poi arriva il terzo e tutti gli altri scalano una posizione e intanto gli amminoacidi si attaccano. La metionina iniziale alla fine viene tolta


\begin{fullpaper}
    \begin{table}
    \begin{tabularx}{0.9\textwidth}{lXXXXXXXX}
    \toprule{}\\
    & U & & C & & A & & G\\
    \midrule{}\\
    \multirow{4}{*}{U}
    & UUU & \multirow{2}{*}{Phe} & UUC & \multirow{4}{*}{Ser} & UAU & \multirow{2}{*}{Tyr} & UGU &\multirow{2}{*}{Cys}\\
    & UUC & & UCC & & UAC & & UGU \\
    & UUA & \multirow{2}{*}{Leu} & UCA & & UAA & \multirow{2}{*}{Stop} & UGA & Stop\\
    & UUG & & UCG & & UAG & & UGG & Trp\\
    \midrule{}\\
    \multirow{4}{*}{C}
    & CUU & \multirow{4}{*}{Leu} & CCU & \multirow{4}{*}{Pro} & CAU & \multirow{2}{*}{His} & CGU & \multirow{4}{*}{Arg} \\
    & CUC & & CCC & & CAC & & CGC & \\
    & CUA & & CCA & & CAA & \multirow{2}{*}{Gln} & CGA & \\
    & CUG & & CCG & & CAG & & CGG & \\
    \midrule{}\\
    \multirow{4}{*}{A}
    & AUU & \multirow{3}{*}{Ile} & ACU & \multirow{4}{*}{Thr} & AAU & \multirow{2}{*}{Asp} & AGU & \multirow{2}{*}{Ser} \\
    & AUC & & ACC & & AAC & & AGC & \\
    & AUA & & ACA & & AAA & \multirow{2}{*}{Lys} & AGA & \multirow{2}{*}{Arg} \\ 
    & AUG & Met e Inizio & ACG & & AAG & & AGG & \\
    \midrule{}\\
    \multirow{4}{*}{G}
    & GUU & \multirow{4}{*}{Val} & GCU & \multirow{4}{*}{Ala} & GAU & \multirow{2}{*}{Asp} & GGU & \multirow{4}{*}{Gly}\\
    & GUC & & GCC & & GAC & & GGC & \\
    & GUA & & GCA & & GAA & \multirow{2}{*}{Glu} & GGA & \\
    & GUG & & GCG & & GAG & & GGG & \\
    \bottomrule{}
    \end{tabularx}
    \caption{Codifica dei codoni}
    \end{table}
    \end{fullpaper}

\subsection{Struttura del tRNA}

Il tRNA ha una struttura secondaria a quadrifoglio. Il tRNA contiene 70--80 nucleotidi e vi è un alto contenuto di basi modificate.

\automarginpicture*{Struttura del tRNA} %7

\autofullpicture*{Basi modificate del tRNA} %8

Il tRNA ha una struttura terziaria molto complessa, come per le proteine è più conservata della struttura primaria. Hanno una struttura compatta e una forma a L. Le braccia sono strette, il che va bene perché durante la sintesi proteica, tre molecole di tRNA sono legate tra di loro.

La struttura è resa stabile da nove interazioni. Un braccio è composto dallo stelo accettore e dallo stelo T, l'altro braccio dallo stelo D e dallo stelo dell'anticodone.

42 basi su 76 sono in doppia elica, 71 basi sono coinvolte in interazioni di impilamento.

I ribosomi non possono distinguere se al tRNA è legato un amminoacido corretto, perciò una traduzione accurata richiede due eventi di riconoscimento ugualmente importanti.
\begin{itemize}
\item L'amminoacido corretto è legato covalentemente al tRNA ad opera di una amminoacil-tRNA sintetasi
\item Il corretto amminoacil-tRNA si appaia ad un codone dell'mRNA associato a un ribosoma
\end{itemize}

Una amminoacil-tRNA sintetasi specifica per un dato amminoacido lega l'amminoacido specifico al residuo 3' terminale del ribosio del corrispondente tRNA, formando un legame amminoacido-tRNA. L'amminoacilazione avviene mediante due reazioni successive, catalizzate dallo stesso enzima.

\vfill
\pagebreak

\autoherepicture{0.7} %9

L'amminoacil-AMP reagisce con il tRNA. Può legarsi all'ossidrile in posizione 2' o in posizione 5' dell'adenosina terminale di tRNA.

\autoherepicture{0.7} %9

Il legame peptidico, a partire da amminoacil-tRNA, è termodinamicamente favorito.

La polimerizzazione degli amminoacidi viene raggiunta per eliminazione di una molecola d'acqua tra il gruppo carbossilico di un amminoacido e il gruppo amminico del successivo. Nelle cellule, il legame viene creato nei ribosomi, catalizzato da enzimi. La reazione comunque è sfavorita e richiede energia per poter andare avanti.

Si vede che quello che entra effettivamente nel ribosoma è il amminoacil-tRNA.

Analizzando la struttura della amminoacil-tRNA sintetasi, si vede che ha un dominio, un'ansa, che riconosce l'anticodone del tRNA giusto, e quando si lega, può attivare la reazione per quel specifico amminoacido che corrisponde al codone del mRNA a cui si lega l'anticodone del tRNA.

Ci sono due classi di tRNA-sintetasi, ma operano allo stesso modo, ovvero tramite il riconoscimento della parte dell'anticodone e tramite l'accoppiamento con l'amminoacido specifico.

\clearpage

Le differenze sono:
\begin{itemize}
\item Riconoscimento dei siti diversi nel tRNA
\item Il braccio terminale CCA adotta diverse conformazioni
\item Le sintesi di classe I legano gli amminoacidi in posizione 2', mentre le sintetasi di classe II legano gli amminoacidi in posizione 3'
\end{itemize}

I ribosomi sono ribonucleoproteine e sono composti da varie subunità contenenti l'RNA ribosomiale.

L'mRNA scorre nella parte bassa del ribosoma. Nel ribosoma ci sono tre siti, ovvero E (Exit), P (Peptidico) ed A (Amminoacidico). Nel ribosoma può entrare qualunque tipologia di dd-tRNA; entrano da A ed escono da E. Quando entra quello giusto, ovvero quello che lega il codone di inizio del mRNA, inizia la sintesi. Il tRNA sta su P e aspetta che arrivi l'amminoacil-tRNA giusto in A. Quando questo arriva, si forma il primo legame peptidico.

A questo punto l'mRNA scorre e fa traslocare il primo tRNA su E e il secondo su P.
Il tRNA ha altre proteine di accompagnamento dette EF-Tu, che non mascherano il codone

\autofullpicture*{Proteine EF-Tu}

\autoherepicture{0.78} %10

Questa reazione scaturisce una variazione conformazionale di EF-G, che spinge meccanicamente il mRNA e lo fa traslocare. Questo consuma energia, che viene fornita da GTP
\[
\text{GTP } \ce{->} \text{ GDP } + \text{P}
\]

Dopo la traslocazione, tutto questo si ripete fino alla fine della sintesi.

La fine della sintesi arriva quando il codone di stop viene riconosciuto da delle proteine che dissociano mRNA dal ribosoma. I fattori di rilascio RF riconoscono i codoni di stop, si inseriscono nel sito A e fanno uscire tutto dal ribosoma. mRNA viene riutilizzato, però viene cambiato dopo una serie do volte, in quanto potrebbe generare errori.

Si possono determinare i livelli di espressione genica in termini di quantità di mRNA corrispondente mediante l'utilizzo di DNA microarray, o gene-chip.

Il principio su cui si basa la tecnica è quello di ibridizzare l'mRNA con oligonucleotidi complementari e di rilevare la presenza e quantità mediante fluorescenza.

\autofullpicture*{Sintesi nei ribosomi}

\autofullpicture*{Terminazione della sintesi}
\part{Introduzione}
\chapterpicture{header_02}
\chapter{Farmaci}

Cos'è un farmaco?
La risposta dipende a chi risponde alla domanda; in ogni caso una definizione generica è:
\begin{quoting}
Il farmaco è una sostanza o composizione che ha un'azione curativa, profilattica o diagnostica delle malattie degli animali o degli esseri umani.
Il farmaco viene somministrato all'uomo o all'animale allo scopo di ripristinare o modificare funzioni organiche
\end{quoting}
In alternativa, è anche possibile descrivere il farmaco come:
\begin{quoting}
Una sostanza che interagisce con un sistema biologico e che produce una risposta biologica
\end{quoting}

L'organismo sul quale il farmaco possiede due stati, ovvero lo stato perturbato (o malato) e lo stato stazionario (o sano)

Un farmaco non è né buono né cattivo. Si prenda esempio la penicillina e la tachipirina; essi sono farmaci ``buoni’’, nel senso che sono utilizzati come antibiotico e come analgesico rispettivamente, tuttavia la penicillina può essere vista come ``cattivo’’, in quanto va a uccidere i batteri presenti.
Un altro esempio è la morfina, che è un antidolorifico che viene utilizzato nei malati terminali, può essere visto come buono, per via della sua azione analgesica, tuttavia se assunto a dosi troppo elevate causa la morte per blocco respiratorio.

Una sostanza simile alla morfina è l'eroina, che dal punto di vista legale è una sostanza ``cattiva’’. Dal punto di vista chimico-farmaceutico, l'eroina è una sostanza che interagisce con un sistema biologico. L'eroina si può trovare anche come farmaco, con il nome di \emph{diamorfina}, per il trattamento dei pazienti terminali. Inizialmente, l'eroina era vista come un farmaco, tuttavia con il tempo si sono visti gli effetti collaterali che comporta l'utilizzo di questa sostanza.

Alcune sostanze rientrano dentro la definizione di farmaci, come la caffeina e la nicotina, rispettivamente come sostanze eccitanti e rilassanti. Altre sostanze, come i veleni, rientrano all'interno della definizione di farmaci.

Il farmaco, per essere sicuro, deve essere dosato attentamente. Tutti i farmaci sono potenzialmente dei veleni. La dose per la quale il farmaco è sicuro è definita come \emph{indice terapeutico} ed è definito come l'intervallo tra la dose minima che produce un effetto tossico e la dose che produce il massimo effetto terapeutico.
Nessun farmaco è totalmente sicuro.

Un altro aspetto importante è quello della tossicità selettiva, in quanto un farmaco può fare diversi effetti a seconda dell'organismo, come per la penicillina che è molto tossica per i batteri, mentre le cellule umane non sono intaccate.
Questo è un aspetto molto importante e un grande vantaggio per definire la tossicità selettiva di un farmaco.

Il farmaco è una preparazione (o insieme di sostanze) che contiene il principio attivo. Il principio attivo si trova in un ambiente pieno di eccipienti, che hanno altre funzioni. Gli eccipienti sono decisi in base alla formulazione. La parte di biotecnologia di formulazione fa parte dello sviluppo del farmaco.

\marginbox*{È preferibile assumere farmaci per via orale, per mezzo di pastiglie, rispetto ad altri modi, come l'iniezione.\\ Le caratteristiche che determinano il modo di assunzione si chiamano \emph{compliance} del farmaco.}

Un farmaco può essere anche visto dal punto di vista industriale; i tre spazi sono tre blocchi fissati nell'industria farmaceutica. Questi tre blocchi devono comunicare tra di loro per sviluppare un farmaco.

\fullpicture*{1_001}{Spazio chimico spazio biologico e spazio fisio-patologico}

\paragraph{Spazio fisio-patologico}
Questo spazio rappresenta tutte le patologie che l'industria farmaceutica ha l'interesse di agire. È uno spazio in continua crescita. Non è competenza del chimico farmaceutico decidere all'interno dello stato fisio-patologico. Questo stato è puramente di competenza medica.

C'è una soglia che permette di determinare lo stato patologico, per essere in seguito trattato. Il medico decide anche la soglia dello stato patologico.

\paragraph{Spazio biologico}
Questo spazio rappresenta le macchine molecolari, che possono essere una proteina, un acido nucleico o altro, che hanno una funzione nello stato sano e che può essere alterato nello stato biologico.

Questo spazio è competenza dei biologi, che vanno a determinare la struttura e la funzione. La struttura è competenza anche del chimico biologico. Il biologo va a scoprire le funzioni delle macchine molecolari.

Il come è alterata la funzione delle macchine biologiche riguarda sempre lo spazio biologico.

\paragraph{Spazio chimico}
Lo spazio chimico si occupa di tutte le potenziali molecole, di cui non si sa ancora se hanno una proprietà biologica o che non sono ancora state sintetizzate. Questo perché le molecole vengono sintetizzate in grande numero, nei laboratori di ricerca.

Lo schema indica che per una patologia, che è determinata dal malfunzionamento di una macchina molecolare all'interno dello spazio biologico.
A monte, esiste un farmaco che, dopo tanto lavoro per arrivare a identificarlo, è quello che va a interagire con i bersagli molecolari. Ripristinando i bersagli, si va a risolvere lo stato patologico dato da una patologia.

I target (o macchine molecolari) possono essere di diversi tipi:
\begin{itemize}
\item \textit{On-target}: sono le macchine molecolari coinvolte nella malattia. Vanno ripristinate alla loro funzione biologica per mezzo del farmaco. Sono le macchine molecolari sulle quali il farmaco presenta un'attività
\item \textit{Secondary target}: sono le macchine molecolari che vengono coinvolte nell'azione del farmaco. Non sono importanti come l'on-target, però sono sempre coinvolte nella patologia
\item \textit{Off-target}: La molecola, oltre a interagire con le macchine appropriate colpisce anche una macchina che non è coinvolta nella patologia, ovvero l'off-target. L'azione del farmaco verso l'off-target determina l'effetto collaterale del farmaco.
\end{itemize}

Il bilanciamento tra l'effetto sull'on-target e sull'off-target del farmaco è quello che da l'efficacia del farmaco

Tempo fa, il rapporto tra gli spazi era 1:1:1, ovvero si assumeva che una molecola interagiva con un bersaglio che fosse relativo ad una patologia. Invece adesso si è nell'era del multi-target, quindi una molecola va a interagire con più target, anche in modo benefico.
Questo apre molte possibilità, in quanto con un singolo farmaco si può andare a colpire più target, nel modo desiderato.

Non è detto che basti un farmaco per ripristinare lo stato sano da una patologia, quindi si utilizzano delle combinazioni di farmaci per risolvere uno stato patologico.
Un farmaco multi-target è differente da una combinazione di farmaci, in quanto il farmaco multi-target presenta solo un principio attivo che va ad agire in modo benefico verso tutti i target, mentre una combinazione di farmaci usa lo stesso principio, però è formata da una miscela di principi attivi.
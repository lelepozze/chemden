\chapterpicture{header_03}
\chapter{Chimica Farmaceutica}

La chimica farmaceutica è una disciplina interdisciplinare. Ad esempio,
le piccole molecole vengono sintetizzate da un chimico organico. Servono
molte competenze che una persona non può avere.
Ad esempio, durante il covid, è stato fatto un virtual screening di
tutte le molecole possibili, per estrarre una piccola quantità di
molecole che potenzialmente possono diventare potenziali farmaci. Il
virtual screening ha prodotto una quarantina di molecole. Da queste,
sono state scremate in base alle interazioni con il target.

Sono stati approfonditi alcune molecole tramite tecniche
spettroscopiche. Quelle molecole più promettenti dal punto di vista del
binding sono state selezionate. Si è arrivati a tre molecole.
Poi le molecole potenziali sono state usati per fare i test cellulari.
Questo è un primo round dello sviluppo accademico. Dopo, con i dati in
mano, si cambiano i gruppi funzionali, si migliora il farmaco e si riprova a sintetizzarlo.

\marginbox*{
    Per target si intende la macchina molecolare, che può essere un enzima, un ormone o un acido nucleico.
}

Il chimico farmaceutico deve saper fare tante cose, è specializzato
nella chimica farmaceutica, ma deve interagire con altre figure, che sono:
\begin{itemize}
    \item Il biologo strutturale che studia la struttura e
    la funzione del target.
    \item Il biochimico che studia le interazioni per
    determinare se il sistema studiato è importante per le patologie
    \item L'esperto clinico non è proprio un chimico. È il primo interlocutore per
    determinare se il lavoro fatto può andare avanti 
    \item Il farmacologo si
    occupa della tossicologia dei potenziali farmaci
    \item Il biologo si occupa
    degli studi in cellula
    \item Il biologo molecolare/chimico molecolare fa
    la struttura delle molecole
    \item Il chimico computazionale fa studi in
    silico
\end{itemize}

La chimica farmaceutica è una scienza precisa e che ha bisogno di tanti
dati. Non si può essere poco precisi per capire come interagisce la
molecola.

Le figure devono raccogliere tanti dati per convincere le altre figure
ad andare avanti a studiare il farmaco. Questo si fa anche a partire
dallo step iniziale. Date tre molecole potenziali è necessario
convincere il virologo per investire tempo e risorse per testare i
potenziali farmaci, e successivamente si va a valutare la tossicità, con
un altra figura di riferimento. Per questo è necessario avere molti
dati. È molto comune che un farmaco venga respinto, piuttosto che venga
approvato.

La chimica farmaceutica è un arte. È un mix di conoscenza, di esperienza
sperimentale (accumulata nel tempo).
La chimica farmaceutica richiede di essere creativi, a volte provando e
a volte avendo l'intuizione, giocando con le molecole.
Serve anche fortuna e casualità. Come ad esempio la penicillina

Il chimico farmaceutico viene associato al chimico organico sintetico;
non è proprio vero. Il chimico farmaceutico deve comunque parlare con la
parte del biologo e la parte cellulare (della chimica). Per fare questo
serve pazienza perché le persone hanno competenze diverse e linguaggi
diversi.

La definizione IUPAC recita:
\begin{quoting}
Medicinal chemistry is a chemistry-based discipline, concerned with the invention, discovery, design, identification and preparation of
biologically active compounds, the study of their metabolism, the
interpretation of their mode of action at the molecular level.
\end{quoting}
\begin{quoting}
Emphasis is put on drugs, but the interests of the medicinal chemist are
not restricted to drugs but include bioactive compounds in general. Med
Chem is also concerned with the study, identification, and synthesis of
metabolic products of these drugs and related compounds
\end{quoting}

Quindi il centro di tutto è una molecola che abbia un'attività
biologica, per riparare lo stato patologico.

La chimica farmaceutica si occupa dei metaboliti della molecola
biologicamente attiva. Si occupa dello studio, identificazione e sintesi
dei metaboliti del composto. Non si studia solo il principio attivo
fuori dall'organismo, ma anche dopo l'assunzione.

Questo porta anche alla produzione di pro-farmaci. I profarmaci non
hanno attività biologica in sé, ma diventano farmaci quando raggiungono
il target, in una forma diversa, ad esempio con un cambio di pH, il
farmaco perde un protone e diventa biologicamente attivo.

I metaboliti potrebbero essere tossici, quindi bisogna lavorare sulla
molecola in modo tale che non si rompa in certi punti e non consenta la
produzione del metabolita tossico.

{\color{Primary} \itshape{} \bfseries{} Cosa succede ad una molecola biologicamente attiva quando viene somministrata?}\\
La molecola biologicamente attiva può essere somministrata in diversi modi, ma quello preferibile è per via orale, perché è più accessibile e comodo per il paziente. Il biotecnologo comunque sceglierà la formulazione migliore.

Il processo farmaceutico coinvolge la dissoluzione e l'entrata del
farmaco nel corpo. Il farmaco dovrà essere biodisponibile.

Da qui in poi inizia la fase farmacologica, che mette insieme il primo e
il secondo processo. Si distinguono due momenti: la fase farmacocinetica e la fase farmacodinamica

\begin{figure}
    \begin{tikzpicture}
    \sffamily
    \node at (0,0) (a) {Drug Administration};
    \node [below=0.5 of a] (b) {Disintegration from dosage form and Dissolution};
    \node [below=0.5 of b] (c) {Absorption};
    \node [below=0.5 of c] (d) {Distribution};
    \node [below=0.5 of d, fill=white, inner sep=1pt] (e) {};
    \node [below=0.5 of e] (f) {Pharmacological Effect};
    \node [below=0.5 of f] (g) {Therapeutic Effect};
    \node [above right=0.30 of d] (h) {Metabolism};
    \node [right=0.30 of d] (j) {Excretion};
    \node [below=0.20 of f] (3) {};
    \node [left=6 of e] (3a) {};
    \node [right=6 of e] (3b) {};
    \draw (3a) -- node[fill=white] (e1) {Site of Action} (3b);
    \draw[-stealth, thick, red] (a) -- (b);
    \draw[-stealth, thick, red] (b) -- (c);
    \draw[-stealth, thick, red] (c) -- (d);
    \draw[-stealth, thick, red] (d) -- (e1);
    \draw[-stealth, thick, red] (e1) -- (f);
    \draw[-stealth, thick, red] (f) -- (g);
    \draw[stealth-stealth, thick, red] (d) -- (h);
    \draw[-stealth, thick, red] (d) -- (j);
    \draw[-stealth, thick, red] (h) -- (j);
    \node [below=0.20 of b] (1) {};
    \node [left=6 of 1] (1a) {};
    \node [right=6 of 1] (1b) {};
    \draw (1a) -- (1b);
    \node [below=0.20 of f] (3) {};
    \node [left=6 of 3] (3a) {};
    \node [right=6 of 3] (3b) {};
    \draw (3a) -- (3b);
    \node [left=2.5 of a] (bb) {};
    \node [anchor=right,below=0.1 of bb] {\color{Secondary} Pharmaceutical Process};
    \node [anchor=right,below=2 of bb] {\color{Secondary} Pharmacokinetic Process};
    \node [anchor=right,below=4 of bb] {\color{Secondary} Pharmacodynamic Process};
    \node [anchor=right,below=5.5 of bb] {\color{Secondary} Therapeutical Process};
    \end{tikzpicture}
    \caption{Schema delle fasi di assunzione di un farmaco}
\end{figure}
    
\paragraph{Fase farmacocinetica}

In base alle proprietà farmacocinetiche si avranno una serie di processi
dell'organismo. I processi sono effettuati dall'organismo verso il
farmaco, o meglio ancora, al principio attivo che è stato reso
disponibile all'organismo.

Si avrà quindi una fase di assorbimento del principio attivo, che poi
verrà distribuito all'interno dell'organismo. In seguito il farmaco
verrà metabolizzato; i metaboliti verranno poi eliminati.

Questa fase viene definita ADME, ovvero \textbf{A}ssorbimento, \textbf{D}istribuzione,
\textbf{M}etabolismo ed \textbf{E}liminazione.

Le fasi di metabolismo e di eliminazione sono fasi che non fanno
continuare l'attività del farmaco.
Può essere che il farmaco venga distribuito bene, però viene tutto
metabolizzato. Quindi non si ha un effetto.

Nella distribuzione del farmaco, si arriva a raggiungere il sito/i siti
di attivazione. Dopo questo step, si passa alla fase di farmacodinamica.

\paragraph{Fase farmacodinamica}

In questa fase, il farmaco effettua un azione sull'organismo. In questa
fase avviene l'effetto farmacologico del farmaco, sull'organismo.
Se il farmaco funziona, si avrà l'effetto terapeutico desiderato.

Queste fasi servono per capire come sviluppare il farmaco. Se il farmaco
non è disponibile (quindi non è solubile) non funzionerà bene, anche se
la molecola funziona bene.
Si può anche discriminare una molecola a monte, se si vede che non
presenta i requisiti giusti.

Il principio attivo è la parte biologicamente attiva del farmaco. Il
farmaco è formato anche dagli eccipienti, questi sono essenziali
all'assunzione del farmaco.

La combinazione tra principi attivi ed eccipienti è competenza del
tecnologo, che studia e sceglie la formulazione migliore del farmaco.
Molti farmaci, più che sviluppati, sono stati scoperti, come la
penicillina. C'è comunque l'intuizione dello scienziato.

Un chimico farmaceutico si occupa di entrambe le cose, sia della
scoperta, sia dello sviluppo razionale.
Questo anche è il motivo del perché molti farmaci sono di origine
naturale, o naturali.

Oltre al lavoro della natura, c'è anche il lavoro dei chimici organici
di sintesi. Negli anni sono state create delle collezioni di sostanze
sintetiche analoghe, che si chiamano librerie.

Ogni laboratorio, o ogni industria farmaceutica ha le proprie librerie.
Le librerie non sono fatte da farmaci, ma fatte da piccole molecole, che
non hanno una vera e propria attività biologica. Tutto fa parte di un
background per fare screening su bersagli diversi.
Le librerie variano da accademia a industrie multinazionali.
Possono contenere molecole di origine naturale o di origine sintetica.

Le molecole possono interagire con i bersagli molecolari che
appartengono dello spazio biologico, che sono più di competenza
biologica.
Grazie a questa interazione si avrà un effetto terapeutico.
Il numero di bersagli è in continua crescita, quindi si ha la
possibilità per intervenire in modo terapeutico su bersagli che non si
conoscono ora, ma si conosceranno nel futuro

Lo screening può essere randomico o focalizzato; in quello randomico, lo
screening viene fatto su un numero enorme di molecole, mentre quello
focalizzato va a guardare l'interazione che ci deve essere con il
bersaglio e quindi va a scegliere le migliori molecole.

Da uno screening non si ricavano direttamente i farmaci, ma innanzitutto
si ricavano dei composti detti ``Hit'', che successivamente verranno
studiati e, se passeranno anche questa fase, verranno chiamati ``Lead''.
Dal Lead compound, si ricava il candidato farmaco, attraverso una serie
di step, si ricava il candidato farmaco.

Quando si deve sintetizzare il farmaco, esso deve avere buone
proprietà.
\part{Scoperta e sviluppo del farmaco}

\chapterpicture{header_04}
\chapter{Sviluppo del farmaco}

La scoperta di nuovi farmaci è un
percorso lungo, costoso e rischioso. Queste sono le tre cose che
è necessario sapere prima di iniziare lo sviluppo.

La pre-discovery è un processo diviso in due parti, ovvero drug-discovery e preclinical-phase

In media questa fase va dai tre ai sei anni, a livello industriale. A
livello accademico, invece, questa fase può durare anche di più.

Nella prima fase, si fa uno screening su 5000--10000 molecole, nella
seconda fase invece, si fa uno screening si arriva a 250 molecole.
Queste 250 molecole sono quelle per il quale vale la pena continuare a
analizzare.

\fullpicture*{3_001}{Fasi di sviluppo di un farmaco}

\section{Fase clinica}

Nella fase clinica, si diminuiscono ancora le molecole. Se si arriva prima, il farmaco deve
essere approvato per poterlo utilizzare. Dopo l'approvazione, il farmaco
comunque continua a essere controllato, nella fase di follow-up.
In seguito, si ha una produzione di scala, susseguita dalla messa in
vendita del farmaco.

Ogni fase può causare il blocco di un potenziale farmaco

\marginbox*{
In media si spendono dai 500 milioni a 1 miliardo di dollari per la
produzione di un singolo farmaco. La casa farmaceutica deve già avere
questi soldi per poter iniziare.
}

Si vede anche che i costi sono aumentati nel corso degli anni, e che
però i nuovi farmaci approvati restano costanti negli anni. Questo
avviene perché, con il passare del tempo, si è diventati più selettivi,
quindi può succedere che un farmaco non venga approvato, anche se per
questo farmaco sono stati spesi molti soldi.

\begin{figure}
\begin{tikzpicture}
\sffamily
\node at (0,0) (a) {Identificazione del bersaglio farmacologico};
\node [below=0.5 of a] (b) {Validazione del bersaglio};
\node [below=0.5 of b] (c) {Identificazione del lead-compound};
\node [below=0.5 of c] (d) {Ottimizzazione del lead-compound};
\node [below=0.5 of d,align=center] (e) {Sviluppo preclinico (tossicità acuta,\\ cronica in animali, teratogenicità e mutagenicità)};
\node [below=0.5 of e] (f) {Medoti di sintesi e formulazione};
\node [below=0.5 of f] (g) {IND: investigational new drug};
\node [below=0.5 of g] (h) {Sviluppo clinico};
\node [below=0.5 of h] (i) {NDA: new drug applicazion};
\draw[-stealth,thick,red] (a) -- (b);
\draw[-stealth,thick,red] (b) -- (c);
\draw[-stealth,thick,red] (c) -- (d);
\draw[-stealth,thick,red] (d) -- (e);
\draw[-stealth,thick,red] (e) -- (f);
\draw[-stealth,thick,red] (f) -- (g);
\draw[-stealth,thick,red] (g) -- (h);
\draw[-stealth,thick,red] (h) -- (i);

\node [below=0.20 of b] (1) {};
\node [left=6 of 1] (1a) {};
\node [right=6 of 1] (1b) {};
\draw (1a) -- (1b);

\node [below=0.20 of d] (2) {};
\node [left=6 of 2] (2a) {};
\node [right=6 of 2] (2b) {};
\draw (2a) -- (2b);

\node [below=0.20 of g] (3) {};
\node [left=6 of 3] (3a) {};
\node [right=6 of 3] (3b) {};
\draw (3a) -- (3b);

\node [left=2.5 of a] (bb) {};
\node [anchor=right,below=0.1 of bb,align=center] {\color{Secondary} Ricerca preliminare\\(6-12 mesi)};
\node [anchor=right,below=2 of bb,align=center] {\color{Secondary} Scoperta del Lead\\(18-24 mesi)};
\node [anchor=right,below=4 of bb,align=center] {\color{Secondary} Studi su animali\\(12-18 mesi)};
\node [anchor=right,below=5.5 of bb,align=center] {\color{Secondary} Sviluppo chimico\\(6-12 mesi)};
\node [anchor=right,below=7.5 of bb,align=center] {\color{Secondary} Fase clinica\\(3-5 anni)};

\end{tikzpicture}
\caption{Fasi dello sviluppo di un farmaco.}
\end{figure}

Può sembrare strano che le industrie farmaceutiche spendano molti soldi
nella struttura cristallografica delle macchine molecolari. Però se si
tiene conto che l'industria farmaceutica può lavorarci in esclusiva,
per via del brevetto, si capisce che l'interesse è molto elevato. Vale la
pena investire soldi in step costosi, che però possono portare un grosso
introito.

L'ultima parte del percorso di un farmaco è quello di essere approvato.
In seguito, si effettua la produzione in massa del farmaco.

Ogni anno vengono pubblicate le statistiche e le classifiche dei
farmaci. Grafico FDA

Nel corso degli anni vengono approvati dei farmaci. Ci sono due
categorie:
\begin{itemize}
\item \emph{NME (New Molecular Entities):} Farmaci costituiti da una
piccola molecola. Si ha quindi una nuova identità farmaceutica.
\item \emph{BLA:} Farmaci biologici/biotecnologici. Questa tipologia di farmaci è
innovativa e molto convenienti. Sono farmaci di nuova generazione.
Questo è un approccio innovativo, in quanto non si utilizzano più delle
piccole molecole, ma si utilizzano delle macromolecole.
\end{itemize}

Le BLA che sono stati approvati recentemente, sono stati studiati da
almeno 10 anni. Il grafico rispecchia gli anni di ricerca precedenti

Il ritorno economico di un farmaco si vede dopo la sua approvazione e
messa in vendita.

I farmaci non aumentano esponenzialmente; non c'è una ragione vera e
propria, l'unica cosa che si può dedurre è che è necessario continuare a
fare ricerca.

\fullpicture*{3_003}{Tipologie di farmaci approvati}

A prescindere dall'importanza della malattia, l'industria guarda il
ritorno economico del farmaco, come si vede nel secondo grafico, che
mostra le percentuali divise in aree terapeutiche, dei nuovi farmaci
messi in commercio. L'area oncologica è quella con più farmaci
sviluppati, in quanto il ritorno economico è molto elevato.

Tante persone, nel corso della loro vita, sviluppano un cancro, quindi
l'area oncologica nel futuro potrebbe fare tanti soldi, al contrario di
altre aree. Ad esempio, le malattie rare non vengono molto ricercate
perché il ritorno economico non è elevato come quello di altre aree.

Questo grafico, come l'altro, mostra lo sviluppo nei dieci anni
precedenti, non mostra lo sviluppo attuale.

Un altro esempio è quello delle malattie infettive, come il covid. Negli
anni precedenti, la ricerca in questo campo è stata bassa perché non si
avvertiva la necessità di svilupparle, né c'erano le indicazioni per un
ritorno economico elevato.

Una volta che il farmaco viene approvato, il farmaco viene brevettato

Grafico dei blockbuster.

Solitamente, le case farmaceutiche possiedono un farmaco blockbuster,
ovvero un farmaco che ha fatto guadagnare alla casa almeno un miliardo
di dollari, nei primi vent'anni. Il limite di tempo è imposto dal fatto
che la casa farmaceutica, di solito, brevetta il farmaco e il brevetto
ha una durata di vent'anni da quando è stato depositato.

Quando una casa farmaceutica brevetta un blockbuster, riesce a
guadagnare tutto quello che in precedenza è stato speso nello sviluppo.
Questo è anche un motivo dell'inglobamento da parte delle aziende più
grandi, di quelle più piccole, in quanto è necessario molto denaro.

Un farmaco blockbuster è un farmaco che può essere utilizzato da tutti,
in modo tale da consentire la maggiore vendita e quindi il maggior
incasso.

Anche se una azienda brevetta un farmaco, è possibile fare ricerca su
questo. I competitor possono comunque mettere in vendita un prodotto
molto simile al farmaco brevettato ed in ogni caso possono utilizzarlo
dopo la caduta del brevetto. Si possono anche brevettare delle classi di
farmaci, non solo singole molecole.

Quando si brevetta una classe di molecole, a mano a mano che si vedono
risultati migliori dei farmaci, l'azienda spinge per mettere in
commercio il farmaco, in quanto è necessario avere dei guadagni per più
tempo possibile.

\subsection{Liditor}



Un esempio di farmaco blockbuster è il Liditor, che è commercializzato
da Pfizer. Questo farmaco serve per abbassare l'LDL, e quindi riduce il
pericolo di infarti o di attacchi cardiaci. Questo farmaco viene dato a
titolo preventivo, non serve per curare una malattia.

La struttura del Liditor presenta diversi gruppi \ce{OH}, che quindi
consentono alla molecola di sciogliersi meglio in acqua.

\marginpicture*{3_002}{Liditor}

Questo richiude dei concetti fondamentali per i farmaci, ovvero che un
farmaco deve parzialmente sciogliersi in acqua, ma deve anche essere
parzialmente olipofilo

Questo perché le cellule sono compartimenti formati da acqua e da parti
idrofobe (come le membrane).

Si usa il coefficiente di ripartizione per determinare quanto la
molecola si ripartisce in due fasi.

Solitamente, una piccola molecola in laboratorio viene sciolga in DMSO,
quindi i test non rispecchiano la situazione biologica. Si scioglie
prima in DMSO e poi in acqua; però la quantità aggiunta di DMSO è
talmente piccola, che la concentrazione della molecola potrebbe essere
sovrastimata.

Il coefficiente di ripartizione è importante, perché può determinare
l'efficacia della molecola, quindi è necessario modificarla per renderla
più o meno solubile

Un'altra caratteristica del Liditor è che possiede un atomo fluoro. Da
un punto chimico non cambia troppo, però da un punto di vista di
polarità cambia. L'atomo di fluoro consente di interagire con il target,
nel modo corretto.

Dopo che il brevetto è scaduto, si vede che le entrate di pfizer
derivanti da questo farmaco sono scese. Questo per via della
competizione di altre case farmaceutiche, che possono riprodurre una
molecola come farmaco generico.

I farmaci approvati, nel grafico precedente, sono raccolti in un grafico
a torta 1-37.

Nella parte grigia del grafico, sono presenti le piccole molecole, che
sono divise in:
\begin{itemize}
\item Piccole molecole
\item Radiotraccianti
\item Piccoli peptidi (peptidomimetici)
\end{itemize}

I peptidomimetici sono considerate molecole che mimano il substrato di
un enzima attivandolo o disattivandolo, a seconda della necessità. Si va
a mimare una cosa che è già esistente in natura, quindi si modificano
delle strutture già esistenti in natura.
I peptidomimetici sono di origine sintetica.

All'interno delle piccole molecole, sono presenti anche gli
oligonucleotidi, che sono piccole molecole di RNA (siRNA), che vanno a
silenziare dei geni (delle espressioni geniche).
Questo serve per sopprimere una parte di espressione genica di una
proteina non desiderata.
I siRNA sono piccoli, non sono troppo lunghe. Per questo rientrano nella
categoria delle singole molecole.

Dall'altra parte, si hanno i farmaci biologici, che non sono altro che
proteine, o macchine molecolari. Le piccole molecole sono la gran parte
delle nuove molecole. Le proteine sono uno strumento per andare ad
interagire con una macchina molecolare a sua volta.
Queste proteine sono prodotte attraverso la tecnica del DNA
ricombinante; sono prodotte in laboratorio.
I più famosi sono gli anticorpi monoclonali. Esistono anche degli
anticorpi bispecifici (che possono riconoscere due target), e anche dei
coniugati anticorpi-proteina (ABC).

Gli ABC si usano se si ha una piccola molecola che ha una buona azione,
però non è selettiva, si può aumentare la selettività tramite l'uso di
un anticorpo. Come approccio può dare speranza a piccole molecole che
sono state abbandonate nel corso dello sviluppo del farmaco

Ci possono essere diversi tipi di farmaci biologici, che possono essere
utilizzati per molti scopi

Le NWE possono essere rappresentate dall'aspirina, che è una piccola
molecola, infatti ha un peso molecolare pari a 180 Da. Non è immunogenica, ovvero non
produce una risposta immunitaria. È una struttura chimicamente stabile,
non soggetta a idrolisi o altre reazioni.

I farmaci biologici, invece, sono macromolecole che pesano 150 000 Da. Ha
un numero di atomi elevato. È immunogenica, in quanto si introducono
delle proteine nel corpo, che potrebbe non essere non tollerata. È
chimicamente instabile, in quanto è molto complessa, e presenta una
tendenza a idrolizzare.

Queste proteine sono prodotte con delle tecnologie automatizzate. Però
per proteine nuove bisogna mettere a punto un nuovo sistema, in quanto
le proteine sono molto dipendenti dall'ambiente e dai processi. Alcune
tecniche/processi rischiano di denaturare le proteine, quindi si ha un
instabilità chimica.

Per sopperire all'instabilità dei farmaci biologici, si utilizzano dei
sistemi che stabilizzano e proteggono la molecola, almeno fino
all'assorbimento e alla distribuzione.

Questi farmaci solitamente vengono distribuiti dalle farmacie
ospedaliere, in quanto devono essere conservate in un modo diverso, ad
esempio, si utilizza un frigo, in quanto le proteine hanno una certa
stabilità ad una certa temperatura. Una temperatura diversa rischia di
denaturare la proteina o di farla precipitare.
Questo è un campo in continua evoluzione.

Tra i farmaci farmaceutici c'è anche l'insulina, che è una proteina
piccola, però è sempre molto più grande dell'aspirina.

L'acido salicilico è un analogo dell'aspirina, più piccolo e meno attivo
dell'acido acetil salicilico, che deve essere modificato per essere più
attivo.

\section{Processo di sviluppo}

Il processo di sviluppo del farmaco è cambiato nel corso del tempo. Si
può stabilire un ``prima'' tra il 1950 e il 1980, e un ``dopo'', a
partire dal 1980.

Il processo vecchio può essere utilizzato anche ai giorni nostri, però è
un approccio superato. I vantaggi che il nuovo approccio ha sono molti
di più e sono anche meno dispendiosi.

La metodologia originaria era sempre basata sulla chimica, quindi basata
dal punto di vista del ligando (piccola molecola) utilizzando un
approccio trial and error. Questo approccio utilizza dei saggi in vivo.

Erano necessari solo un farmacologo e un chimico organico, perché si
facevano delle prove in vivo, su cavie viventi che non sono molto
complicate.

In base a questo si vedeva un effetto, poi si ritornava alla fase di
sviluppo chimico e poi si ritornava alla sperimentazione su cavie.

Questo approccio è molto costoso, però al tempo era l'unico che si
poteva fare.

Lo sviluppo attuale prevede comunque un ciclo di ottimizzazione, però
fatto in modo diverso.

Si può utilizzare questo approccio, in quanto sono stati fatti molti
passi nello spazio biologico, che permette di conoscere il target.

Questo ci permette di cambiare la visione dell'approccio vecchio.

Il nuovo approccio ha permesso lo sviluppo di macromolecole, però si può
usare anche per le molecole più piccole.

Queste macromolecole possono avere delle limitazioni per quanto riguarda
il trasporto, prima di arrivare al bersaglio.

Quando lo incontrano, sono molto attive e molto specifiche. Il farmaco
può essere reso più stabile attraverso la sua formulazione.

Prendendo lo sviluppo del farmaco piroxicam (un antinfiammatorio) si
vede che sono passati 18 anni di sviluppo prima della sua approvazione.
Questo farmaco è stato sviluppato utilizzando il vecchio metodo.

Il farmaco è stato sviluppato per avere un'attività simile ad altri
farmaci già presenti, però con una struttura chimica diversa, per avere
un esclusiva.

Un altro esempio di farmaco è lo ziprasidone, che è un farmaco per la
schizofrenia. L'area terapeutica è nuova. Questo farmaco è stato
sviluppato a ponte del cambio di approccio; inizialmente si è utilizzato
il vecchio approccio, mentre in seguito si è passati a quello nuovo.
Sono stati introdotti i primi saggi in vitro, e si è andati a vedere
quali recettori legava. Si è visto che se una molecola riusciva a legare
due recettori diversi, la sua attività migliora. Quindi, prima di fare
la sintesi, si è andati a vedere se si riusciva ad avere una selettività
per entrambi i siti. In seguito, sono stati fatti dei saggi in vivo.

I saggi in vitro servono per schermare alcuni analoghi rispetto ad
altri.

Lo sviluppo di questo farmaco è una via di mezzo tra il paradigma
vecchio e quello nuovo. Sono stati fatti quindi degli studi in vitro tra
il ligando-recettore.

L'approccio moderno è continuamente sviluppato, per poter migliorare lo
studio di un farmaco.

Si prenda esempio il farmaco Imatinib, che è un antitumorale, che è il
primo esempio di un farmaco basato su un target (target-based).
Il farmaco è stato sviluppato con molti test in vitro. Ad esempio, sono
stati guardati il legame ligando-recettore, la permeabilità, l'uptake,
etc.

Si va a guardare come la molecola si comporta in acqua, ma anche il
comportamento se è presente una membrana. Questi saggi possono essere
fatti in vitro.
In seguito vengono fatti i test cellulari, che però sono sempre detti
test in vitro.

Noi ci concentriamo sul drug-discovery. La fase della scoperta del lead
è molto importante.
La scoperta del lead fa parte della parte della drug discovery, mentre
l'ottimizzazione del lead fa parte dello sviluppo chimico.
In seguito c'è la parte preclinica, dove si fanno studi sugli animali.

Prima di tutto però c'è una fase iniziale di identificazione del
bersaglio farmacologico, che però non è di competenza dei chimici, ma
dei biologi. Comunque si ha un approccio farmaceutico (chimico) nel
vedere l'interazione del bersaglio e quindi la scelta del bersaglio.

La parte successiva è di identificazione del lead. Questa
schematizzazione è molto lasciva, però il lead.
Il lead è definito come un candidato maturo del farmaco, in quanto ci
sono dei candidati precoci del farmaco (hit).

In seguito all'ottimizzazione del lead, si ha lo sviluppo pre-clinico,
che fornisce l'IND. L'IND è la collezione di dati raccolti
dall'industria farmaceutica, che poi servono per continuare lo sviluppo
del farmaco nel minimo dettaglio (fatto dagli enti preposti).
Dopo l'IND, si ottiene un'altra serie di dati, che viene chiamata NDA.

Nella fase di discovery, si arriva ad ottenere un lead compound. Quando
si ha l'identificazione del lead, è necessario ulteriormente ottimizzata
per diventare un candidato farmaco, tramite la fase di ottimizzazione.
In questa fase, vengono effettuati cicli di modifica della struttura per
modificare l'attività del farmaco.

In seguito, c'è lo sviluppo preclinico (studi farmacologici, ovvero studi di
farmacocinetica e farmacodinamica), ma si fa anche
l'ottimizzazione della sintesi industriale (a livello chimico) per
renderlo scalabile e promettente per la produzione su larga scala.

{\color{Primary} \itshape{} \bfseries{} Come si può identificare il lead compound?}\\
Si può fare in modo classico (progettazione razionale). Con questo modo,
si va a migliorare la molecola guardando la struttura. È fatto da un
chimico, che deve avere intuito per capire la struttura corretta.

Si può anche fare attraverso gli HTS (High Throughput Screening), che
possono essere random (si può fare su un ampio numero di identità
chimiche).
Oppure si possono fare degli screening su molecole più focalizzate
(librerie di analoghi) con il quale ci si approccia ad un metodo
razionale.

Il lead compound non è il candidato farmaco, ma deve essere ottimizzato
per diventarlo.
Questo si fa attraverso le relazioni struttura-attività, per capire
quali sono i gruppi funzionali importanti per l'interazione per il
bersaglio.

L'interazione legame-bersaglio è il primo step fondamentale, ma a volte
non è sufficiente, perché potrebbe essere necessario aggiustare il sito
di legame. Le SAR quantitative vengono utilizzate in questo caso.
Una entità chimica per poter essere considerata nello screening nello
sviluppo di un farmaco, deve avere proprietà consone ad essere un
farmaco.

Le proprietà in questione sono
\begin{itemize}
\item Acido/base
\item Redox
\item Energia di ionizzazione
\item Energia degli orbitali molecolari
\item Volume molecolare
\item Peso molecolare
\item $\log P$
\end{itemize}
Le più importanti proprietà sono acido/base, volume molecolare e peso molecolare e
$\log P$; queste caratteristiche servono per interagire con il bersaglio nel modo
corretto.

Queste proprietà vengono trasportate in modo matematico ad essere una
funzione. Quindi se la funzione assume un certo valore, allora la
molecola può essere considerata per lo sviluppo. Se non supera una certa
soglia vengono scartate, perché costa tempo svilupparle.
Il \(\log{} P\) può essere calcolato in diverse fasi organiche.

Non si guarda la molecola da sola, ma si deve pensare all'interazione
della molecola con il bersaglio

\fullpicture*{3_004}{Procedura per arrivare ad un candidato farmaco.}

\section{Hit compound}

Finora si è sempre parlato di lead, che è quella molecola che deve
essere ulteriormente funzionalizzata per le proprietà farmacocinetiche,
per diventare un candidato farmaco. Il lead è quella unità molecolare
che è ottimizzata, però deve essere ulteriormente ottimizzata per
diventare un candidato farmaco.

il lead è già il risultato di un processo che è durato anni. Possono
esserci più lead per uno sviluppo di un farmaco. Questo perché è
necessario avere più di una possibilità per avere un farmaco.

Il lead compound viene sviluppato a partire dal hit. Può essere
sviluppato partendo dai saggi in vitro, che sono stati ottimizzati
rispetto a quelli in vivo. Questo metodo è il metodo vecchio. Il metodo
vecchio può essere frutto di una scoperta casuale, oppure altre
strategie possono essere quelle di guardare l'attività di prodotti
naturali. La molecola naturale può essere estratta oppure sintetizzata,
se la disponibilità non è elevata; in questo caso la molecola è
sintetica, ma di origine naturale.

Un altra tecnica è il \emph{Drug Repurposing Strategy}, che avviene
quando si conosce una molecola che è già utilizzata in terapia, che però
viene utilizzata in un altra patologia. È necessaria l'intuizione per
capire quale molecola utilizzare e che cosa modificare per avere
l'attività desiderata.

Un altra strategia è quella di avere un farmaco con degli effetti
collaterali, che quindi interagisce con altri target. Si va quindi a
guardare l'effetto collaterale di un farmaco, in modo tale da poterlo
utilizzare per un altra patologia.

Si può anche lavorare sui metaboliti attivi. Il farmaco viene assorbito
e metabolizzato. A volte, il metabolita ha la funzione farmaceutica, ma
il farmaco assunto non la possiede. In questo caso, il farmaco viene
chiamato \emph{profarmaco}.

Infine si può utilizzare la progettazione razionale per costruire un
farmaco.
Questi sono degli approcci classici, che sono già utilizzati da molti
anni.

Gli approcci moderni utilizzano un composto precedente al lead compound,
ovvero l'hit compound, che tradotto in italiano è il candidato precoce.
È possibile che ci siano più hit compound che portano ad uno stesso
lead.

L'ottimizzazione che l'hit compound subisce è chiamata
\emph{ottimizzazione Hit-to-Lead}.

Per ottenere l'hit compound, ci sono diversi modi. Uno di questi è
l'\emph{HTS}, anche chiamato High Throughput Screening, che comporta
l'analisi di una vasta libreria di composti, che permette dfi avere una
prima discriminazione tra i composti che possono essere hit. Questa è
una tecnica sperimentale, che viene operata in vitro.

Si ha poi il virtual screening, che viene effettuato a livello
computazionale. Il numero di molecole che si possono analizzare è molto
grande.

Infine si ha il FBDD, o \emph{Fragment based screening}, che è basato sullo
studio dei frammenti piuttosto che sulla struttura della molecola. Si
parte dai frammenti, ovvero da entità che sono molto più piccole
rispetto ad una molecola, e che di per sé non possono essere utilizzate
direttamente in ambito farmaceutico.

\fullpicture*{3_005}{Fragment based screening.}

Le caratteristiche che deve avere l'hit compound è necessario avere una
attività in vitro.
Il legame e l'attività sono concetti differenti. Il legame è necessario
affinché ci sia attività, però se c'è solo il legame, non è detto che la
molecola abbia un attività. Quindi l'hit compound deve legarsi e deve
possedere un'attività farmaceutica.

L'attività farmaceutica deve essere già presente per concentrazioni di
50--100 \mu{}mol. Un altra caratteristica del hit compound è che deve
avere una struttura definita e caratterizzata
\ft{Non è detto che il composto sintetizzato sia puro, quindi è necessario verificare la struttura e la purezza del campione. È possibile che un intermedio di reazione possieda una piccola attività farmaceutica}.
È necessario riconoscere il peso molecolare e determinare la struttura
tramite l'uso di NMR, MS o altre tecniche.

Il grado di purezza ideale per un farmaco è 98 \%, sia quando il farmaco
è comprato, sia quando il farmaco è sintetizzato. In certi casi, ad
esempio quando le molecole sono sintetizzate in ambito accademico, si
può accettare anche una purezza del 95 \%.

La purezza può essere intesa anche come purezza enantiomerica. È
possibile che, come farmaco, si stia usando solo un enantiomero, oppure
la miscela racemica. A volte è meglio avere una miscela enantiomerica,
mentre altre volte è meglio utilizzare sono un enantiomero.

Le proprietà drug-like di un hit sono il \(\log{} P\) e il peso
molecolare. per ora non si considerano dei valori numerici, però ci sono
dei valori prestabiliti per decidere se si può usare un hit oppure no.

Le caratteristiche del lead sono leggermente differenti rispetto a
quelle dell'hit.

Le tecniche di screening elencate sono utilizzate sia a livello
industriale, sia a livello accademico.

\subsection{HTS}

L'High Throughput Screening è una procedura sperimentale dove si va a
fare uno screening di una grande quantità di molecole. Lo screening può
riguardare una classe di molecole (screening razionale) o molte classi
di molecole (screening randomico).

Questo screening permette di analizzare tanti composti. L'ordine dei
milioni di molecole si raggiunge solo a livello industriale. In questo
caso, lo screening viene fatto in modo automatizzato, per velocizzare la
procedura, con quindi una riduzione dei costi.

Per capire se c'è binding si utilizzano i fenomeni di fluorescenza. Si
può anche utilizzare il \emph{trasferimento di energia per risonanza},
che consente di redigere la fluorescenza da una molecola all'altra.

Lo screening viene fatto utilizzando delle piastre contenenti dei
pozzetti con la sostanza da analizzare. La fluorescenza viene scelta
perché è una tecnica molto sensibile.

Sono automatizzate anche le operazioni di analisi, sia del composto che
dei dati. Questo per avere un risparmio in tempo e denaro.

Alcuni sistemi possono essere Medium Throughput Screening, che
utilizzano dei saggi differenti. Questi saggi possono essere meno
sensibili e quindi è necessario più tempo per avere dei risultati
attendibili.

A livello accademico, non è presente l'automazione, caratteristica
invece dell'industria.

Le piastre sono già catalogate, quindi la macchina sa a quale segnale
corrisponde una determinata molecola. Viene introdotto un bersaglio
molecolare modificato con un gruppo cromoforo, che quindi possiedono un
segnale di fluorescenza. In seguito si fanno dei lavaggi e poi si
analizza la fluorescenza, per vedere dove il target si è legato con la
molecola. Il bersaglio può essere una proteina, un acido nucleico o
altro.

Le piastre possono essere utilizzate molte volte, quindi si può
utilizzare lo stesso schema di molecole per determinare il legame in
diversi bersagli. In questo modo, le possibilità di screening sono
ampliate in quanto è possibile analizzare più bersagli.

Con questi saggi si può anche determinare l'espressione genica. in
questo caso, nella piastra è presente del DNA immobilizzato. Si aggiunge
un composto, che è estratto da una cellula e, quando avviene il legame,
è presente il gene. Si può quindi indagare la presenza di un gene, che
può essere sovraespresso e sottoespresso e che quindi può essere causa
di alcune malattie.

Le molecole possono essere in soluzione o essere immobilizzate sulla
superficie della piastra. Questo consente di effettuare dei lavaggi,
però questo causa anche degli svantaggi. Ad esempio, l'interazione che
si instaura tra molecola e target potrebbe essere sottostimata, in
quanto parte della molecola è immobilizzata sulla superficie. Gli
svantaggi sono bassi rispetto ai vantaggi.

Questa strategia ha velocizzato molto lo sviluppo degli hit.

\subsection{Fragment based drug design}

Il fragment based drug design è basato sui frammenti di una molecola.
Questi frammenti sono delle entità molecolari chimiche molto più
piccole, che di per sé non verrebbero prese in considerazione per lo
sviluppo di un farmaco. I frammenti non hanno un'affinità molto elevata
per il target; per questo è necessario utilizzare delle tecniche
particolari, come l'NMR e l'XDR.

Queste tecniche utilizzano delle quantità di composto molto elevate. Si
può utilizzare una concentrazione della molecola, disciolta in solvente
organico, solitamente DMSO, molto elevata e questo consente di
analizzare dei frammenti che non sono molto solubili. Si va a vedere se
è presente l'interazione tramite l'NMR.{} Si costruisce la molecola a
partire dalle interazioni dei frammenti.

Un altro vantaggio è che è possibile modificare immediatamente il
frammento, in modo tale che abbia una affinità maggiore per il target.
Inoltre, i processi di sintesi per i frammenti sono più semplici.

Nel momento in cui gli screening vengono fatti attraverso l'NMR, si
possono ottenere delle informazioni strutturali di interazioni del
target, che possono essere utili per un secondo frammento.


Il linker deve essere ottimizzato per avere la giusta distanza per
consentire il legame tra i due frammenti.

I vantaggi di questa metodologia è che l'affinità è additiva, quindi il
frammento somma ha la somma delle affinità dei due frammenti
costituenti. Si ha una bassa affinità per un singolo frammento però, se
vengono sommate, formano un affinità elevata

\subsection{Virtual screening}

Il virtual screening viene eseguito in silico. È necessario conoscere
sperimentalmente la struttura del target. La struttura del target si può
ottenere dalla cristallografia a raggi X, dall'NMR o dalla cryo-em, che
è una spettroscopia elettronica di recente utilizzo.

Se si conosce la struttura del bersaglio molecolare, si può fare uno
screening basato sulla struttura del bersaglio e viene chiamato
\emph{Structure based virtual screening}. In questo screening si
utilizza un programma che simula l'interazione tra il target e una
molecola. Quindi si stimano le interazioni in silico.

Questo approccio ha uno svantaggio, ovvero non tiene conto della
presenza di acqua; questo approccio infatti dà una rappresentazione dei
possibili legami che si possono instaurare.

Gli algoritmi di docking generano possibili strutture tridimensionali
delle molecole che legano il sito di legame del target. Le strutture
tridimensionali sono classificate in accordo alle interazioni
composto-target.

Conoscendo la struttura del bersaglio, si può utilizzare un secondo
approccio, ovvero il \emph{De Novo drug design}, che è una sorta di
fragment based drug design in silico. Dal momento che si conosce la
struttura del target, si può trovare un frammento che inibisca il
target, in base alle interazioni che il frammento può fare con il
target.

Il nome \emph{De Novo} indica che questo metodo porta alla creazione di
nuove entità molecolari. Si crea in silico una serie di analoghi e, in
seguito, si sintetizza. Se la sintesi risulta complicata, allora è
possibile rivedere la molecola per cambiare alcuni frammenti.

Questo metodo utilizza il docking, in quanto si va a guardare le
interazioni che un target necessita per essere inattivato.

Se non si conosce la struttura del target, si può utilizzare un altro
metodo per fare un virtual screening, che viene chiamato \emph{Ligand
based virtual screening}. Partendo dai dati sperimentali, si può
ottenere un ligando differente. Si lavora su alcuni parametri, ad
esempio il \(\log{} P\), il peso molecolare e il volume molecolare.

Questo metodo consente di scremare tra un numero elevato di molecole, le
molecole che possono avere le caratteristiche necessarie per legare
meglio il target. Questo consente di ridurre le molecole che verranno in
seguito testate in vitro.

Da questi screening si ottiene un hit compound, che, in seguito alle
opportune ottimizzazioni, può diventare un lead compound.

\section{Lead compound}

Un lead compound è caratterizzato da una potenza definita, quindi vi è
uno step in più rispetto all'attività dell'hit compound. Il lead
compound deve legarsi al target, deve avere un'attività e deve averla
molto elevata.

Per definire la potenza, si guarda l'IC\ped{50}, che indica la
concentrazione in cui il 50\% di questa concentrazione è attiva. Più
l'IC\ped{50} è basso, più la molecola è attiva rispetto al target.

L'attività richiesta per un lead compound deve essere tra 1 e 10 nM,
quindi vi è un ordine di grandezza di differenza con l'attività del hit
compound. Se una molecola non è molto attiva, allora non viene portato
avanti il suo sviluppo.

Il lead compound deve avere un \emph{selettività} e una
\emph{specificità} ben definite. Per specificità si intende che la
molecola deve legarsi bene in un determinato punto del target. Per
selettività invece si intende che la molecola deve legarsi con un target
in modo esclusivo, oppure deve legarsi a più target che interessano la
patologia. Se c'è un interazione con altri target secondari, è
necessario che gli effetti collaterali sviluppati da questa interazione
non siano importanti.

Al giorno d'oggi è necessario conoscere anche il meccanismo d'azione del
lead compound, che viene definito \emph{MOA}, o a livello molecolare
\emph{MMOA}. Ad esempio, è necessario capire se la molecola va a legarsi
in modo covalente al target e se questo legame covalente è in qualche
modo reversibile. Per meccanismo d'azione si intende la tipologia di
interazioni chimiche che un farmaco crea con il target.

Il meccanismo d'azione di un farmaco non è sempre noto. Ad esempio, i
farmaci sviluppati con il metodo classico sono stati sviluppati
guardando l'effetto che il farmaco aveva e non in base al meccanismo
d'azione. Oggigiorno, i farmaci che sono stati messi in commercio senza
sapere il meccanismo d'azione sono stati caratterizzati e il meccanismo
d'azione è diventato noto.

Queste tre caratteristiche riguardano la
farmacodinamica\ft{La farmacodinamica è ciò che il farmaco fà all'organismo. Per farmacodinamica si possono intendere anche le modalità di legame che il farmaco ha rispetto al target.}
del farmaco, ovvero l'interazione del farmaco con il bersaglio.

Il lead compound deve essere ulteriormente ottimizzato per diventare un
candidato farmaco. L'ottimizzazione del lead compound viene fatta solo
su alcune molecole, non si tutte. Viene inoltre caratterizzato il
profilo farmacocinetico, facendo dei saggi in vivo. Questa parte
riguarda più un ambito farmacologico rispetto ad un ambito
chimico-farmaceutico.

I saggi in vivo vengono effettuati su animali viventi; questo è un male
necessario. Inizialmente, si utilizzano i topi come cavie. In seguito la
cavia dipende da che farmaco si sta studiando; non è detto che un cane
risponda meglio di una scimmia, o viceversa. Non c'è una sequenza di
sperimentazione animale ben definita.

Dopo aver caratterizzato la farmacocinetica, si iniziano a fare degli
studi farmacologici di assorbimento, di distribuzione, di metabolismo e
di escrezione (ADME). È necessario caratterizzare i metaboliti, in
quanto è possibile che siano tossici o che siano farmacologicamente
attivi. Questi studi vengono fatti inizialmente sulla cellula e, in
seguito, sull'intero organismo.

Negli studi ADME si guarda anche la formulazione del farmaco. La forma
preferita è una formulazione orale. Le molecole che inizialmente non
hanno le caratteristiche per diventare dei farmaci assunti per via orale
vengono scartati in questa fase. A volte, se ne vale la pena, si cambia
la formulazione per avere un farmaco iniettabile, o assunto in altri
modi.

Per un chimico farmaceutico, è necessario conoscere gli studi a cui il
lead è sottoposto, anche se non riguardano direttamente il suo lavoro.

Partendo dall'hit compound, si fanno dei cicli \emph{Relazione
Struttura-Attività}, che lo fanno arrivare al lead compound. Le SAR
(Structure-Activity Relations) permettono di identificare le parti di
una molecola che sono necessarie per avere un buon legame con il target.

Ognuno di questi round porta a capire la connessione tra una struttura e
una caratteristica ricercata, come la selettività o la specificità. Non
c'è un numero prestabilito di cicli effettuati. Questi cicli ricordano
il metodo vecchio.

I cicli sono caratterizzati da tre fasi. La prima fase è rappresentata
dai test in vitro di una molecola. La seconda fase è rappresentata
dall'analisi dei risultati ottenuti nei test e dall'interpretazione dei
risultati. In questa fase si va a capire cosa si può modificare. La
terza fase è rappresentata dalla sintesi della molecola modificata.
Questi cicli si ripetono fino ad una ottimizzazione della struttura del
farmaco.

\halfpicture*{3_006}{Cicli per l'ottimizzazione.}

\subsection{Ottimizzazione H2L}

Un lead compound è composto da diverse parti. La parte centrale viene
detta scaffold, che è una struttura rigida e planare, composta da
sistemi aromatici.

I sistemi aromatici spesso comprendono degli eterocicli. Questo perché
ci sono numerosi vantaggi nella sintesi, in quanto un eteroatomo
permette di aggiungere altri gruppi funzionali, che diventano utili
nello schema di interazione. In questo modo si può creare e modificare
le interazioni con il bersaglio.

I gruppi funzionali devono essere posizionati strategicamente e devono
avere delle caratteristiche tali per cui vi sia il legame con il target.

\fullpicture{3_007}{
Esempio di interazioni del farmaco con il
target. Lo spazio del sito catalitico è rappresentato a sinistra. Le
parti in giallo corrispondono alle parti che non hanno interazioni,
mentre le parti in grigio sono quelle dove sono presenti le interazioni
con il farmaco.}{fig:pag:60}

Nell'immagine{} \ref{fig:pag:60} si vede che inizialmente è presente un
frammento. L'IC\ped{50} è la più elevata rispetto alle strutture
possibili. In seguito, vi è uno sviluppo del farmaco e l'IC\ped{50}
passa da 200 \mu{}M a 0.031 \mu{}M. Via via che l'IC\ped{50} aumenta,
aumentano anche le interazioni del farmaco con il target. La conoscenza
della struttura del target è importante per questo tipo di sviluppo.

Inizialmente si occupa la parte sinistra della tasca; in seguito si
inizia ad occupare la parte destra della tasca. Si nota anche che sono
presenti degli alogeni, che fanno variare la polarità dell'anello,
mentre non cambia in modo sostanziale, il volume molecolare. Questa
sostituzione è molto comune, a livello farmaceutico, in quanto si va a
cambiare la polarità di una parte della molecola.

Via via che si modifica la molecola, si vede che cambia il \(\log{} P\)
e il peso molecolare. Si vede che, inizialmente, il \(\log{} P\) è
basso, però l'attività non è molto elevata. Nell'ultimo step, si vede
che il peso molecolare è aumentato, però è ancora accettabile, mentre il
coefficiente di ripartizione è aumentato in modo considerevole. In
seguito, si andrà a determinare quanto deve essere elevato il
\(\log{} P\) per essere considerato accettabile.

Il composto rappresentato in figura \ref{fig:pag:60} deve ancora essere
ottimizzato, almeno a livello di coefficiente di ripartizione, per poter
diventare un buon lead compound.

\fullpicture*{3_008}{Ottimizzazione Hit-to-Lead}

\clearpage

In seguito all'ottimizzazione H2L, bisogna ottimizzare il lead
ulteriormente, in modo tale che abbia le caratteristiche per diventare
un candidato farmaco. Inoltre deve risultare semplice da sintetizzare.

Una volta che si ha un candidato farmaco, inizia la fase IND, ovvero
\emph{Investigational new drug}, che consiste nella raccolta della
documentazione di un candidato farmaco per gli studi clinici.

\begingroup\herepicture{3_009}{0.8} \captionof{figure}{Fasi per la messa in commercio di un farmaco.}\endgroup

Nella prima fase clinica, si valuta la sicurezza del candidato farmaco.
Il test viene effettuato su un numero di pazienti molto basso. In questo
test, si valuta che gli effetti collaterali non siano molto importanti.
Questo test viene effettuato su pazienti sani.

Nella seconda fase clinica, si valuta l'efficacia del farmaco. In questa
fase, si effettua il test sui pazienti malati; in ogni caso, il numero
di pazienti è ridotto.

L'ultima fase viene effettuata su un numero maggiore di pazienti,
valutando l'efficacia su un campione maggiore di persone.

In seguito, si ha un altro step importante, ovvero NDA, che è l'acronimo
di \emph{New Drug Application}. In questa fase si ha una raccolta di
dati aggiuntiva, che permette di chiedere una valutazione del candidato
farmaco per la messa in commercio.

Supponendo che dopo un certo periodo il farmaco venga messo in
commercio, il controllo non è finito. Per gli anni successivi, viene
continuamente valutato. Questa fase è la \emph{farmacovigilanza} e serve
per avere un quadro più chiaro sugli effetti collaterali. Questa fase
viene chiamata anche \emph{follow-up}.

\fullpicture*{3_010}{Fasi che un lead compound subisce per poter diventare un
farmaco.}

\subsection{Imatinib}

Si prenda l'Imatinib come esempio di sviluppo del farmaco. L'imatinib è
stato uno dei primi farmaci ad essere sviluppato con i metodi moderni.

L'Imatinib è un farmaco antitumorale, utilizzato come trattamento per la
leucemia mieloide cronica. Questo farmaco è un inibitore della tirosin
chinasi. La tirosin chinasi è una proteina specifica che viene
sintetizzata da chi ha una determinata mutazione genetica. Questa
mutazione comporta la mutazione di un cromosoma, detto \emph{cromosoma
filadelfia}. Questo cromosoma è formato da una traslocazione di
materiale genetico dal cromosoma 9 al cromosoma 12. La traslocazione
porta alla formazione di un cromosoma sbagliato, che porta
all'espressione della proteina tirosin-chinasi BCR-ABL.{}

\herepicture{3_011}{0.8}

Le chinasi sono delle proteine che agiscono su un'altra proteina,
fosforilandola, utilizzando ATP.{} La proteina fosforilata è la forma
attiva della proteina sintetizzata.

\herepicture{3_012}{0.8}

\marginbox*{Le chinasi sono dei bersagli molecolari che sono molto studiati, in quanto sono promettenti da un punto di vista farmaceutico. Le proteine sono considerate dei target, in quanto sono l'espressione finale delle informazioni genetiche presenti nel DNA}

\fullpicture*{3_013}{
La tirosin-chinasi è formata da due subunità e viene prodotta
solo nelle cellule tumorali. L'Imatinib va a legarsi nel sito dell'ATP,
impedendo il binding con l'ATP. La chinasi smette di funzionare e la
proteina tirosin-chinasi smette di essere attivata e prodotta.
}

L'Imatinib, essendo specifico esclusivamente per questa chinasi, è molto
selettivo, quindi comporta pochi effetti collaterali. In ogni caso è
possibile che la proteina muti, quindi è necessario effettuare il
follow-up.

\marginpicture{3_013}{Scaffold di Imatinib}{fig:ImatinibScaffold}

Lo scaffold della molecola è rappresentato nella figura
{}\ref{fig:ImatinibScaffold}. Questo scaffold è stato ottenuto da uno
screening iniziale, che aveva come target un'altra chinasi. Quindi c'è
stata una certa casualità nello scoprire questo scaffold. In seguito,
determinando la struttura della tirosin-chinasi, si è potuto modificare
questo scaffold ad-hoc, in modo tale da avere le giuste interazioni. Lo
scaffold rappresenta l'hit compound.

In seguito, l'hit compound è stato ottimizzato ulteriormente,
introducendo una piridina.
In seguito, è stato aggiunto un linker. Il linker ha consentito al
farmaco di avere anche un'attività verso la chinasi target.

\herepicture{3_015}{0.8}

Per avere una selettività verso la chinasi-target è stato necessario
introdurre un metile. A livello di sintesi organica, non c'è una
differenza enorme nella struttura. Questa modifica, a livello chimico,
comporta la modifica della struttura tridimensionale del farmaco. La
modifica tridimensionale comporta una selettività maggiore verso la
chinasi.

Per aumentare la solubilità, a causa della presenza di anelli aromatici,
è stato necessario introdurre un altro sostituente
Questo approccio è stato utilizzato per sviluppare l'Imatinib. Lo
sviluppo è avvenuto a livello razionale.


Prima dello sviluppo di questa molecola, non erano disponibili dei
farmaci antitumorali specifici. Inizialmente, si utilizzavano degli
alchilanti del DNA oppure dei veleni delle topoisomerasi, che agiscono
su tutte le cellule, non solo su quelle tumorali. Si vede inoltre che
gli acidi nucleici possono essere considerati dei target dei farmaci,
anche se questo approccio comporta degli effetti collaterali maggiori.

\marginbox*{Le cellule tumorali hanno una vita più veloce rispetto alle cellule sane. Gli alchilanti del DNA vengono utilizzati in quanto agiscono più velocemente sulle cellule tumorali rispetto a quelle sane, però comunque agiscono anche su queste ultime.}

\subsection{Target}

Con lo studio della biologia molecolare, si ha sempre una più ampia
conoscenza dei target, che possono essere implicati in una patologia.
Nel 2000, si è visto che i target molecolari erano circa 500. Questo
numero però è in crescita.

Dal 2000 in poi, si è assistito alla rivoluzione genomica. In questa
rivoluzione, si dà importanza anche al genoma, ovvero a ciò che è
presente prima delle proteine. Questo consente di avere delle buone
prospettive per migliorare le terapie farmaceutiche.

\fullpicture*{3_016}{Prospetto dei potenziali target nell'era post-genomica.}

Originariamente, i target erano solo proteine. Da quando è stato
completato lo \emph{Human Genome Project}, è stato visto che i geni
presenti nel DNA non sono solo codificanti, ma sono presenti anche dei
geni non codificanti. Vi è il 98\% del DNA umano che non è codificante.
Il genoma umano presenta dei geni non codificanti, che possono essere
sovraespressi o sottoespressi in alcune malattie. Oggigiorno, i target
considerati sono sia il DNA, che l'RNA.{}

\marginbox*{Nel 2022, è stato messo in commercio il primo farmaco che ha come bersaglio l'RNA.}

Un farmaco deve poter agire su un target, ma non tutti i bersagli
associati ad uno stato patologico sono necessariamente adatti per
l'intervento del farmaco. Questo procedimento viene definito
\emph{validazione del target}.

La validazione del target va di pari passo allo sviluppo visto in
precedenza. È possibile che una molecola non agisca bene nei test in
vitro, o nei test in vivo.

\paragraph{Farmacogenomica}

Il DNA umano presenta tre miliardi di paia di basi. Due individui
indipendenti hanno circa una variazione di un paio di basi ogni mille,
quindi sono presenti circa un milione di differenze genetiche tra due
individui.

Le differenze genetiche possono comportare ad alcune variazioni di
risposta ai farmaci. Questo non è sempre una cosa negativa.

Nei gemelli omozigoti, il DNA di base è uguale; tuttavia l'ambiente
causa delle modifiche al DNA, che vengono chiamate variazioni
epigenetiche, che quindi comporta la variazione di espressione genica.

La \emph{farmacogenomica} è lo studio dell'interazione tra le
caratteristiche del farmaco e degli individui. Questo studio si rende
necessario, in quanto può esserci la variazione di alcuni marker
biologici, che possono modificare l'individuo a livello di
farmacocinetica e farmacodinamica.

La farmacogenomica ha lo scopo di individuare la miglior terapia per un
individuo o per una certa popolazione.

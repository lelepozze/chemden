\chapterpicture{header_05}
\chapter{Target del farmaco}

Si andranno a vedere le caratteristiche chimiche che aiutano a predire
le caratteristiche farmacodinamiche e farmacocinetiche.

La \emph{farmacocinetica} riguarda cosa fa l'organismo al farmaco, in
termini di assorbimento, distribuzione, metabolismo ed escrezione. La
\emph{farmacodinamica} riguarda cosa fa il farmaco all'organismo, in
particolare sulla macchina molecolare.

Inizialmente si considera solo il processo farmacologico, ovvero la
parte di farmacodinamica. Questo perché è più facile partire dal
presupposto che il farmaco sia già distribuito nell'organismo e inoltre
è necessario che il farmaco si leghi al target.

\fullpicture{4_001}{Il target solitamente è una proteina. Il target presenta
diversi siti; nell'immagine sono presenti solo i siti che interessano al
farmaco.}{fig:ProteinaTarget}

Nell'immagine {}\ref{fig:ProteinaTarget} si vede che è presente il sito
dove il farmaco andrà a legarsi. Questa zona si chiama \emph{zona
ortosterica}. Il farmaco possiede dei gruppi di legame, che vengono
impiegati per formare dei legami con la zona ortosterica

Il \emph{sito allosterico} può essere legato da una molecola e che
modula l'attività del target; il sito allosterico non è coinvolto
direttamente nel legame con il substrato.

Le interazioni tra farmaco e recettore sono solitamente non covalenti,
anche se esistono dei farmaci che legano il target in modo covalente,
come gli alchilanti del DNA o l'acido acetilsalicilico.

I due requisiti fondamentali affinché l'interazione si instauri, è che
sia presente una complementarietà sterica e una complementarietà
elettronica. Si ricordi che le i virtual screening non consentono di
visualizzare bene la formazione del
legame\ft{Anche le immagini utilizzate sono delle rappresentazioni del recettore e del legante, ottenute senza tenere conto della presenza di acqua.},
in quanto non conteggiano l'acqua presente.

Non tutti i farmaci si legano ai recettori. Ad esempio, gli antiacidi
non possiedono un interazione recettore-ligando, ma avviene
semplicemente una reazione di neutralizzazione.

\begin{quoting}
Corpora non agunu nisi fixata\\
(I corpi non agiscono se non fissati)\par
\raggedleft{--- Paul Ehrlich}
\end{quoting}

In primo luogo, è necessario avere un interazione per poter avere
l'azione di un farmaco. Quindi si è andati a studiare i processi
biologici per capire i meccanismi che sono dietro all'interazione
biologica.

\section{Recettori}

Un recettore è una macromolecola biologica responsabile di una risposta
biologica in seguito all'interazione con un farmaco. Questa descrizione
riguarda l'ambito della chimica farmaceutica.

\fullpicture*{4_002}{Recettori nella membrana fosfolipidica. Visto che i
recettori sono presenti nella membrana, il coefficiente di ripartizione
gioca un ruolo essenziale nella distribuzione del farmaco}

Le classi di recettori sono:
\begin{itemize}
\item
  \emph{Lipoproteine o glicoproteine}, spesso si trovano inserite nelle
  membrane plasmatiche o di organelli nel citoplasma.
\item
  \emph{Enzimi:} molti farmaci modificano in modo specifico l'attività
  di enzimi coinvolti in processi vitali per la cellula.
\item
  \emph{Acidi nucleici (DNA, RNAs):} alcuni antibiotici e antitumorali
  agiscono su acidi nucleici.
\item
  \emph{Lipidi}, come i lipidi di membrana.
\end{itemize}

La risposta biologica varia di caso in caso; dipende infatti dagli
effetti che il farmaco ha sul recettore.

In chimica biologica, gli enzimi vengono utilizzati come catalizzatori
biologici, mentre in chimica farmaceutica, gli enzimi sono dei target.
L'attività enzimatica viene modificata con il farmaco.

Le prime due classi di recettori sono proteine. Le proteine sono state
studiate per prime e le più studiate a livello farmaceutico. Le proteine
più studiate sono quelle che possiedono delle tasche dove accogliere il
substrato con il quale interagiscono. Esse sono le più studiate in
quanto sono le più semplici da studiare.

Recentemente, si è iniziato a studiare gli acidi nucleici e i farmaci
che possono essere utilizzati. Anche l'RNA, in quanto mediatore tra le
proteine e il DNA è studiato.

I lipidi hanno delle interazioni molto diverse da quelle con i farmaci.
Ad esempio, si può rompere la membrana tramite l'uso di tensioattivi.

La flessibilità delle proteine influenza il farmaco. Infatti, le
immagini che si vedono sono statiche, però le molecole a 37 °C in
ambiente acquoso non sono ferme. È per questo che la flessibilità di
entrambi i partner è molto importante. La possibilità di migliorare
questa interazione è importante per l'affinità.

Una proteina non esiste in una sola conformazione, ma esiste almeno in
due. In realtà, le conformazioni sono molte di più, e possiederanno
delle energie molto simili. Noi, per semplicità, distinguiamo solo due
conformazioni, ovvero quella \emph{attiva} e quella \emph{inattiva}.

Un ligando può andare a interagire con delle modalità differenti con il
bersaglio: questo comporta che ci siano tre modelli per il
riconoscimento molecolare. Nella \emph{selezione conformazionale}, la
piccola molecola sceglie di legarsi a una delle due conformazioni,
solitamente quella attiva. L'equilibrio viene spostato sulla forma che
viene legata dalla piccola molecola.

\herepicture{4_003}{0.5}

Nell'\emph{induzione conformazione} si ha un equilibrio di
conformazioni, però la piccola molecola, interagendo con il bersaglio,
induce il recettore a cambiare conformazione. Questa conformazione è
raggiungibile solo se il recettore è legato alla piccola molecola.
Nel riconoscimento\emph{lock-and-key:}, la proteina e la piccola
molecola non necessitano di cambi conformazionali per poter legare tra
di loro.

Quindi sono possibili diverse conformazioni del recettore, una attiva e
una inattiva. I \emph{meccanismi d'azione} si basano sul legame del
farmaco con le diverse conformazioni del recettore.

Un \emph{farmaco agonista} si lega preferenzialmente con la
conformazione attiva, quindi sposta l'equilibrio verso la conformazione
attiva. Il recettore quindi assume la forma attivata e quindi il target
sarà attivato. Un agonista induce una risposta positiva.
Un \emph{farmaco antagonista} si lega con entrambe le conformazioni,
quindi sia con quella attiva, sia con quella inattiva, però induce una
risposta negativa.
Infine, un \emph{farmaco agonista inverso} funge da agonista in quanto
si lega preferenzialmente alla conformazione inattiva, però inducendo
una risposta positiva.

\marginbox*{Il concetto di allosteria è quando si modifica un sito che non è il sito attivo della macchina molecolare, le cui variazioni comportano una variazione di attività della macchina molecolare}

Vi è un ultimo meccanismo, che è rappresentato degli \emph{inibitori
allosterici}, in quanto inibiscono l'attività della macchina molecolare,
senza agire sul sito catalitico, ma agendo sul sito allosterico, che
causa delle variazioni conformazionali che riguardano anche il sito
catalitico.

Si distinguono anche i \emph{meccanismi conformazionali} da
\emph{meccanismi cinetici}. I meccanismi conformazionali coinvolgono dei
cambiamenti di conformazione, in cui l'effetto biologico è causato dal
cambio di conformazione. I meccanismi conformazionali sono elencati in
seguito

Un \emph{agonista completo} si lega al posto del substrato e induce una
risposta positiva. Un esempio di agonista completo è la morfina, in
quanto si lega al suo recettore e causa una risposta positiva, ovvero
causa l'effetto analgesico.

\herepicture{4_004}{0.8}

Un \emph{agonista parziale} si lega al sito catalitico in modo parziale,
e induce una risposta positiva. La risposta è meno forte rispetto
all'agonista completo.

\herepicture{4_005}{0.8}

Un \emph{antagonista non competitivo} non si lega al sito attivo, ma
interagisce nel sito allosterico. Il substrato non si lega più, quindi
induce una risposta negativa.

\herepicture{4_006}{0.8}

Un \emph{antagonista incompetitivo} si lega in un sito a parte,
riducendo comunque l'attività della macchina molecolare, inducendo una
risposta negativa. Questo tipo di antagonista viene spesso utilizzato
quando la macchina molecolare è anche un poro. Ad esempio, nei canali
del calcio, ci sono dei siti che sono legati dal glutammato, che funge
da antagonista incompetitivo. Quando il glutammato si lega al canale,
blocca il passaggio degli ioni \ce{Ca^{2+}}.

\herepicture{4_007}{0.8}

Un \emph{modulatore allosterico}, invece, si lega al sito allosterico
della macchina molecolare. Questo comporta un cambio conformazionale,
che permette alla macchina molecolare di legare bene il substrato,
inducendo quindi una risposta positiva.

\herepicture{4_008}{0.8}

Il meccanismo cinetico, invece, si basa sulla cinetica di legame, non da
cambiamenti conformazionali. Nello schema sottostante, si ha il target,
il substrato e una piccola molecola, detta \emph{competitore}. Il
competitore compete con il substrato per formare il legame con il sito
attivo. In questo caso sarà presente un equilibrio tra il complesso
enzima-substrato e il complesso enzima-competitore; in base a questo
equilibrio si avrà una risposta più o meno negativa. La risposta sarà
sempre negativa in quanto il competitore andrà a legarsi al posto del
substrato, inibendo il funzionamento della macchina molecolare.

\herepicture{4_009}{0.8}

\marginbox*{Il legame covalente può essere irreversibile o reversibile.}

La dissociazione del competitore dal target è molto lenta, in quanto
l'affinità del competitore per il target è molto elevata. Solitamente,
la maggiore affinità è consentita dalla formazione di un legame
covalente.

\section{Riconoscimento molecolare}

Le molecole possono interagire con le \emph{interazioni classiche} della chimica biologica, quindi attraverso interazioni forti, come legami covalenti, e interazioni deboli, come legami ad idrogeno e legami di van der Waals.
C'è la possibilità di avere accettori e donatori di legami ad idrogeno.
Il sistema \pi{} può interagire con un sistema \pi{} o con un catione.
Il riconoscimento molecolare però utilizza altri tre concetti, ovvero i possibili \emph{equilibri}, la \emph{chiralità} e la \emph{rigidità} di una struttura.

Quindi, le caratteristiche di una molecola che influenzano la farmacodinamica sono:
\begin{itemize}
  \item Presenza di gruppi acido/base.
  \item Legami forti tra dipoli.
  \item Legami \pi.
  \item Legami deboli tra dipoli.
  \item Equilibrio tautomerico.
  \item Equilibrio conformazionale.
  \item Chiralità.
\end{itemize}

\marginpicture{4_010}{Istamina}{fig:IstaminaMolecola}

La molecola in figura \ref{fig:IstaminaMolecola} è l'istamina. Non è un principio attivo,
infatti causa la reazione allergica. Dal punto di vista interattivo, la
molecola serve e viene usata. La molecola funge da agonista per i propri
recettori.
L'istamina interagisce con due recettori e quindi ha due interazioni: è un allergenico e favorisce la secrezione gastrica.

L'analisi dal punto di vista farmaceutico è fatta per sviluppare degli
antiacidi. Conoscendo le interazioni, sono stati sviluppati dei farmaci;
stimando l'esistenza dei recettori, sono state sviluppate delle molecole
che hanno azione antiacida ma non allergenica.

Quando si guarda una molecola, bisogna fare l'analisi farmaceutica.
L'istamina presenta due tautomeri. Le differenze dei tautomeri sono la
posizione del doppio legame e la posizione dell'idrogeno. Dal punto di
vista farmaceutico, cambia l'interazione. Cambia l'interazione dei due
azoti. C'è un equilibrio tra le due forme. È necessario guardare anche
gli equilibri tautomerici che la molecola potrebbe avere. Non interessa
solo dal punto di vista della sintesi, ma dal punto di vista
interattivo.

\herepicture{4_011}{0.8}

Le molecole possono essere dei rotameri, quindi cambia l'orientazione di
un legame nella rappresentazione in 2D. Bisogna considerare anche le
rotazioni nella rappresentazione e nell'interazione della molecola.
Il linker serve da `ruler', per avere la spazialità corretta da
interagire con il target. deve anche avere una certa flessibilità.

\herepicture{4_012}{0.8}

La natura voleva l'istamina flessibile. Partendo da questa struttura, si
può sviluppare un farmaco simile.
Il legame ammidico non viene considerato rotante, e dal punto di vista
farmaceutico i legami terminali simmetrici non sono rotanti. Anche se il
legame ruota non cambia nulla.
I cicli danno un certo grado di stabilità.

Se il farmaco ha uno schema di interazioni, bisogna considerare tutte le
forme. Lo schema interattivo viene definito \emph{farmacoforo}.

\herepicture{4_014}{0.8}

\fullpicture*{4_013}{
I legami rotabili per l'istamina sono tre. Dal punto di vista
farmaceutico sono solo i primi due (a partire dal collegamento con
l'anello).
}

\clearpage

\subsubsection{Morfina}

La morfina è il capostipite di una serie di farmaci; è una molecola di origine naturale. La
morfina viene estratta, può essere anche sintetizzata, però non è la via
preferita.

Ci sono molti centri chirali, ce ne sono cinque (segnati con i legami
pieni o vuoti). Ci possono essere \(2^5\) stereoisomeri. La morfina è
descritta da questa serie di centri chirali.

L'azoto amminico non può donare il doppietto, ma può accettare un acido.
C'è anche un sistema \pi, che influenza gli ossidrili, che quindi saranno
più propensi a donare l'idrogeno.

La molecola è complessa. È un sistema policiclico complesso (anche in
3D, non è planare).

A livello di 3D, si può vedere che la molecola non ha possibilità di
ruotare. La rigidità della molecola può essere fondamentale come la sua
flessibilità, in quanto diverse molecole interagiscono con bersagli
diversi.

I cicli bloccano i gruppi funzionali, quando è necessario in modo che
siano disponibili per interagire con uno specifico recettore.

\marginbox*{La morfina è un agonista che dà effetto analgesico.}

Il compito del chimico farmaceutico è quello di ottenere una molecola
con un'azione simile, ma di entità chimica diversa. La molecola deve
poter far guadagnare la casa farmaceutica. Si guardano le relazioni
struttura-attività, per ottenere qualcosa che agisca allo stesso modo.

\fullpicture*{4_015}{Struttura tridimensionale della morfina.}

\subsection{Chiralità}

La terza componente è la chiralità. La molecola può avere delle
caratteristiche chirali, così come i bersagli della molecola. Sia
proteine che acidi nucleici sono chirali.
La miscela racemica può andare bene, ma a volte è necessario separare i due
stereoisomeri

\fullpicture*{4_016}{Chiralità.}

Ad esempio, due molecole che sono originarie dal carvone. I recettori
olfattivi sono chirali, perché reagiscono con due modi diversi a seconda
della chiralità della molecola. Si sentiranno due odori diversi.
Questo è per dire che le macchine molecolari possono subire diversi
effetti a seconda di che stereoisomero è presente.

\fullpicture*{4_017}{Esempio dell'effetto della chiralità su alcune molecole.}

Un caso celebre è la talidomide, che possiede due stereoisomeri: l'isomero R ha un effetto sedativo, mentre l'isomero S è teratogeno.

\section{Farmacoforo}

Le caratteristiche della molecola possono essere riassunte in \emph{caratteristiche elettroniche} e \emph{caratteristiche spaziali}, che permettono di far interagire la molecola con il bersaglio nel modo desiderato,
agendo da agonista o antagonista, a seconda dei casi.
Una sola modifica può dare attività diversa. Ad esempio, studiando
l'istamina si è arrivati a farmaci che hanno un'azione opposta, ovvero hanno un azione antiistaminergica.

\fullpicture*{4_018}{
Lo schema interattivo della molecola con il suo partner. Identifica il
collocamento dei gruppi funzionali nello spazio.
}

Le interazioni deboli, come le interazioni di van der Waals, il legame ad idrogeno sono molto importanti e consentono di avere interazioni specifiche.

Il farmacoforo deve essere ben definito, e deve essere di molecole con
accertata attività biologica. Serve per lo sviluppo di farmaci simili.

Quando si guarda lo screening di nuove molecole, anche senza conoscere
la forma/geometria del bersaglio.
Ad esempio, da uno studio in silico, si ottengono tre composti attivi.
Da questi si ottengono le interazioni con il target (colorate). Si
ottiene uno schema farmacoforo, in modo grafico.

Poi entra in gioco la fantasia del chimico farmaceutico; si va a vedere
cosa si può modificare nel farmaco. Ad esempio si può cambiare il gruppo
rosso, con un altro gruppo (di natura simile), o anche cambiare la
geometria dell'anello benzenico.

Questi sono definiti \emph{isosteri} gli atomi o gruppi di atomi che possono
essere interscambiati per caratteristiche chimiche. Questi gruppi possono essere donatore \pi,
donatori e accettori.
Se si cambia l'atomo con un isostero, potrebbe cambiare anche l'attività

I \emph{bioisosteri} invece sono gruppi interscambiabili che però non alterano
l'attività biologica.

Gli isosteri univalenti possono essere \ce{NH2}, \ce{CH3}, \ce{OH} , \ce{F}, etc.

\fullpicture*{4_019}{Isosteri}

\marginbox*{
Come visto in precedenza, lo scaffold è lo ``scheletro'' del farmaco. Le
strutture planari sono composte anche da eteroatomi.
}

Gli eteroatomi sono posizionati per due ragioni: posizionare nuovi
gruppi funzionali e possono essere a loro volta gruppi funzionali.

\marginbox{Cos'è un farmaco generico?}{
Il principio attivo è diverso, così come il nome.\\
Il farmaco generico non ha un nome commerciale, ma avrà il nome del
principio attivo. Cambia la formulazione (eccipienti), in modo tale da
non avere da pagare per il brevetto.\\
La variabilità genetica può comportare un aumento o una perdita
dell'efficacia del farmaco generico.
}


Quando si hanno eterocicli, si può fare un `tuning', che permette di
ottenere l'interazione migliore, o anche le caratteristiche chimico
fisiche che lo rendono più vicino ad un potenziale farmaco.
Avere gli eterocicli, o poterli scambiare mantenendo l'attività permette
di semplificare gli schemi di sintesi della molecola.
Gli eterocicli consentono di avere una nuova identità che non è sotto brevetto.

\section{Peptido-mimetici}

Per utilizzare i peptidomimetici è necessario modificare il gruppo per ingannare le proteasi, in modo tale
che non degradino il legame peptidico.

I peptidi sono considerati dei buoni esempi per creare un composto. Ad
esempio, le proteasi sono utilizzate come target; lavorano su substrati
molto specifici.
Il substrato è un punto di partenza per far interagire le proteasi in un
determinato modo. Non può essere lo stesso substrato. In primo luogo,
perché non sono stabili e perché verrebbero degradate immediatamente.
Non sono nemmeno disponibili, all'inizio.
Per creare un peptidomimetico, si va a sostituire un l'intorno del
legame peptidico, in modo tale da ingannare la macchina molecolare e
fare in modo che non sia il substrato della proteasi

Un esempio di peptidomimetico è il nuovo farmaco per il COVID-19, che si chiama
Nirmatrelvir.

\herepicture{4_020}{0.8}

Si introducono dei sistemi ciclici per la rigidità, piuttosto che
l'introduzione di gruppi polari, per aumentare la solubilità.
Partendo da uno schema di peptide, si vanno a modificare le parti, si
ottiene un peptidomimetico che ha una certa efficacia.
Questo farmaco va ad interagire con una proteasi, bloccandola. Il farmaco è selettivo solo per la proteasi virale, mentre lascia intatta la proteasi umana.

Una cosa importante del farmaco è che deve rimanere nella sua cavità. Si
potrebbero utilizzare dei legami covalenti, però questo può portare a
effetti collaterali. Quello che si preferisce è avere un legame
covalente reversibile.
Trovare una molecola che abbia queste caratteristiche è molto difficile.
Se il farmaco non è consono al resto dell'organismo, si può tornare
indietro.

La formulazione del farmaco, non c'è solo il principio attivo. Il contro
del peptidomimetico è che anche se è modificato, comunque va incontro ad
un metabolismo molto rapido. Ha una farmacocinetica molto veloce.
Si deve avere una composizione che contiene una agente preservante.

\fullpicture{4_021}{Esempio per lo sviluppo di un peptidomimetico}{fig:esempioProva}

Un esempio di sintesi è quello raffigurato nell'imamgine \ref{fig:esempioProva}. L'amminoacido iniziale vicino all'anello è detto linker. L'anello è un
anello antrachinonico.

\marginpicture*{4_023}{Target del peptidomimetico}

L'obiettivo del bersaglio era avere un RNA, che ha una struttura
dinamica, e che comporta la presenza di ``tasche'', zone di
configurazione dell'acido nucleico, che possono interagire con altro.
L'antrachinone è planare.

Questi possono inserirsi nelle zone dinamiche è posizionare le catene,
mascherando la struttura dell'acido nucleico.
Sono stati utilizzati diversi amminoacidi; è stato anche esplorato
l'amminoacido invertito, che a volte sono risultati più attivi.
L'amminoacido carico (dicatione) ha dato risultati migliori. La
lunghezza ideale per avere l'attività desiderata è stata ricercata.
Si possono fare altre cose nella SAR, però questo è un buon approccio.

Altri tre tipi di analisi sono:
\begin{itemize}
  \item
  L'omologazione, ovvero la variazione
  della lunghezza di certi gruppi funzionali. Questo ci da anche una
  variabilità di flessibilità, e di interazioni 
  \item
  Si può lavorare sulle
  catene laterali, e sulla ramificazione. Detta chain-branching
  \item 
  L'ultima
  cosa è quella di cambiare un anello con una catena o viceversa. Le
  differenze riguardano la flessibilità/rigidità della molecola. Dipende
  dal sistema che si cerca di interagire
\end{itemize}

Le molecole attive possono contenere i cicli; che possono essere
espansi, ridotti, etc.

\fullpicture{4_022}{Proguanil e Cicloguanil}{fig:esempioProva2}

In un altro esempio in figura \ref{fig:esempioProva2}, si vuole sviluppare un farmaco antimalarico, a partire da un farmaco già esistente, il Cicloguanil. Il farmaco sviluppato è il Proguanil.
L'anello non è ciclico. L'attività è molto bassa, però aumenta quando
gli enzimi ciclizzano il farmaco e lo rendono molto più attivo. Questo è
il concetto di pro-farmaco.

\section{Interazioni ligando-target}

Tornando all'interazione ligando-target, si vanno a vedere i tipi di
legame:
\begin{itemize}
  \item Legame ionico
  \item Legame ad idrogeno
  \item Interazioni di van der Waals
  \item Interazione dipolo-dipolo
\end{itemize}

I legami sono importanti perché possono avvenire tra il farmaco e il
target. I legami più utilizzati sono quelli deboli.

\paragraph{Legame ionico}
È un legame molto forte, e determina la prima interazione con il farmaco
del recettore.

\paragraph{Legami idrogeno}
Si ha un accettore e un donatore del legame a idrogeno. I legami possono
essere intra o inter molecolari. Possono favorire o non favorire
l'interazione con il ligando. Il legame prevede la direzionalità.
Bisogna anche guardare i gruppi donatori/accettori a seconda del pH.

\paragraph{Interazioni di van der Waals}
Sono interazioni a corto raggio, permettono l'interazione debole a
breve raggio.

Le interazioni possono essere anche repulsive. Anche queste devono
essere delle forze della giusta identità e direzionalità.

\fullpicture*{4_024}{
Grafico $pK_a$ delle ammine. Si nota il salto di $pK_a$ da ammine aromatiche a alifatiche. Il salto è in prossimità del ph fisiologico, 7.4.
La percentuale tra forma protonate è non protonate nelle ammine alifatiche è circa 30\%/70\%. Le
ammine aromatiche presentano l'equilibrio opposto con le stesse
percentuali. Le ammine sono molto presenti nei farmaci
}

\subsection{Acqua}

L'acqua è un elemento essenziale, perché l'ambiente è acquoso.
Infatti, l'acqua è il solvente del farmaco e dovrà stare bene. Influenza molto la
farmacocinetica dell'acqua. La struttura è caratteristica per
determinare se una sostanza sta bene o no nell'acqua.
L'acqua è fondamentale per la proteina. Per questo non ci si può
scordare l'acqua dall'insieme.

L'acqua ha un effetto, così come gli ioni, che può essere schermante per alcuni legami, come il legame ad idrogeno, o descermante per altri.
Per fare uno screening è necesario usare dell'acqua a concentrazione fisiologica ($\ce{[Na+]} = 150 \text{mM}$); non può usare l'acqua milliQ.
L'interazione di due molecole cambia con la forza ionica, anche l'acqua
stessa può mascherare i legami.

In prima battuta si sciolgono le molecole/proteine in DMSO, poi si può
sciogliere in acqua. È comunque difficile far sciogliere le sostanze in
acqua.
Non si può togliere dal computo, l'azione dell'energia libera,
dell'entropia e dell'entalpia.

\fullpicture{4_025}{
L'acqua va a interagire con il sito catalitico. In una
proteina, l'esterno della proteina è riempito da siti idrofilici.
Viceversa, all'interno ci sono amminoacidi più idrofobici, in quanto non
devono andare in contatto con l'acqua.\\
L'enzima avrà uno shell di idratazione diverso a seconda della sua
struttura; l'acqua deve togliersi dal sito di interazione.\\
Nella prima situazione, l'acqua si sposta fuori dal sito attivo, mentre nella seconda situazione, l'acqua si sposta all'interno del sito attivo.
}{fig:AcquaCheSiSposta}

Come si vede in figura \ref{fig:AcquaCheSiSposta}, si vede che l'acqua si deve spostare, quindi è richiesta energia per spostare l'acqua; l'energia richiesta per spostare l'acqua viene chiamata \emph{contributo entalpico}. L'acqua però è legata in una struttura in modo ordinato, quindi spostando l'acqua si aumenta il disordine del sistema. Questo contributo viene chiamato \emph{contributo entropico}.
Le due cose si devono bilanciare, per avere un buon legame. L'acqua da
quindi un doppio effetto, sia di solvatazione che di desolvatazione.
La legge che governa questo spostamento è
\[
\Delta G = \Delta H - T \Delta S
\]

Affinché ci sia una interazione, è necessario che avvenga una
desolvatazione. Se l'energia per la desolvatazione è maggiore
dell'energia dell'energia del legame, il legame non è stabile. Questo
avviene quando il legame è polare.
A volte una molecola molto polare non va bene, quindi si rimuove il gruppo per consentire
il binding del target con il substrato.

\fullpicture*{4_026}{
Esempio di aumento di affinità per il recettore. Si aumenta l'affinità di 2.1 rispetto al composto iniziale, se si aggiunge un gruppo rigido, ma con una catena flessibile.\\
Quando si aggiungono entrambe le terminazioni, non si ottiene la somma,
ma è qualcosa di più. Si ha quindi un'interazione maggiore rispetto alle
singole componenti.
}

\subsection{Efficacia e potenza}

Un farmaco è caratterizzato da due componenti:
\begin{itemize}
\item
  \emph{Efficacia}, ovvero l'entità dell'effetto generato
\item
  \emph{Potenza}, che è misura di concentrazione o di dose, per avere una determinata efficacia.
\end{itemize}

L'efficacia, o meglio l'efficacia massima, è un indice della risposta massima che un farmaco può produrre, quindi innalza l'altezza della curva dose-risposta.
La potenza è un indice di quanto farmaco deve essere somministrato per ottenere la risposta desiderata.
L'efficacia è una qualità importante per un farmaco, mentre la potenza, di solito, non lo è.

\fullpicture*{4_027}{Nell'immagine a sinistra, si comparano due farmaci in base all'efficacia. Si vede quindi che la miperidina ha un efficacia più grande della pentazocina.\\
Nell'immagine a sinistra, si comparano altri due farmaci, ovvero la miperidina e la morfina. Si vede quindi che per ottenere l'effetto analgesico, la miperidina richiede dosi più elevate rispetto alla morfina. Quindi la morfina è più potente rispetto alla miperidina.\\
Si noti che, se si somministrano dosi sufficientemente elevate, la miperidina può produrre lo stesso grado di analgesia della morfina.
}

\subsection{Meccanismo di reazione}

Un agonista può essere totale o parziale. Dipende se lega la forma
attiva o passiva. Se lega solo la forma attiva è totale, e si raggiunge
il plateau a una soglia di risposta più elevata
Un agonista inverso consente di avere una risposta opposta a quella
dell'agonista totale, ovvero provoca una risposta inibitoria.
L'antagonista ha una risposta inibitoria.

L'\(IC_{50}\) o \emph{concentrazione inibente} è la concentrazione di un inibitore enzimatico (farmaco, tossina o veleno, ecc.) necessaria per inibire il 50% del bersaglio in esame (enzima, cellula, recettore o microrganismo).
L'\(IC_{50}\) è perciò un parametro utilizzato per valutare l'efficacia di una sostanza nell'inibire il target ed è uno dei metodi comunemente usati nella ricerca farmacologica per misurare la potenza di un antagonista.
L'\(IC_{50}\) misura l'attività inibitoria ed è quindi una valutazione
farmacologica. Solitamente si vuole ottenere un inibitore.
Viene espressa secondo la seguente formula:
\[
IC_{50} = \biggl[1 + \frac{[\text{substrato}]}{K_M} \biggr] \cdot K_i
\]

\fullpicture*{4_028}{Meccanismi di reazione}
\chapterpicture{header_06}
\chapter{Farmacocinetica}

La farmacocinetica è lo studio dei processi che vengono effettuati dal corpo nei confronti
del farmaco.
Si guarda l'acronimo ADME. Si guarda che quindi il farmaco si
possa assorbire e distribuire. Non non guarderemo le fasi di metabolismo
e di eliminazione.

La parte di metabolismo serve per valutare i metaboliti ,che a loro
volta possono avere un'attività biologica.

\section{Solubilità}

La solubilità in questo caso è considerata in acqua. Parlare di acqua è un'approssimazione, in quanto il solvente
che si trova nelle cellule è molto più carico di ioni rispetto all'acqua
pura.

Si vuole guardare se dalla struttura si possono avere dei parametri
termodinamici che interessano la farmacocinetica.
Quindi si va a caratterizzare le piccole molecole in base alla
solubilità in acqua. La soglia minima è di 1 mg/mL.

Per guardare l'attività della molecola, bisogna guardare anche la
solubilità, perché anche da questa dipende l'attività. Si può anche
operare sulla solubilità per aumentare l'attività del farmaco.
Si può anche parlare di concentrazione, però è più difficile da
confrontare. Per questo si usano i mg/mL.

Ci sono tre categorie:
\begin{itemize}
  \item Molecole insolubili.
  \item Molecole poco solubili.
  \item Molecole solubili.
\end{itemize}

La categoria delle molecole insolubili è formata dalle piccole molecole strutturate in
screening. Avranno una particolare importanza perché se modificate,
potrebbero diventare più attivi.
Il pH di misura è 7.4, ovvero il pH biologico

\marginbox*{
Non sempre troppo solubile è bene, si vanno a limitare le interazioni
del farmaco con il recettore.
}

Per fare questo, si utilizza una formula che deriva da Dwayne Frisen. Si
sono considerate tutte le molecole che si conosce la solubilità in
acqua. Si va a guardare quindi la molecola nella forma neutra, in quanto
la parte neutra è la forma più importante, perché la forma ionica si sa
già che è insolubile.

Lui ha notato che tutti i farmaci insolubili hanno un rapporto
di $\nicefrac{C}{N+O}$ maggiore di 5. (Rapporto tra eteroatomi con
elettronegatività maggiore di quella del carbonio e carbonio).
Si guarda quindi la struttura e si ottiene una previsione di solubilità.
Se la molecola ha questo rapporto compreso tra 3 e 5 è leggermente
solubile, se è inferiore a tre è solubile.

\fullpicture*{5_001}{Formula di Dwayne Frisen}

Questa attenzione non si ha per eliminare i farmaci con una bassa
attività e solubilità, ma per modificare il farmaco, ad esempio a
sciogliersi meglio in acqua. Se però un farmaco è inattivo anche se è
solubile, allora il farmaco viene scartato.
Non è una regola vera al cento per cento, però c'è una deviazione del
2\%. La percentuale di errore è bassa e quindi per noi è una regola.

Quando si è sotto il limite della solubilità, si deve prestare ancora
più attenzione, perché l'instabilità può essere diversa. Un farmaco che
avrà 0.9 mg/mL è ben diverso da un farmaco che ha 0.09 mg/mL.
Si può pensare di aggiungere un idrossile, o un gruppo polare, per
consentire di aumentare la solubilità della piccola molecola.
Se non ci sono alternative, si utilizza quello che si ha, anche se non è
la soluzione ideale.

Questo metodo consente di categorizzare le piccole molecole. Per
l'istamina, ad esempio, il rapporto è pari a 5/3, ovvero ci si trova ad
un rapporto che consente alla molecola di essere solubile, sicuramente.
L'operazione è molto veloce, in quanto serve la formula bruta.
La categorizzazione serve per dare una priorità ad uno screening sulle
molecole da studiare.

Guardando l'istamina, si vede che è soggetta ad un equilibrio acido
base. La forma carica (protonata) è una forma molto solubile, ma anche
la forma non protonata (neutra) è comunque solubile (sopra la soglia di
solubilità).

Questa non è sempre la situazione attuale. Se l'istamina non avesse un
azoto, allora è debolmente solubile, se invece non avesse tutti e due
gli azoti, allora la molecola sarebbe proprio insolubile.

\section{Passaggio nelle membrane}

Le molecole, oltre ad essere solubili in acqua, devono essere in parte apolari, in quanto devono attraversare la membrana cellulare. Si parte dal presupposto che le
cellule siano composte da membrane fosfosfolipiciche.

Come si comporta un farmaco che è trascinato dal sangue vicino alla
membrana cellulare?
Se la molecola è troppo lipofila, resta intrappolata all'interno della
membrana. Questo può essere voluto o meno, in quanto potrebbe essere che
il target sia all'interno della membrana.

\marginbox*{Le catene alifatiche danno una certa idrofobicità.}

La molecola deve raggiungere il bersaglio, e non è detto che riesca a
passare una membrana, oltre la quale c'è la macchina molecolare target.
Si usa il \emph{coefficiente di ripartizione} (o meglio il logaritmo del
coefficiente di ripartizione).
Il solvente organico non è preso a caso, ma è uno specifico, che imita
la doppia membrana.

Il comportamento della molecola è dato da diversi parametri:
\begin{itemize}
\item
  Necessità di trasportatori
\item
  Possibilità di rimanere in membrana
\end{itemize}

Come si calcola il coefficiente di ripartizione ?
Si misura la concentrazione da una parte e dall'altra, in seguito alla
ripartizione. Si può usare l'UV-Vis, però c'è un limite, ovvero la
molecola deve poter assorbire.
In alternativa, si può usare una massa, perché è una tecnica veloce e
accurata.
Ci sono dei metodi computazionali che permettono di avere un'idea della
solubilità della molecola.

Il \(\log P\) sperimentale si va a calcolare se serve la determinazione
corretta.
Si parla di coefficiente di ripartizione come ripartizione di olio e
acqua. Sarà maggiore di uno, se il composto è più idrofobico, e minore
di uno se il composto è più idrofilico.
Come visto, si possono ``salvare'' i composti più idrofobici, che
presentano attività biologica.

\marginbox*{
Solubilità e \(\log P\) sono due misure termodinamiche diverse, che
insieme danno un'idea del comportamento termodinamico del composto.
}

La fase organica su cui si misura il \(log P\) non è una fase organica
qualunque. Si usa il \emph{n-ottanolo}, perché ricorda la forma del
fosfolipide. È un alcol costoso, sintetizzato con una sintesi complessa
e di difficile conservazione (tende a ossidarsi).

La catena è lunga il giusto per avere un alcol della giusta polarità. Se
è troppo polare, l'alcol si mescola all'acqua. Se è troppo lunga la
catena, l'alcol diventa apolare e l'alcol assume lo stato solido, che lo
rende non adatto alla misura termodinamica.
Quest'alcol inoltre è trasparente all'UV, quindi si utilizza anche per
questo. Il solvente non interferisce con la misura.

Lipinski ha guardato le caratteristiche delle piccole molecole, per non
essere assunte per via orale. La tabella viene letta con il senso
opposto, ovvero si guardano le caratteristiche per assumerlo per via
orale.

Due delle due caratteristiche più importanti sono il \(\log P\) e il
peso molecolare. Si ha la regola vista in precedenza (quanto vale il
\(\log P\)) per definire se il farmaco è un buon candidato per essere
somministrato per via orale.
Il \(log P\) deve essere minore di 5.

Sono state notate anche le caratteristiche di donatori di legami H e
accettori di legami di H. Queste caratteristiche servono per vedere
quali regole servono per sviluppare il farmaco. Devono essere presenti
tutte le condizioni; o in linea di massima una può mancare, però il
farmaco deve essere revisionato.
Anche questa regola è vera al 98\%. Si possono sempre avere delle
eccezioni.

Prima di quest'osservazione, non si eliminava un farmaco per queste
caratteristiche. Se il farmaco si presuppone che venga utilizzato in un
altro modo (assunzione diversa), si può pensare di infrangere queste
regole, altrimenti sono adamantine.
Dallo sviluppo Hit-to-lead, si vede che se i parametri sono troppo
diversi da quelli ideali, si va allo step precedente e si cerca un modo
di risolverlo.

Si va a relazionare il \(log P\) in funzione ai farmaci esistenti. Non
ci sono limiti inferiori, però ci sono quelli superiori.
Alcuni farmaci sono presenti alle estremità del grafico in figura \ref{fig:GraficoCancaroBis}. Sono molto
solubili o molto poco solubile, dipende dall'utilizzo del farmaco.

\fullpicture{5_002}{
Distribuzione del $\log{} P$ nei farmaci. Si è visto che, per avere dei composti che abbiano
una ragionevole probabilità di essere assorbiti per via orale, il loro $\log{} P$ non deve essere maggiore di 5.
La distribuzione è stata calcolata su più di 3\,000 farmaci in commercio.
}{fig:GraficoCancaroBis}

\marginbox*{
Un farmaco che non deve attraversare la membrana sarà molto polare, in
quanto non necessita l'interazione con la membrana.
}

Un farmaco può avere delle cariche, però deve avere una forma carica e
una non carica, legate da un equilibrio a pH fisiologico. Il composto
con carica ha l'azione farmacologica, però non riesce a passare la
barriera; grazie a questo equilibrio, la parte neutra però può
attraversare la membrana e grazie all'equilibrio, può ritornare alla
forma attiva.
L'equilibrio sarà spostato verso i prodotti, a prescindere dalla forma acida
o basica, perché si va a rimuovere solo una forma, quindi l'altra dovrà
cambiare per l'equilibrio che a sua volta passerà attraverso.
Se la molecola non passa la membrana, non riesce a trovare il suo
bersaglio.
\herepicture{5_003}{0.8}

Può essere che il composto sia sempre carico. In questo caso è
necessario un trasportatore. Mentre, se il composto è molto apolare, può
fermarsi all'interno.

Per avere un modello teorico, si è iniziato a raccogliere i \(log P\), di piccole molecole organiche,
in n-ottanolo. Il benzene è stato il primo composto.
È stato deciso che l'errore doveva essere approssimato alla seconda
cifra decimale.

Partendo dal benzene, si è andato a modificare la struttura della
molecola. Sono state analizzate quindi delle molecole simili, come il
bromo-benzene, il toluene, il nitro-benzene e il fenolo. Sono stati
ottenuti dei valori diversi di \(\log P\) rispetto al benzene.
Più aumenta il \(\log P\) e più la molecola è idrofoba; è vero anche il
contrario.

Quindi si sono iniziati a tabulare le differenze a seconda della
struttura. Si è andato a calcolare il contributo al \(log P\) dei
sostituenti, a seconda dell'idrofilicità o idrofobicità.
Con segno positivo, si determinano i composti più idrofobi, mentre con
segno negativo, si identificano i composti più idrofilici. La scala è
quindi in ordine della idrofobicità dei gruppi sostituenti.

Questo è stato fatto per una serie di molecole, e quindi sono state
determinate effettivamente le costanti per ogni gruppo. In particolare
sono state divisi i dati in base alla molecola, se aromatica o alifatica
(come ad esempio il benzene e il cicloesano). Si nota che ci sono delle
differenze.
Lo stesso conto si può fare anche per due sostituenti. Quindi vengono
verificati che i contributi per ogni gruppo siano effettivamente
additivi.
Si possono mettere a confronto i dati dei gruppi funzionali; però
l'incertezza sta nella seconda cifra decimale.

Si nota però che serve anche conteggiare la posizione del sostituente,
specialmente in un sistema aromatico. Si iniziano ad avere delle
difficoltà nello spiegare gli errori rispetto ai dati sperimentali.
Non si ha un sistema preciso, però il sistema può funzionare.
Prendendo quattro atomi di carbonio e disponendoli in una catena, si ha
un \(log P\) differente, rispetto alla sostituzione di un ter-butile.
Sicuramente ci sono delle differenze tra i due. Però i valori sono
simili (accettabili).

\begin{table}
\begin{tabular}{lcc}
Sostituenti & \pi{} atomatici & \pi alifatici\\
\ce{F} & 0.14 & -0.17\\
\ce{Cl} & 0.71 & 0.39\\
\ce{Br} & 0.86 & 0.60\\
\ce{I} & 1.12 & 1.00\\
\ce{OH} & -0.67 & -1.16\\
\ce{OCH3} & -0.02 & -0.47\\
\ce{SCH3} & 0.61 & 0.45\\
\ce{CN} & -0.57 & -0.84\\
\ce{COOH} & -0.28 & -0.67\\
\ce{COOCH3} & -0.01 & -0.27\\
\ce{COCH3} & -0.55 & 0.71\\
\ce{NH2} & -1.23 & -1.19\\
\ce{N(CH3)2} & -0.28 & -0.85\\
\ce{NO2} & -0.28 & -0.85\\
\ce{CH3} & 0.56 & 0.50\\
\end{tabular}
\caption{Costanti idrofobiche di alcuni sostituenti. La seconda colonna è stata calcolata in cicloesano.}
\end{table}

Si nota che questo sistema ha dei limiti, quindi si aggiunge un
coefficiente di correzione. Quindi viene calcolato questo coefficiente
in base alla posizione e in base alla presenza o meno di catena, o anche
in base a diverse geometrie della molecola (ad esempio anello
condensato) o presenza di legame H.

Questa correzione viene chiamata \emph{regola della costitutività}. Quindi
il \(log P\), calcolato da Hansch-Fujita è calcolato, come
\[
C \log{} P = \log{} P + \sum{} \pi_{X_i} + \sum{} \Delta{} \pi{}
\]

La previsione risulta affidabile; la retta di correlazione tra dati
sperimentali e calcolati è di 45 °, quindi si ha una buona correlazione.

\fullpicture*{5_004}{Retta di correlazione tra il $\log{} P$ sperimentale e il $\log{} P$ stimato}

Per calcolare il \(log P\), si va prima a stimarlo (calcolando), però si
va anche a determinarlo sperimentalmente.

Più recentemente, sono state utilizzate delle piastre con una membrana
(strumento PAMPA), e man mano che si vuole fare un esperimento, si va a
riempire un pozzetto. Si va quindi a riempire una parte donatore e una
parte accettore del composto. Si va quindi a vedere se nella membrana
passa qualcosa, e si va quindi a vedere la diffusione e la ripartizione
della molecola.

\fullpicture*{5_005}{Rappresentazione grafica semplificata del PAMPA. La membrana creata sul compartimento donatore addizionata della soluzione a concentrazione nota del composto viene messa a contatto con il tampone del compartimento accettore, permettendo l'eventuale diffusione passiva dal primo al secondo.}

Si utilizzano piccole quantità di reagente, e il limite di rivelabilità
è dato dallo strumento, non tanto dalla concentrazione.

Per membrana si può intendere anche la \emph{barriera ematoencefalica}, che è
composta da tre altre membrane. Se bisogna far passare un farmaco o al
contrario, se si vuole che non passi si può lavorare sulla molecola.

Oltre alle regole d Lipinski, per avere una sostanza che passi la
barriera emato-encefalica, servono dei parametri più restrittivi, ovvero è necessario avere:
\begin{itemize}
  \item \emph{Lipofilicità:} $0 \leq{} \log{} P\leq{} 3$
  \item Il \emph{peso molecolare} deve essere inferiore a 450 Da.
  \item \emph{Carica:} $4 \leq pKa \leq 10$
  \item Il numero dei \emph{legami ad idrogeno}, mediati da ossigeno e azoto, deve essere minore o uguale a cinque.
\end{itemize}

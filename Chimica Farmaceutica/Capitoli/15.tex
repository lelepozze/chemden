\cleardoublepage{}
\begingroup

\def\newpage{}
\part{Classi di farmaci}

\pagebreak

\thispagestyle{empty}
\vspace*{\fill}
Si inizia a guardare i farmaci da un punto di vista sistematico. Quindi
ora si guardano i farmaci dal punto di vista delle classi.
Ogni classe ha una sua caratteristica.

Inizialmente si va a guardare la fisiologia del problema che è alla base della sintomatologia, da un punto di vista farmaceutico, non farmacologico.
Quindi si vedrà il farmaco dal punto di vista della struttura. Non si
vedrà l'aspetto medico vero e proprio (con le relative dosi/assunzioni).

Si vedrà anche l'evoluzione delle classi. E anche si va a dare un occhio
ai farmaci più vecchi, già presistenti.
Si vedono anche le difficoltà che sono presenti nella classe.

Dai tre spazi (chimico, biologico e fisiopatologico), si va a guardare
cosa c'è da sistemare (biologico) e questo si fa con una molecola.
La molecola interagisce sicuramente con più di una macchina molecolare.
È possibile che ci sia un secondo target, però principalmente, si vedono
effetti off-target.

L'importante è che il rapporto beneficio/effetti collaterali sia sempre
favorevole. Non si possono trascurare gli effetti collaterali di un
farmaco.
\vfill

\endgroup
\documentclass{chemden}

%###########################################

\author{Pozzebon Daniele}
\title{Chimica Farmaceutica}
\edition{Prima Edizione}

%###########################################
\usepackage[backend=biber,style=verbose,sorting=ynt,citestyle=alphabetic]{biblatex}
	\addbibresource{Bibliografia.bib}
\usepackage{chemden-indice}

%###########################################
\begin{document}

\ifinput{cover} % copertina

\chapterpicture{header_01}
\frontmatter{}
\ifinput{frontmatter}


\mainmatter{}

% \ifinput{Capitoli/00} % non utilizzato

\ifinput{Capitoli/01} % Introduzione
\ifinput{Capitoli/02}
\ifinput{Capitoli/03} % Scoperta e sviluppo del farmaco
\ifinput{Capitoli/04}
\ifinput{Capitoli/05}

%\ifinput{Capitoli/06} % non utilizzato
%\ifinput{Capitoli/07} % non utilizzato
%\ifinput{Capitoli/08} % non utilizzato
%\ifinput{Capitoli/09} % non utilizzato
%\ifinput{Capitoli/10} % non utilizzato
%\ifinput{Capitoli/11} % non utilizzato
%\ifinput{Capitoli/12} % non utilizzato
%\ifinput{Capitoli/13} % non utilizzato
%\ifinput{Capitoli/14} % non utilizzato

\ifinput{Capitoli/15} % Classi di farmaci 
\ifinput{Capitoli/16}
\ifinput{Capitoli/17}
\ifinput{Capitoli/18}
\ifinput{Capitoli/19}
\ifinput{Capitoli/20}

\backmatter{}
\chapterpicture{header_01}
\begin{fullpaper}

%%% uncomment these lines for the bibliography
\chapterpicture{header_12}
\nocite{*}
\chapter*{Bibliografia}
\addcontentsline{toc}{chapter}{Bibliografia}
\markboth{BIBLIOGRAFIA}{BIBLIOGRAFIA}
\printbibliography[heading=none]

\cleardoublepage

\chapterpicture{header_13}

\printindex

\cleardoublepage

\begingroup
\pdfbookmark[0]{Note}{Note}
\NoteAFineLibro
\endgroup


\cleardoublepage
\thispagestyle{empty}
\includegraphics{empty}
\clearpage
\vspace*{18cm}
\begingroup
\phantomsection
\thispagestyle{empty}
\pdfbookmark[0]{Colophon}{Colophon}
\setlength{\parindent}{6pt} % Spazio di rientro per la prima riga
\setlength{\parskip}{3pt} % Spazio tra un paragrafo e l'altro
\colophon
\endgroup




\end{fullpaper} % backmatter

\end{document}


% Organizzazione
% 
% ========================================================================
%  
% - Risentire le registrazioni.
%   Se manca qualcosa, partire da lì a copiare.
% 
% - Riorganizzare il tutto nei capitoli già predisposti
%   
% - Cosa importante: inserire le immagini dopo ogni registrazione,
%   così da alleggerire il carico di lavoro.
% 
% - Le immagini non sono disordinate come quelle di chimica biologica,
%   però le slides non sono disposte nell'ordine giusto,
%   o a volte si riprende la medesima slide.
%   Quindi è necessario non usare la numerazione automatica, ma inserirle a mano.
% 
% - Scrivi gli appunti in Markdown, che è più veloce da utilizzare di LaTeX
% 
%
% ==========================================================================
% 
% Una bozza dei capitoli può essere:
%
% I) Introduzione
%     Farmaci
%     Chimica Farmaceutica
% 
% II) Scoperta e sviluppo del farmaco
%         HITS
%         Ottimizzazione HIT-to-LEAD
%     Target del farmaco
%         Recettori
%     Riconoscimento molecolare
%         Agonista
%         Antagonista
%         Agonista inverso
%     Rapporto struttura-attività (SAR)
% 
% III) Caratteristiche di un farmaco
%     Farmacocinetica
%     Farmacodinamica
% 
% IV) Classi di farmaci
%     Farmaci antistaminici
%     Farmaci antinfiammatori non steroidei (FANS)
%     Farmaci analgesici oppioidi
%     Farmaci antitumorali
%     Farmaci ansiolitici

Esercizio possibile del compito pag 3-41

C'è anche una banda intensa a 1715 cm-1, all'IR

Quindi c'è la presenza di un carbonio carbonilico. Si ha almeno una
insaturazione, il grado di insaturazione è 1.

Uno dei 7 carboni è sp\ap{2}.

Tutti gli altri carboni sono alifatici, non c'è nessun'altra
insaturazione.

L'integrale è in rapporto 6:1. In realtà il segnale è 12:2, in quanto il
numero di carboni è pari a 14.

12 protoni del gruppo metilico (4 gruppi metilici), 2 protoni del gruppo
\ce{C-H}, (non metilenico).

Non si vedono idrogeni molto deschermati, quindi è un chetone.

La molecola è simmetrica, quindi si devono avere tre carboni a lato per
ogni parte del chetone.

I protoni vanno identificati con lettere. Per i protoni equivalenti si
usa la stessa lettera. I protoni del ch3 sono a, i protoni del gruppo ch
sono b.

Dall'NMR si può ottenere un rapporto molare, non in massa. A seconda dei
segnali - pag 3-37

In questo esempio, l'integrale è 1:1, però i rapporti dei due composti
sono 1:2 \ce{CHCl3} e \ce{CH2Cl2}.

Si ottiene quindi il rapporto molare, che è 2:1, ovvero il cloroformio è
in quantità doppia rispetto al diclorometano.








Spettro del pa-nitrotoluene.

Effetti dei sostituenti. Qual è il protone più deschermato? Il protone c
(vicino a \ce{NO2}), è deschermato in quanto \ce{NO2} toglie densità elettronica
per risonanza. Quindi i protoni c sono i più distanti dallo zero, più
deschermati.

I doppietti in questione sono ``segnali sporchi'', nel senso che
presentano altri segnali di bassa intensità.


Come esempio si ha lo spettro in figura 5-8.

Vedendo la struttura della molecola, ci si aspetta di vedere i protoni
aromatici e quelli alifatici (due parti)

Aromatici: si trovano sopra 7 ppm e ci si aspetta di avere un segnale
con 10 protoni. I segnali sono almeno tre: doppietto per due protoni
doppietto di doppietti (o tripletto) per due protoni doppietto di
doppietti (o tripletto) per un protone

Se si guarda lo spettro, non si vede questo, perché i segnali si
sovrappongono. Però si vede il segnale per dieci protoni aromatici.

Poi si vede per il metile a sx, si vede un doppietto, accoppia con i
protoni amminici.

Il protone amminico darà un singoletto, per un protone.

I protoni a dx sono diastereotopici, in quanto si ha un centro chirale.
Quindi di indica che sono A e A'

Il segnale sarà un doppietto, con una costante geminale che integra per
un protone.

Viceversa, l'altro protone darà un doppietto, sempre con una costante
geminale.

Nello spettro ingrandito, si vede che i segnali aromatici sono un
multipletto. Quindi si indica

7.27-7.18, m, 10 H

Per il sistema AB, diastereotopici, si vede che si ha un multipletto.
Assomiglia a un quartetto, però la distanza tra i rami, si vede che la
distanza non è uguale. Ma accoppia il primo con il secondo e il terzo
con il quadro. Si può dire che

3.6 ppm, ABq 2 H \ap{6}J = 12 Hz

Chemical shift, quartetto AB, accoppiamento geminale e costante J

Il protone amminico si trova come un singoletto a 1. La forma allargata
è dovuta che questo protone può essere scambiato

Esempio 5-7

Si ha un dietere. Ci sono cinque segnali a, b, c, d, e.

Il segnale a è un singoletto con 3 protoni Il segnale b è un tripletto
con 2 protoni Il segnale c è un tripletto con 2 protoni Il segnale d è
un tripletto con due protoni Il segnale e è un tripletto con due protoni

Il segnale più a dx è il segnale a.

Il segnale a 3.82 e 3.49 sono segnali del primo ordine. Si può misurare
J.

Con i due segnali al centro, non si sa J, perché è un sistema di secondo
ordine. Questo è causato in quanto il chemical shift è molto simile.

Si dirà che da 3.7 a 3.65 e da 3.6 a 3.54 si hanno due sistemi AB
(multipletti) con due protoni





5-12 - 1-bromo-4-clorobenzene

Quindi si vede che passando da uno strumento a 60 MHz a uno a 400 MHz si
vede che gli spettri passano da secondo a primo ordine, però i segnali
sono segnali sporchi, in quanto i nuclei non sono magneticamente
equivalenti.

Prendendo la p-benzo-aldeide, si vede la stessa cosa

Questo effetto si vede anche nell'orto sostituzione 5-11

Quindi si vede che se si lavora con uno strumento ad alto campo, si può
dire che si hanno due doppietto di doppietti (400 MHz)

pag 5-10

Spiegazione della equivalenza chimica e equivalenza magnetica.







Il benzidrolo, sint. in laboratorio, si può vedere il protone
ossidrilico.





Dopo si indicano i segnali, come ppm, intensità, molteplicità, costante
di accoppiamento.

Gli spettri sono NMR, IR e MS.

Data la formula bruta, si calcola il grado di insaturazione. Per questa
formula, il grado di insaturazione è pari a 5.

Dalle insaturazioni si guarda se può esserci un ciclo.

Poi si cerca di disporre gli atomi, in base allo spettro IR e NMR.

In questo caso, quattro insaturazioni sono dovute all'anello aromatico,
ne rimane una che si ipotizza sia dovuta alla presenza di un carbonile,
in quanto all'IR, si vede un picco a 1760 cm\ap{-1}. Si può anche assegnare
il picco direttamente ad un cloruro acilico, in quanto questo ha un
segnale spostato verso cm\ap{-1} più elevati.

Da qui si assegnano i successivi atomi; si ottiene un benzene con un
cloruro acilico e due atomi di cloro attaccati all'anello.

Per vedere la posizione degli idrogeni nell'anello, si guarda lo spettro
dell'NMR.

Spettro a 90 MHz

ppm 8.179 Chemical shift = 8.167 ppm d Area = 1 8.156 J =

7.996 d 7.973 Chemical shift 7.902 J1 = 7.879 J2 =

7.637 7.543

Dalla costante J, si calcola la disposizione degli idrogeni; le costanti
orto forti, le costanti meta sono meno forti, le costanti para sono
piccole. È difficile determinare la struttura, a questo punto.

Per capire lo ione molecolare, bisogna guardare l'isotopo del cloro
(alogeno) più importante. In questo caso è il 35Cl.

La massa isotopica è 208. Il picco più esterno è a 208.

Si prova a stimare la molecola con una perdita di cloro. Si vede quindi
che il cloro perso è del cloruro acilico.

Il secondo picco più abbondante è a 145; questo si ottiene per perdita
di CO (perdita di 28).

Pag 60, a quale di queste tre molecola appartiene lo spettro? Si vede
che c'è una simmetria, quindi la molecola è la terza, si dovrebbero
avere tre segnali (due )

Data del parziale: non ad aprile, perché c'è laboratorio. La validità è
di un anno solare. La data è da decidere.


\section{NMR bidimensionale}

Non viene sempre visto. Cos'è NMR bidimensionale.

In realtà si può vedere lo spettro in tre dimensioni. Si hanno due
frequenze (o ppm) e nel terzo asse si hanno le intensità.

Lo spettro viene visto dall'alto, perché è difficile visualizzarlo. Il
piano di visione è quello delle due scale dei ppm.

Normalmente si vede un grafico come 2-66.

Quando si legge un esperimento bidimensionale, ogni picco è definito da
due spostamenti chimici. Il picco si dà come due coordinate, una sul
primo spostamento chimico e una sul secondo spostamento chimico.

Si vede che c'è una correlazione tra i due segnali. Che relazione c'è?
dipende dall'esperimento.

Si può fare una correlazione in base alla costante J, come \ap{3}J. Ma si può
anche fare la correzione come vicinanza nello spazio (meno di 3
Amstrong).

Per questo si chiamano picchi di correlazione, e scegliamo noi che
correlazione c'è.

Quanti esperimenti ci sono bidimensionali? Molti; nella tabella sono
viste le tecniche più comuni.

La correlazione può essere accoppiamento scalare o accoppiamento
dipolare.

Si possono avere esperimenti omo correlati, che vedono la correlazione
tra nuclei uguali, ad esempio due protoni. La correlazione può essere
l'accoppiamento scalare, il più comune è il COSY, che va a vedere
l'accoppiamento \ap{6}J e \ap{3}J.

Un esperimento molto simile è il TOCSY, che utilizza la correlazione
scalare, ma non si limita ad accoppiamenti a corta distanza, ma consente
di vedere anche accoppiamenti \ap{5}J e \ap{6}J.

L'esperimento NOESY e ROESY danno correlazione in nuclei che non
accoppiamo, ma sono vicini allo spazio.

L'HMQC vede l'accoppiamento tra due atomi, accoppiamento scalare. Si usa
solitamente per \ap{1}H e \ap{13}C. Quando si vede una correlazione, si vede che
esiste una correlazione tra di loro. Ad esempio, ci interessa sapere che
protoni sono legati ad uno specifico carbonio. Quindi si cerca
l'accoppiamento 1J tra carbonio e idrogeno. Un'altra variante di questo
esperimento è l'HMBC.

In ultimo, si ha un esperimento HSQC, che vede anche accoppiamenti a
lunga distanza, quindi si riescono a vedere \ap{6}J tra carbonio e idrogeno.

Questo è importante per capire la struttura di un composto. Leggendo
questi spettri, si può capire come sono legati gli atomi, facendo una
correlazione alla volta.

\subsection{COSY}

L'esempio visto oggi è il COSY, che viene utilizzato per la
caratterizzazione dell'aspirina.

Lo spettro COSY ha questo aspetto 2-69.

Si vede che ci sono dei picchi che formano una diagonale. Non danno
informazione, perché l'idrogeno accoppia con sé stesso. Lo spostamento
chimico è lo stesso.

Invece, i picchi che si trovano al di fuori della diagonale danno
informazione strutturale. Il COSY è simmetrico rispetto alla diagonale,
quindi si può leggere sia da una parte che dall'altra.

I picchi appaiono doppi in quanto si va a vedere lo spostamento chimico
degli ``stessi atomi'' di idrogeno.

Nel bordo dello spettro si mette il corrispondente spettro
monodimensionale. (solo le frequenze)

Nello spettro si vedono 6 segnali, da a a f. Da questi sei segnali, 2
sono singoletti (CH3 e CH3); per questi segnali non si devono vedere
picchi di correlazione.

Mentre ci si aspetta di vedere gli altri picchi, perché questi
accoppiano tra di loro. Sono segnali in rosso, verde e blu, nello
spettro.

I due singoletti danno segnale nella diagonale, ma non in altre aree.

Per gli altri, se si può fare, è quella di partire dai segnali già
identificati.

Ad esempio, si cerca il segnale del metile a, che è un doppietto;
accoppia con b.

Si vede nello spettro che il segnale a è presente fuori dalla diagonale,
quindi si vede con cosa accoppia.

Per fare in modo veloce, si parte dal segnale a dx; e si vede che
accoppia con un altro segnale ,che a sua volta accoppia con un terzo
segnale. Si ottiene una scala.

Esercizi; si hanno gli spettri 1D \ap{1}H e \ap{13}C, ma si hanno anche gli
spettri 2D accoppianti \ap{1}H-\ap{1}H e \ap{1}H-\ap{13}C.

\vfill
\pagebreak